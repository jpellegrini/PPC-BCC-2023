\documentclass{article}
\usepackage[brazil]{babel}
\usepackage[utf8]{inputenc}
\usepackage{fontspec}
\usepackage{float}
\usepackage{longtable}
\usepackage{tikz}
\usetikzlibrary{shapes,snakes,positioning,calc}
\usepackage{amsmath,amssymb}
\usepackage{makecell}

\setmainfont[
    BoldItalicFont=calibri-bold-italic-2.ttf,
    BoldFont      =calibri-bold-2.ttf,
    ItalicFont    =calibri-italic.ttf]{calibri-7.ttf}

\usepackage{xcolor}
\usepackage{sectsty}
%\sectionfont{\color{blue}}
\sectionfont{\MakeUppercase}
%\subsectionfont{\color{blue}} 
\subsectionfont{\MakeUppercase} 
\usepackage{geometry}
 \geometry{
 a4paper,
% total={170mm,257mm},
 left=30mm,
 right=30mm,
 top=25mm,
 bottom=25mm,
 }
\usepackage{hyperref}
\hypersetup{colorlinks=true,
linkcolor=black,
urlcolor=blue}
\urlstyle{same}

\usepackage{footmisc}

\usepackage{sectsty}
\usepackage{titlesec}

\usepackage{indentfirst}
\usepackage{hyperref}
\setlength{\parindent}{.5cm}
\setlength{\parskip}{1em}
\renewcommand{\baselinestretch}{1.5}

\usepackage{natbib}
\usepackage{graphicx}
\usepackage{multirow}
\usepackage{tocloft}

\renewcommand\cftsecleader{\cftdotfill{\cftdotsep}}

\setlength\parindent{0pt}

%\newcommand{\disciplina}[6][yellow!20]{
%\node [draw=black, fill=#1, very thick, rectangle, rounded corners, inner sep=03pt, %inner ysep=3pt, text width=#5pt, minimum height=40pt, text centered, align=left] (#4) at (#2,#3) {

%   \begin{minipage}{#5pt}
%   \linespread{1.0}\selectfont
%       \centering
%       \scriptsize{#6}
%   \end{minipage}
%};
%}

\newcommand\T{\rule{0pt}{5.6ex}}       % Top strut
\newcommand\B{\rule[-1.2ex]{0pt}{0pt}} % Bottom strut

\newcommand{\disciplina}[6][yellow!20]{
\node [draw=black, fill=#1, very thick, rectangle, rounded corners, inner sep=03pt, inner ysep=3pt, text width=\the\numexpr #5-6pt, minimum height=40pt, text centered, anchor=west] (#4) at (#2,#3) {

   \begin{minipage}{\the\numexpr #5-6pt}
   \linespread{1.0}\selectfont
       \centering
       \scriptsize{#6}
   \end{minipage}
};
}


\begin{document}
\begin{tabular}{rp{.5\textwidth}r}
 \multirow{2}{*}{\includegraphics[scale=.03]{imagens/brasao.png}} & \centering \Large{Ministério da Educação} & \multirow{2}{*}{\includegraphics[scale=2.3]{imagens/ufabc-logo.png}}\\
     & \centering \Large{Universidade Federal do ABC} &\\
\end{tabular}

\vspace{9cm}

\begin{center}
\Large{PROJETO PEDAGÓGICO DO CURSO DE \\\textbf {BACHARELADO EM CIÊNCIA DA COMPUTAÇÃO}}
\end{center}

\vspace{9cm}

\begin{center}
    SANTO ANDRÉ \\ 2021
\end{center}

\newpage

\begin{tabular}{l}
\textbf{Reitor da UFABC}\\
Prof. Dr. Dácio Roberto Matheus\\
\\
\textbf{Pró-Reitora de Graduação}\\
Profa. Dra. Fernanda Graziella Cardoso\\
\\
\textbf{Diretora do Centro de Matemática, Computação e Cognição}\\
Profa. Dra. Tatiana Lima Ferreira\\
\\
\textbf{Coordenação do Curso de Bacharelado em Ciência da Computação}\\
Prof. Dr. Jerônimo Cordoni Pellegrini - Coordenador\\
Prof. Dr. Gustavo Sousa Pavani - Vice-coordenador\\
\\
\textbf{Equipe de Trabalho}\\
Prof. Dr. Alexandre Donizeti Alves\\
Prof. Dr. Aritanan Borges Garcia Gruber\\
Profa. Dra. Carla Lopes Rodriguez\\
Profa. Dra. Carla Negri Lintzmayer\\
Prof. Dr. Carlos da Silva dos Santos\\
Prof. Dr. Denis Gustavo Fantinato\\
Profa. Dra. Denise Hideko Goya\\
Prof. Dr. Emílio de Camargo Francesquini\\
Prof. Dr. Francisco de Assis Zampirolli\\
Prof. Dr. Fabricio Olivetti de França\\
Prof. Dr. Gustavo Sousa Pavani\\
Prof. Dr. Jerônimo Cordoni Pellegrini\\
Prof. Dr. João Marcelo Borovina Josko\\
Prof. Dr. Marcio Katsumi Oikawa\\
Prof. Dr. Maycon Sambinelli\\
Prof. Dr. Paulo Roberto Miranda Meirelles\\
Prof. Dr. Thiago Ferreira Covões\\
Prof. Dr. Vladimir Emiliano Moreira Rocha\\




\end{tabular}



\newpage

\tableofcontents

\newpage
\section{Dados da Instituição}
\label{sec:dados_instituicao}

\textbf{Nome da Unidade}: Fundação Universidade Federal do ABC (UFABC).

\textbf{CNPJ}: 07.722.779/0001-06

\textbf{Lei de Criação}: Lei nº 11.145, de 26 de julho de 2005, publicada no
Diário Oficial da União (DOU) em 27 de julho de 2005, alterada pela Lei nº
13.110, de 25 de março de 2015, publicada no DOU em 26 de março de
2015\footnote{Disponível em:
\url{http://www.ufabc.edu.br/a-ufabc/documentos/lei-de-criacao-da-ufabc}.
Acesso em 12 de dezembro de 2022.}.


\newpage
\section{Dados do Curso}
\label{sec:dados_curso}

\textbf{Curso}: Bacharelado em Ciência da Computação.

\textbf{Diplomação}: Bacharel(a) em Ciência da Computação.

\textbf{Carga horária total do curso}: 3.280 horas.

\textbf{Tempo esperado de integralização}: cinco anos.  

\textbf{Tempo máximo de integralização}: dez anos. Devido a características
específicas da UFABC, o cálculo do tempo mínimo depende do bacharelado
interdisciplinar de acesso. Ver maiores detalhes na resolução
correspondente\footnote{Disponível em:
\url{https://www.ufabc.edu.br/images/consepe/resolucoes/resolucao_166_-_desligamento.pdf}.
Acesso em 12 de dezembro de 2022.}.

\textbf{Regime de Ensino}: presencial.

\textbf{Estágio}: o estágio curricular não é obrigatório.

\textbf{Trabalho de conclusão de curso}: obrigatório no último ano do curso.

\textbf{Turnos de oferta}: matutino e noturno.

\textbf{Número de vagas por turno}: 70 (total de 140 vagas anuais).

\textbf{Câmpus de oferta}: Câmpus Santo André

\textbf{Endereço}: Av. dos Estados, 5.001 - Bairro Bangú. Santo André
- SP. CEP 09.280-560.

\textbf{Atos legais}: Lei de criação da UFABC. Resolução ConsEPE nº
195 de 2015 e Resolução ConsEPE nº211 de 25/10/2016 que aprovam
revisões do projeto pedagógico. Curso reconhecido pela Portaria MEC nº
406, de 11 de outubro de 2011, publicada no DOU em 14 de outubro de
2011.


\newpage
\section{Apresentação}
\label{sec:apresentacao}

A Fundação Universidade Federal do ABC (UFABC) é uma fundação pública criada
pela Lei nº 11.145 de 26 de julho de 2005, sancionada pelo Presidente da
República e publicada no Diário Oficial da União (DOU) em 27 de julho de 2005 e
alterada pela Lei nº 13.110, de 25 de março de 2015, publicada no DOU em 26 de
março de 2015.
É uma instituição de ensino superior, extensão e pesquisa, vinculada ao
Ministério da Educação (MEC), com sede e foro no Município de Santo André,
situada na Avenida dos Estados, 5001, bairro Santa Terezinha, Santo André, CEP
09280-560, no Estado de São Paulo e com limite territorial de atuação
multicampi na região do ABC paulista, nos termos do Artigo 2º da lei
mencionada.

A UFABC possui autonomia administrativa, didático-científica, gestão financeira
e disciplinar, rege-se pela legislação federal que lhe é pertinente, pelo
Regimento dos Órgãos da Administração Superior e das Unidades Universitárias e
pelas Resoluções de seus Órgãos.

A instituição foi criada para atender a um anseio antigo da região do ABC
paulista por uma universidade pública e de qualidade.
A UFABC busca ser reconhecida como uma referência no panorama nacional e
internacional, por meio de sua atenção às demandas regionais, produzindo
pesquisas e formando profissionais de alta qualidade para enfrentá-las.
Sua missão é \textit{``facilitar e induzir a interdisciplinaridade, promovendo
a visão sistêmica e a apropriação do conhecimento pela sociedade, sem
esmorecimento da rigorosa cultura disciplinar''}.
Para esse propósito, a UFABC procura ter um olhar voltado para o mundo e, ao
mesmo tempo, procura caminhar lado a lado com a sociedade e o setor produtivo.

Nesse propósito, a atuação acadêmica da UFABC se dá através de cursos de
graduação, pós-graduação e extensão, visando a formação e o aperfeiçoamento de
recursos humanos solicitados para o progresso da sociedade brasileira.
Além disso, a instituição promove e estimula a pesquisa científica, tecnológica
e a produção de pensamento original no campo da ciência e da tecnologia.

Todos os alunos de graduação da UFABC ingressam por meio de algum curso
interdisciplinar:
\begin{itemize}
    \item Bacharelado em Ciência e Tecnologia (BC\&T);
    \item Bacharelado em Ciências e Humanidades (BC\&H);
    \item Licenciatura em Ciências Naturais Exatas (LCNE); ou
    \item Licenciatura em Ciências Humanas (LCH).
\end{itemize}
Além desses, são ofertados 20 cursos de bacharelado específicos e 5 cursos de
licenciatura específicas.



\subsection{O curso de Bacharelado em Ciência da Computação da UFABC}

O curso de Bacharelado em Ciência da Computação (BCC), previsto no Projeto
Pedagógico Institucional da UFABC (PPI-UFABC), faz parte do planejamento global
da instituição, que tem entre seus objetivos tornar-se um pólo produtor de
conhecimento de nível nacional e internacional no âmbito das ciências, cultura
e artes.

O BCC é de grande relevância dentro da UFABC: foi, nos anos entre 2018
e 2022, um dos mais procurados por alunos do Bacharelado em Ciência e
Tecnologia. Além disso, o corpo docente do BCC tem grande interseção
com os de outros cursos (Bacharelado em Ciências Biológicas;
Bacharelado em Biotecnologia; Engenharia da Informação; Bacharelado em
Matemática; Bacharelado em Física; Engenharia Aeroespacial; Engenharia
de Gestão; Licenciatura em Matemática; e Bacharelado em Neurociências).

O BCC está sediado no câmpus Santo André e iniciou seu funcionamento a partir
do Edital de Vestibular ocorrido em 02 de maio de 2006, publicado no DOU,
Seção 3, nº 83, 3 de maio de 2006, pg.\ 25.

O BCC tem a duração mínima de cinco anos, podendo ser concluído em prazo menor
a depender do desempenho do aluno e do regime de matrículas da UFABC.
A duração máxima do curso é de dez anos, conforme a Resolução ConsEP nº 166,
de 8 de outubro de 2013.
Deve-se atentar ao prazo máximo de 18 quadrimestres para integralização do
BC\&T, conforme Resolução ConsEPE nº 166, de 8 de outubro de 2013.

A admissão no BCC pode ser realizada por discentes que estão cursando ou já
concluíram o BC\&T.
As disciplinas sugeridas na matriz curricular do BC\&T podem ser cursadas
paralelamente às disciplinas sugeridas na matriz curricular do BCC.
Apesar disso, a colação de grau no BCC está vinculada à colação de grau no
BC\&T, de modo que o aluno que desejar colar grau no BCC já deve possuir o grau
de Bacharel(a) em Ciência e Tecnologia.
A colação de grau de ambos os cursos também pode ser realizada de forma conjunta.

Além de garantir aos egressos uma sólida e abrangente formação em Ciência da
Computação por meio de suas disciplinas obrigatórias e de opção limitada, o
curso se compromete com atividades complementares à sua formação, tais como
monitoria acadêmica e iniciação científica.

Este projeto pedagógico baseia seu conteúdo na integração dos seguintes
documentos reguladores:
\begin{itemize}
    \item PPI-UFABC de 2017;
    \item Projeto pedagógico do BC\&T (2023);
    \item Diretrizes curriculares nacionais para os cursos de graduação da Computação (2016);
    \item Referenciais de formação para cursos de Graduação em Computação (2017);
    \item Plano de Desenvolvimento Institucional (PDI) da UFABC (2013-2022).
\end{itemize}

No terceiro quadrimestre de 2010, formou-se a primeira turma do BCC e, em março
de 2011, a comissão designada pelo INEP/MEC emitiu parecer favorável ao
reconhecimento do curso, atribuindo ao mesmo o conceito máximo 5 (cinco).
Na aplicação do Exame Nacional de Desempenho de Estudantes (Enade) realizado em
2021, o curso também obteve conceito máximo 5 (cinco).


\newpage
\section{Perfil do Curso}
\label{sec:perfil_curso}

O BCC da UFABC propõe formar profissionais com carácter interdisciplinar e
multidisciplinar, com formação teórica consistente e vivência prática que
permita contribuir para o desenvolvimento científico e tecnológico da Ciência
da Computação, atuando profissionalmente em empresas de tecnologia, em pesquisa
científica ou em ações empreendedoras.

Além de uma formação básica sólida e uma proposta de desenvolvimento ético e
científico, o curso promove fortemente uma construção interdisciplinar, em
consonância com PPI-UFABC.
Os egressos do curso podem atuar em nível regional, nacional e internacional,
atendendo à crescente demanda por profissionais qualificados nas diversas áreas
em que a Ciência da Computação pode atuar.

A Computação está presente na rotina da população em praticamente todas as suas
atividades sociais, econômicas e científicas.
Podemos facilmente identificar a influência de algoritmos e recursos
computacionais em diversas atividades comuns, tais como ler notícias,
comunicar-se com outras pessoas, viajar, trabalhar, estudar, etc.
Dispositivos computacionais estão presentes em eletrodomésticos, veículos,
telefones celulares, televisores e computadores, entre outros.
A Ciência da Computação é certamente uma das áreas de futuro mais promissor,
abrindo várias oportunidades de desenvolvimento tecnológico e alimentando
iniciativas empreendedoras que buscam soluções para problemas gerais e
específicos da sociedade.
A demanda por profissionais é reconhecidamente alta e com tendência de
expansão, necessitando de cursos de formação que contribuam para atender de
forma qualificada a essa perspectiva de crescimento.

A estrutura curricular do BCC se baseia em vários documentos de referência:
\begin{itemize}
    \item Diretrizes Curriculares Nacionais (DCN) dos cursos da área de
    Computação;
    \item Proposta curricular das associações:
    \begin{itemize}
        \item ACM (\textit{Association for Computing Machinery});
        \item IEEE-CS (\textit{IEEE Computer Society});
        \item SBC (Sociedade Brasileira de Computação).
    \end{itemize}
\end{itemize}

O BC\&T, curso de ingresso de todo aluno do BCC, contribui com a formação
básica e divide-se em seis eixos didáticopedagógicos estruturantes:
\begin{itemize}
    \item Estrutura da Matéria;
    \item Energia;
    \item Processos de Transformação;
    \item Representação e Simulação;
    \item Informação e Comunicação;
    \item Humanidades.
\end{itemize}
Assim, aliada ao BC\&T, a estrutura curricular do BCC abrange diversas áreas de
formação.


\subsection{Justificativa de oferta do curso}

A UFABC está localizada na região conhecida como ABC Paulista, apelido que faz
referência às cidades de Santo \textbf{A}ndré, São \textbf{B}ernardo do Campo e
São \textbf{C}aetano do Sul, e também faz parte da região metropolitana de São
Paulo (RMSP).
A RMSP é altamente urbanizada (98\%) formada por 39 municípios e uma população
próxima de 22 milhões de habitantes (2021), que a faz figurar entre as dez mais
populosas do mundo\footnote{Dados disponíveis em
\url{https://perfil.seade.gov.br/}. Acesso em 14 de dezembro de 2022.}. 

Do ponto de vista econômico, a RMSP é considerada o maior pólo de riqueza do
Brasil, com PIB per capita no valor de R\$ 56.649,03 (2018).
A atividade econômica está fortemente ligada à prestação de serviços (85,5\%),
embora o setor industrial também tenha relevância (14,3\%), sendo grande a
contribuição do ABC Paulista. 
Do ponto de vista educacional, é uma região em que mais da metade (57,5\%) da
população jovem entre 18 e 24 anos possui, no mínimo, o Ensino Médio completo
(censo 2010).
É também uma região com grande número de escolas e faculdades, públicas e
privadas. 

A Computação é uma das áreas mais promissoras em termos crescimento e
desenvolvimento.
Praticamente todos os setores utilizam recursos computacionais para automatizar
tarefas, desenvolver produtos, otimizar a utilização e monitoramento de
recursos, inovar, planejar políticas de expansão, controlar atividades, etc.
Durante a pandemia de COVID-19, foi uma das poucas áreas que apresentou
crescimento e permitiu que muitas atividades econômicas e sociais pudessem ser
preservadas, apesar das dificuldades e restrições sanitárias.
Segundo levantamento da Associação Brasileira das Empresas de Tecnologia da
Informação e Comunicação (Brasscom) realizado em 2021, a demanda não atendida
por profissionais no Brasil deve atingir 420 mil vagas até 2024, sendo que
formam-se aproximadamente 46 mil por ano.

Nesse contexto, o ABC Paulista pode ser visto como uma região estratégica para
o apoio ao desenvolvimento tecnológico local e nacional.
O ABC é uma região com forte participação industrial, conurbada em uma área com
forte demanda por serviços.
Além disso, é uma região com alto índice educacional, integrada à RMSP e ao
Brasil por meio de grandes rodovias, grandes aeroportos, ferrovias e o porto de
Santos, o maior da América Latina.
É uma região estratégica para implantação de empresas nacionais e
internacionais, de diversos setores sociais e econômicos.

A Computação é uma das áreas de conhecimento mais presente e influente na vida
de empresas e pessoas.
Encontramos técnicas, teorias, produtos e metodologias associadas à Ciência da
Computação em diversas iniciativas empresariais, políticas, sociais e
tecnológicas.
A busca por profissionais qualificados é uma necessidade de diversas entidades
que buscam inovação, otimização de recursos, pesquisa e desenvolvimento.

Outra característica da Ciência da Computação é sua aplicabilidade, capaz de
contribuir com diversas áreas de conhecimento, o que lhe garante alta
capacidade interdisciplinar e integradora.
A implantação do BCC, sob essa visão, é naturalmente identificada aos
princípios norteadores da UFABC e às necessidades das comunidades local,
regional e nacional.



\newpage
\section{Objetivos do curso}

\subsection{Objetivo geral}

Formação de profissionais com perfil multidisciplinar e sólido conhecimento científico e tecnológico na
área de Ciência da Computação, capazes de atuar em áreas de desenvolvimento, pesquisa,
gestão ou consultoria.

\subsection{Objetivos Específicos}

\begin{itemize}
    \item Incentivar o perfil pesquisador do estudante, visando promover o
    desenvolvimento científico e tecnológico da Ciência da Computação;

    \item Preparar o estudante para atuar profissionalmente em organizações,
    com espírito empreendedor e com responsabilidade social;

    \item Proporcionar atividades acadêmicas que estimulem a
    interdisciplinaridade, bem como a aplicação e renovação dos conhecimentos e
    habilidades de forma independente e inovadora, nos diversos contextos da
    atuação profissional;

    \item Formar estudantes que possam estar em sintonia com a nova realidade e
    necessidade do aprendizado contínuo e autônomo, exigido pela sociedade do
    conhecimento e organizações dos dias atuais;

    \item Promover no estudante uma postura ética e socialmente comprometida de
    seu papel e de sua contribuição no avanço científico, tecnológico e social
    do País.
\end{itemize}

Com base nesses objetivos, pode-se definir que o(a) bacharel(a) em Ciência da
Computação da UFABC deverá conhecer os fundamentos de sua ciência, suas raízes
históricas e suas interligações com outras ciências.



\newpage
\section{Requisito de acesso}
\label{sec:acesso}

\subsection{Formas de acesso ao curso}

O processo seletivo para acesso aos Cursos de Graduação da UFABC é anual, e
inicialmente realizado pelo Sistema de Seleção Unificado (SISU), do MEC, onde
as vagas oferecidas são preenchidas em uma única fase, baseado no resultado
do Exame Nacional do Ensino Médio (ENEM).

O ingresso nos cursos de formação específica, após a conclusão dos cursos
interdisciplinares, se dá por seleção interna, segundo a Resolução ConsEPE, nº
256 de 23/06/2022.
Sendo assim, o ingresso ao BCC é realizado após o ingresso no BC\&T.

Existe ainda a possibilidade de transferência, facultativa ou obrigatória, de
alunos de outras Instituições de Ensino Superior (IES) para o BCC.
No primeiro caso, mediante transferência de alunos de cursos afins, quando da
disponibilidade de vagas, através de processo seletivo interno (art. 49 da Lei
nº 9.394 de 1996 e Resolução ConsEPE nº 254 de 08/06/2022); para o
segundo, por \textit{transferências ex officio} previstas em normas específicas
(art.\ 99 da Lei 8.112 de 1990, art.\ 49 da Lei 9.394 de 1996 regulamentada
pela Lei 9.536 de 1997 e Resolução ConsEPE nº 10 de 2008).

\subsection{Regime de matrícula}

Na UFABC, o ano letivo regular é constituído por 3 (três) quadrimestres,
definidos conforme calendário acadêmico lançado anualmente.
O processo de matrículas em disciplinas é conduzido de forma unificada pela
Pró-Reitoria de Graduação (Prograd) da UFABC.
Antes do início de cada quadrimestre letivo, cada aluno(a) deve solicitar a sua
matrícula, indicando as disciplinas que deseja cursar no quadrimestre
correspondente.
O período de matrícula é determinado pelo calendário da UFABC definido
anualmente pela Comissão de Graduação.

A matrícula de alunos ingressantes é realizada de forma automática e
obrigatória, obedecendo à matriz curricular do curso interdisciplinar de
ingresso.
A partir do quadrimestre letivo seguinte, o(a) aluno(a) entra no regime de
matrícula regular, obedecendo ao procedimento citado anteriormente.

Por não apresentarem pré-requisitos, todas as disciplinas podem ser solicitadas
livremente e a qualquer momento no processo de matrícula.
Apesar disso, deve-se ressaltar que cada disciplina possui uma lista de
recomendações, que expõe disciplinas que desejavelmente deveriam ter sido
cursadas anteriormente.
Embora não exista o bloqueio formal do pré-requisito, é importante que cada
estudante considere a lista de recomendações como um elemento orientador que
busca auxiliar o cumprimento bem-sucedido da matriz curricular.

É essencial ressaltar que o número de vagas e turmas é limitado, e o
preenchimento de vagas na matrícula segue os critérios de seleção adotados pela
Prograd.
Em casos particulares (como em disciplinas de Trabalho de Conclusão de Curso ou
de Estágios), os pedidos de matrícula são ainda analisados pela coordenação do
BCC, que poderá autorizá-los, ou não, dentro de seus critérios de adequação e
viabilidade pedagógica.

É importante ainda que o(a) estudante observe os critérios de permanência do
curso e jubilação (desligamento), regulados pela Resolução ConsEPE nº 166, de
8 de outubro de 2013.


\newpage
\section{Perfil do egresso}
\label{sec:perfil_egresso}

O BCC baseia-se em dois conjuntos fundamentais de documentos para a composição de sua proposta pedagógica e curricular e formação do perfil de egresso:
\begin{itemize}
	\item Diretrizes curriculares nacionais (DCNs) para cursos na área de Computação (ver Seção \ref{subsec:fund_legal});
	\item Projeto pedagógico institucional da UFABC (PPI).
\end{itemize}

Em relação às DCNs da área de Computação, a estrutura curricular se orienta nas exigências quanto ao "Perfil Geral de Egressos de Cursos na Área de Computação" e o "Perfil Específico de Egressos de Cursos de Bacharelado em Ciência da Computação", listados a seguir. 

O BCC trabalha a formação de seus egressos para que:
\begin{itemize}
	\item possuam sólida formação em Ciência da Computação e Matemática que os capacitem a construir aplicativos de propósito geral, ferramentas e infraestrutura de software de sistemas de computação e de sistemas embarcados, gerando conhecimento científico e inovação;
	\item desenvolvam visão global e interdisciplinar de sistemas e entendam que esta visão transcende os detalhes de implementação dos vários componentes e os conhecimentos	dos domínios de aplicação;
	\item conheçam a estrutura dos sistemas de computação e os processos envolvidos na sua construção e análise;
	\item dominem os fundamentos teóricos da área de Computação e como eles influenciam a prática profissional;
	\item sejam capazes de agir de forma reflexiva na construção de sistemas de computação, compreendendo o seu impacto direto ou indireto sobre as pessoas e a sociedade;
	\item sejam capazes de criar soluções, individualmente ou em equipe, para problemas complexos caracterizados por relações entre domínios de conhecimento e de aplicação;
	\item reconheçam o caráter fundamental da inovação e da criatividade e compreendam as perspectivas de negócios e oportunidades relevantes.
\end{itemize}

    Além disso e de forma mais ampla, o BCC trabalha seus egressos para desenvolver a capacidade de: 
\begin{itemize}
	\item identificar problemas que tenham solução algorítmica;
	\item conhecer os limites da computação;
	\item resolver problemas usando ambientes de programação;
	\item tomar decisões e inovar, com base no conhecimento do funcionamento e
	\item características técnicas de hardware e da infraestrutura de software dos sistemas de computação consciente dos aspectos éticos, legais e dos impactos ambientais decorrentes;
	\item compreender e explicar as dimensões quantitativas de um problema;
	\item gerir a sua própria aprendizagem e desenvolvimento, incluindo a gestão de tempo e competências organizacionais;
	\item preparar e apresentar seus trabalhos e problemas técnicos e suas soluções para audiências diversas, em formatos apropriados (oral e escrito);
	\item avaliar criticamente projetos de sistemas de computação;
	\item adequar-se rapidamente às mudanças tecnológicas e aos novos ambientes de trabalho;
	\item ler textos técnicos na língua inglesa;
	\item empreender e exercer liderança, coordenação e supervisão na sua área de atuação profissional;
	\item realizar trabalho cooperativo e entender os benefícios que este pode produzir.
	
\end{itemize}


De forma complementar, o BCC trabalha sua estrutura curricular e suas ações em conformidade com as três principais políticas institucionais previstas no PPI da UFABC: interdisciplinaridade, excelência e inclusão social. Por meio das iniciativas institucionais da UFABC, o aluno possui ampla liberdade para complementar sua formação em diversas áreas de conhecimento trabalhadas pelos cursos da universidade. Com suas várias oportunidades de integração com iniciativas de pesquisa e extensão, os alunos também encontram oportunidades para participar de projetos de pesquisa científica e tecnológica, além de acesso a equipamentos e técnicas avançadas de pesquisa moderna. Utilizando diversas iniciativas de inclusão social, a UFABC, dentro de suas possibilidades orçamentárias e legais, busca ampliar seu alcance para a comunidade local, promovendo ações que buscam democratizar o acesso ao ensino superior, compartilhar os resultados de iniciativas científicas, e abrir oportunidades para alunos em situação de vulnerabilidade.

Com isso e em consonância com as DCNs da área de Computação, também é papel do curso garantir que seu egresso seja dotado:

\begin{itemize}
	\item de conhecimento das questões sociais, profissionais, legais, éticas, políticas e humanísticas;
	\item da compreensão do impacto da computação e suas tecnologias na sociedade no que concerne ao atendimento e à antecipação estratégica das necessidades da sociedade;
	\item de visão crítica e criativa na identificação e resolução de problemas contribuindo para o desenvolvimento de sua área;
	\item da capacidade de atuar de forma empreendedora, abrangente e cooperativa no atendimento às demandas sociais da região onde atua, do Brasil e do mundo;
	\item de utilizar racionalmente os recursos disponíveis de forma transdisciplinar;
	\item da compreensão das necessidades da contínua atualização e aprimoramento de suas competências e habilidades;
	\item da capacidade de reconhecer a importância do pensamento computacional na vida cotidiana, como também sua aplicação em outros domínios e ser capaz de aplicá-lo em circunstâncias apropriadas; e
	\item da capacidade de atuar em um mundo de trabalho globalizado.
	
\end{itemize}







\newpage
\section {Organização Curricular}

\subsection{Fundamentação Legal}

A seguir são elencados os documentos legais externos (Diretrizes Curriculares Nacionais, Leis, Decretos, Resoluções, Pareceres, Portarias, Normativas etc.), de ordem federal, estadual, de órgão de classe, dentre outros, bem como os internos (Projeto Pedagógico, Plano de Desenvolvimento Institucional) que fundamentam a estrutura curricular do curso de
Bacharelado em Ciência da Computação da UFABC.

\begin{itemize}

\item BRASIL. Presidência da República. Casa Civil. Subchefia para Assuntos Jurídicos. Lei 9.394, de 20 de dezembro de 1996. Estabelece as diretrizes e bases da educação nacional. Disponível em: \url{https://www.planalto.gov.br/ccivil_03/Leis/L9394.htm}. Acesso em: 07 jul. 2016.

\item BRASIL. Presidência da República. Casa Civil. Subchefia para Assuntos Jurídicos. Lei 10.639, de 9 de janeiro de 2003. Altera a Lei 9.394, de 20 de dezembro de 1996, que estabelece as diretrizes e bases da educação nacional, para incluir no currículo oficial da Rede de Ensino a obrigatoriedade da temática ``História e Cultura Afro-Brasileira'', e dá outras providências. Disponível em: \url{http://www.planalto.gov.br/ccivil_03/leis/2003/l10.639.htm}. Acesso em: 07 jul. 2016.

\item BRASIL. Presidência da República. Casa Civil. Subchefia para Assuntos Jurídicos. Lei 11.645, de 10 de março de 2008. Altera a Lei 9.394, de 20 de dezembro de 1996, modificada pela Lei 10.639, de 9 de janeiro de 2003, que estabelece as diretrizes e bases da educação nacional, para incluir no currículo oficial da rede de ensino a obrigatoriedade da temática ``História e Cultura Afro-Brasileira e Indígena''. Disponível em: \url{http://www.planalto.gov.br/ccivil_03/_ato2007-2010/2008/lei/l11645.htm}. Acesso em:
07 jul. 2016.

\item BRASIL. Ministério da Educação. Conselho Nacional de Educação. Conselho Pleno. Resolução 1, de 17 de junho de 2004. Institui Diretrizes Curriculares Nacionais para a Educação das Relações Étnico Raciais e para o Ensino de História e Cultura Afro-Brasileira e Africana. Disponível em:
\url{http://portal.mec.gov.br/cne/arquivos/pdf/res012004.pdf}. Acesso em: 07 jul. 2016.

\item BRASIL. Ministério da Educação. Secretaria da Educação Superior. Referenciais Orientadores para os Bacharelados Interdisciplinares e Similares. 2010. Disponível em: \url{http://www.ufabc.edu.br/images/stories/comunicacao/bacharelados-interdisciplinares_referenciais-orientadores-novembro_2010-brasilia.pdf}. Acesso em: 07 jul. 2016.


\item BRASIL. Ministério da Educação. Conselho Nacional de Educação. Referenciais orientadores para os Bacharelados Interdisciplinares e Similares das Universidades Federais. Câmara de Educação Superior. Parecer CNE/CES 266, de 5 jul. 2011.Disponível em: \url{http://portal.mec.gov.br/index.php?option=com_content&view=article&id=16418&Itemid=866} Acesso em: 07 jul. 2016.

\item BRASIL. Presidência da República. Casa Civil. Subchefia para Assuntos Jurídicos. Decreto 5.626, de 22 de dezembro de 2005. Regulamenta a Lei  10.436, de 24 de abril de 2002, que dispõe sobre a Língua Brasileira de Sinais - Libras, e o art. 18 da Lei 10.098, de 19 de dezembro de 2000. Disponível em: \url{https://www.planalto.gov.br/ccivil_03/_Ato2004-2006/2005/Decreto/D5626.htm}. Acesso em: 07 jul. 2016.

\item  BRASIL. Presidência da República. Casa Civil. Subchefia para Assuntos Jurídicos. Lei 9.795, de 27 de abril de 1999. Dispõe sobre a educação ambiental, institui a Política Nacional de Educação Ambiental e dá outras providências. Disponível em: \url{http://www.planalto.gov.br/ccivil_03/leis/l9795.htm}. Acesso em: 07 jul. 2016.

\item BRASIL. Presidência da República. Casa Civil. Subchefia para Assuntos Jurídicos. Decreto 4.281, de 25 de junho de 2002. Regulamenta a Lei 9.795, de 27 de abril de 1999, que institui a Política Nacional de Educação Ambiental, e dá outras providências. Disponível em: \url{http://www.planalto.gov.br/ccivil_03/decreto/2002/D4281.htm}. Acesso em: 07 jul. 2016.

\item BRASIL. Ministério da Educação. Conselho Nacional de Educação. Câmara de Educação Superior. Resolução 2, de 18 de junho de 2007. Dispõe sobre carga horária mínima e procedimentos relativos à integralização e duração dos cursos de graduação, bacharelados, na modalidade presencial. Disponível em: \url{http://portal.mec.gov.br/cne/arquivos/pdf/2007/rces002_07.pdf}. Acesso em: 07 jul. 2016.

\item BRASIL. Ministério da Educação. Gabinete do Ministro. Portaria Normativa 40, de 12 de dezembro de 2007. Institui o e-MEC, sistema eletrônico de fluxo de trabalho e gerenciamento de informações relativas aos processos de regulação, avaliação e supervisão da educação superior no sistema federal de educação, e o Cadastro e-MEC de Instituições e Cursos Superiores e consolida disposições sobre indicadores de qualidade, banco de avaliadores (Basis) e o Exame Nacional de Desempenho de Estudantes (ENADE) e outras disposições. Disponível em: \url{http://portal.mec.gov.br/cne/arquivos/pdf/2007/port40_07.pdf}. Acesso em: 07 jul. 2016.

\item BRASIL. Comissão Nacional de Avaliação da Educação Superior. Resolução 1, de 17 de junho de 2010. Normatiza o Núcleo Docente Estruturante e dá outras providências. Disponível em: \url{http://portal.mec.gov.br/index.php?option=com_docman&task=doc_download&gid=6885&Itemid}. Acesso em: 07 jul. 2016.

\item BRASIL. Presidência da República. Casa Civil. Subchefia para Assuntos Jurídicos. Decreto 5.622. Regulamenta o art. 80 da Lei 9.394, de 20 de dezembro de 1996, que estabelece as diretrizes e bases da educação nacional. Disponível em: \url{http://www.planalto.gov.br/ccivil_03/_ato2004-2006/2005/Decreto/D5622compilado.htm}. Acesso em: 07 jul. 2016.

\item BRASIL. Ministério da Educação. Portaria n° 4.059, de 10 de dezembro de 2004. Regulamentação de disciplinas na modalidade semipresencial. Disponível em: \url{http://portal.mec.gov.br/sesu/arquivos/pdf/nova/acs_portaria4059.pdf}. Acesso em: 07 jul. 2016.

\item BRASIL. Ministério da Educação. Conselho Nacional de Educação. Câmara de Educação Superior Parecer CNE/CES nº 136/2012, aprovado em 8 de março de 2012 - Diretrizes Curriculares Nacionais para os cursos de graduação em Computação. Disponível em: \url{http://portal.mec.gov.br/component/content/article?id=12991}. Acesso em: 07 jul. 2016. Publicado no DOU no 134, de 12 de julho de 2012.

\item BRASIL. Ministério da Educação. Conselho Nacional de Educação. Conselho Pleno. Parecer CNE/CP n° 003, de 10 mar. 2004. Disponível em:
\url{http://portal.mec.gov.br/cne/arquivos/pdf/003.pdf}. Acesso em: 07 jul. 2016.

\item BRASIL. Ministério da Educação. Conselho Nacional de Educação. Conselho Pleno. Resolução n° 1, de 30 de maio de 2012. Estabelece Diretrizes Nacionais para a Educação em Direitos Humanos. Disponível em: \url{http://portal.mec.gov.br/index.php?option=com_docman&view=download&alias=10889-
rcp001-12&category_slug=maio-2012-pdf&Itemid=30192}. Acesso em: 07 jul. 2016.

\item BRASIL. Presidência da República. Casa Civil. Subchefia para Assuntos Jurídicos. Lei 12.764, de 27 de dezembro de 2012. Institui a Política Nacional de Proteção dos Direitos da Pessoa com Transtorno do Espectro Autista; e altera o § 3o do art. 98 da Lei 8.112, de 11 de dezembro de 1990. Disponível em: \url{http://www.planalto.gov.br/ccivil_03/_ato2011-2014/2012/lei/l12764.htm}. Acesso em: 07 jul. 2016.

\item FUNDAÇÃO UNIVERSIDADE FEDERAL DO ABC. Projeto Pedagógico. Santo André,
2006. Disponível em: \url{http://www.ufabc.edu.br/images/stories/pdfs/institucional/projetopedagogico.pdf}. Acesso em: 07 jul. 2016.

\item FUNDAÇÃO UNIVERSIDADE FEDERAL DO ABC. Plano de Desenvolvimento
Institucional. Santo André, 2013. Disponível em: \url{http://www.ufabc.edu.br/index.php?option=com_content&view=article&id=7880%3Areso
lucao-consuni-no-112-aprova-o-plano-de-desenvolvimento-institucional-2013-
2022&catid=226%3Aconsuni-resolucoes&Itemid=42}. Acesso em: 07 jul. 2016.

\item BRASIL. Ministério da Educação. Conselho Nacional de Educação. Câmara de Educação Superior. Parecer CNE/CES nº 136/2012, aprovado em 8 de março de
2012 - Diretrizes Curriculares Nacionais para os cursos de graduação em Computação. Disponível em: \url{http://portal.mec.gov.br/index.php?option=com_docman&view=download&alias=11205-pces136-11-pdf&category_slug=julho-2012-pdf&Itemid=30192}.

\item BRASIL. Ministério da Educação. Conselho Nacional de Educação. Câmara de Educação Superior. Resolução CNE/CES 5, de 16 de novembro de 2016 - Institui as Diretrizes Curriculares Nacionais para os cursos de graduação na área da Computação, abrangendo os cursos de bacharelado em Ciência da Computação, em Sistemas de Informação, em Engenharia de Computação, em Engenharia de Software e de licenciatura em Computação, e dá outras providências. Disponível em: \url{http://portal.mec.gov.br/index.php?option=com_docman&view=download&alias=52101-rces005-16-pdf&category_slug=novembro-2016-pdf&Itemid=30192}.

\end{itemize}


\subsection{Regime de Ensino}

\subsubsection{Estrutura Curricular}
A distribuição da quantidade de créditos e da carga-horária a serem cumpridas em cada uma das categorias de disciplinas para a obtenção do grau de Bacharel em Ciência da Computação é dada a seguir:

\begin{center}
\begin{tabular}{|l|c|c|}
\hline
Categoria & Créditos & Carga horária (horas)\\
\hline\hline
Disciplinas obrigatórias do BC\&T & 90 & 1080\\
\hline
Disciplinas obrigatórias do BCC & 124 & 1488 \\
\hline
Disciplinas de opção limitada do BCC & 30 & 360 \\
\hline
Disciplinas livres & 12 & 144\\
\hline
Total & 256 & \\
\hline\hline
Atividades complementares do BC\&T & & 120\\
\hline
Total de horas & & 3192\\
\hline
\end{tabular}
\end{center}


O currículo do curso tem um eixo central de disciplinas, obrigatório para todos os alunos (excetuando as obrigatórias do BC\&T), que padroniza a formação dos acadêmicos da UFABC. Este eixo totaliza 124 créditos, que corresponde a 48,4\% do curso.

Há um conjunto de disciplinas que podem ser selecionadas pelos estudantes, oferecendo autonomia para projetarem esta carga horária de acordo com seus interesses e aptidões. Tais disciplinas são oferecidas em dois grupos: disciplinas de opção limitada e disciplinas livres. As disciplinas de opção limitada devem ser selecionadas dentre aquelas constantes da Tabela 8 e
totalizam 30 créditos da matriz curricular.

As disciplinas livres objetivam a formação complementar do acadêmico, permitindo a escolha das disciplinas dentre as oferecidas nos cursos de graduação da UFABC. Totalizam 12 créditos da matriz curricular.



\subsubsection{Interdisciplinaridade}
O Bacharelado em Ciência e Tecnologia (BC\&T) é a base da matriz curricular do BCC, de maneira que a formação proposta proporciona interdisciplinaridade e flexibilidade curricular. As disciplinas obrigatórias do BC\&T organizam o conhecimento em eixos (Energia, Processos de Transformação, Representação e Simulação, Informação e Comunicação, Estrutura da Matéria e Humanidades), visando despertar o interesse dos alunos para a investigação de cunho interdisciplinar. Os cursos de graduação da UFABC estão estruturados em um sistema de créditos que permite diferentes organizações curriculares, de acordo com os interesses e aptidões dos alunos. Através das disciplinas livres, os alunos poderão se aprofundar em quaisquer áreas do conhecimento, partindo para especificidades curriculares de cursos de formação profissional ou explorando a interdisciplinaridade e estabelecendo um currículo individual de formação.

É importante destacar que a interdisciplinaridade do presente projeto pedagógico e a possibilidade de escolher disciplinas livres, permite que o discente formado no BCC da UFABC esteja alinhado com as seguintes diretrizes legais:

\begin{itemize}

\item Decreto 5.626 de 22 de Dezembro de 2005: a disciplina de LIBRAS, cuja
ementa faz parte do rol de disciplinas dos cursos de licenciatura da UFABC, pode ser cursada pelos alunos do BCC.

\item Lei no 11.64, sobre a obrigatoriedade da temática ``História e Cultura Afro-Brasileira e Indígena'' e Resolução 01/2004, de 17 de junho de 2004: o aluno do BCC pode escolher cursar disciplinas livres que fazem parte do rol de disciplinas da UFABC e que envolvem a temática da História e Cultura Afro-Brasileira e Indígenas.

\item Política Nacional de Educação Ambiental (Lei nº 9795/1999 e decreto  4.281, de 25/06/2002): muitas disciplinas livres oferecidas no rol de disciplinas de engenharia ambiental podem ser cursadas pelos alunos do BCC, permitindo assim a integração desse projeto pedagógico com a educação ambiental.
\end{itemize}



\subsection{Estratégias Pedagógicas}

A UFABC foi concebida definindo a interdisciplinaridade como uma referência pedagógica. Sendo assim,  o desenvolvimento
do perfil do egresso é trabalhado com uma formação interdisciplinar com alto grau de liberdade para incorporar  
componentes curriculares à sua formação. Além de cobrir os assuntos pertinentes à formação definida pelas DCNs de Computação,
esse modelo possibilita que o aluno desenvolva competências em outras áreas de seu próprio interesse.

Seguindo a recomendação da matriz curricular, os primeiros quadrimestres letivos de curso são preenchidos por disciplinas do BC\&T, onde o(a) aluno(a) tem contato com várias áreas da Ciência, fortalecendo sua base científica e humanística, além de experimentar os primeiros contatos
com disciplinas da área de Computação (Bases Computacionais da Ciência, Natureza da Informação, Comunicação e Redes, Processamento da Informação). Aos poucos, o(a) aluno(a) vai encontrar janelas de horários para incluir disciplinas específicas de Computação enquanto finaliza sua formação no BC\&T. O projeto pedagógico prevê, ainda, 25\% de carga horária em disciplinas livres e de opção limitada, em que o(a) aluno(a) poderá escolher os componentes curriculares que completarão a sua formação.

Na UFABC as disciplinas não possuem pré-requisitos entre disciplinas. Mesmo assim, a estrutura da matriz curricular sugere uma composição
que favorece o desenvolvimento contínuo das competências e habilidades do egresso durante o desenvolvimento do curso, concentrando 
disciplinas que abordam temas avançados e específicos no final do curso e disciplinas fundamentais em seu início.

O uso de Tecnologias de Informação e Comunicação (TICs) é uma realidade próxima dos estudantes na UFABC. Muitas disciplinas
utilizam ambientes virtuais de aprendizagem (AVAs) para gestão de conteúdo em disciplinas presenciais e semipresenciais. Durante
a pandemia, algumas disciplinas foram também ofertadas na modalidade online com sucesso. Todos os cursos possuem páginas 
específicas em que seus conteúdos e documentos ficam acessíveis à comunidade (projeto pedagógico, informações gerais, documentos, links para outras páginas de recursos, etc.). Uma importante parcela da carga horária total é trabalhada em aulas práticas, ofertadas em laboratórios de informática com computadores ou laboratórios de {\it hardware} com dispositivos eletrônicos.
 

Em termos de acessibilidade, a UFABC tem se preocupa com a garantia de acesso às
pessoas com deficiência e/ou com mobilidade reduzida. Seguindo as determinações do
Decreto n° 5.296/2004 47 e da Lei 10.098/2000 48, os dois câmpus da UFABC possuem
acessibilidade arquitetônica, garantindo o uso autônomo dos espaços por pessoas com
deficiência e/ou com mobilidade reduzida. Através do Núcleo de Acessibilidade da Pró-Reitoria
de Assuntos Comunitários e Políticas Afirmativas (PROAP), a UFABC tem procurado a
excelência no quesito inclusão. Nesse sentido, dentre as disciplinas oferecidas pela UFABC,
destacamos o oferecimento da disciplina NHI5010-15 - LIBRAS.

Políticas de educação ambiental e de educação em direitos humanos são tratadas 
nas disciplinas ofertadas pela UFABC relacionadas à Educação Ambiental: 
ESZU025-17 - Educação Ambiental; ESHC034-17 - Economia e Meio Ambiente; 
ESZU006-17 – Economia; Sociedade e Meio Ambiente e
ESTE004-17 – Energia, Meio Ambiente e Sociedade.

Dentre as disciplinas ofertadas pela UFABC relacionadas à Educação em Direitos
Humanos citamos: ESHR028-14 - Regime Internacional dos Direitos Humanos e a Atuação
Brasileira; ESZP029-13 - Movimentos Sindicais, Sociais e Culturais; ESZP014-13 - Diversidade
Cultural, Conhecimento Local e Políticas Públicas; BHQ0001-15 - Identidade e Cultura e
ESHP004-13 - Cidadania, Direitos e Desigualdades; ESZR002-13 - Cultura, Identidade e
Política na América Latina, e; ESHR027-14 - Trajetórias Internacionais do Continente Africano.




\newpage
\subsection{Matriz Curricular Recomendada}

\scalebox{.7}{
\begin{tikzpicture}

%Q1
\node [draw, rotate=90, black,rectangle, minimum width=120pt, minimum height=10pt, rounded corners] at (-25pt,-40pt) {\footnotesize{1o. ANO}};

\node [draw, rotate=90, black,rectangle, minimum width=40pt, minimum height=10pt, rounded corners] at (-10pt,0pt) {\footnotesize{1o. quad}};

\disciplina {0pt}{0pt}{a}{63}{Base Experimental das Ciências Naturais\\ (0-3-2)}

\disciplina {63pt}{0pt}{b}{42}{Bases Computacionais da Ciência \\(0-2-2)} 

\disciplina {105pt}{0pt}{c}{84}{Bases Matemáticas\\(4-0-5)}

\disciplina {189pt}{0pt}{d}{42}{Bases Conceituais da Energia \\(2-0-4)}

\disciplina {231pt}{0pt}{e}{63}{Estrutura da Matéria \\(3-0-4)}

\disciplina {294pt}{0pt}{f}{63}{Evolução e Diversificação da Vida na Terra \\(3-0-4)}


%Q2
\node [draw, rotate=90, black,rectangle, minimum width=40pt, minimum height=10pt, rounded corners] at (-10pt,-40pt) {\footnotesize{2o. quad}};

\disciplina [yellow!20]{0pt}{-40pt}{bis0505}{63}{Natureza da Informação\\(3-0-4)}

\disciplina [yellow!20]{63pt}{-40pt}{bis0505}{63}{Geometria Analítica\\(3-0-6)}

\disciplina [yellow!20]{126pt}{-40pt}{bis0505}{84}{Funções de Uma Variável\\(4-0-6)}

\disciplina [yellow!20]{210pt}{-40pt}{bis0505}{84}{Fenômenos Mecânicos\\(3-1-4)}

\disciplina [yellow!20]{294pt}{-40pt}{bis0505}{63}{Biodiversidade: Int. Organismos e Ambiente\\(3-0-4)}

%Q3
\node [draw, rotate=90, black,rectangle, minimum width=40pt, minimum height=10pt, rounded corners] at (-10pt,-80pt) {\footnotesize{3o. quad}};

\disciplina [yellow!20]{0pt}{-80pt}{bis0505}{105}{Processamento da Informação\\(3-2-5)}

\disciplina [yellow!20]{105pt}{-80pt}{bis0505}{84}{Funções de Várias Variáveis\\(4-0-4)}

\disciplina [yellow!20]{189pt}{-80pt}{bis0505}{84}{Fenômenos Térmicos\\(3-1-4)}

\disciplina [yellow!20]{273pt}{-80pt}{bis0505}{105}{Transformações Químicas\\(3-2-6)}


%Q4
\node [draw, rotate=90, black,rectangle, minimum width=120pt, minimum height=10pt, rounded corners] at (-25pt,-160pt) {\footnotesize{2o. ANO}};

\node [draw, rotate=90, black,rectangle, minimum width=40pt, minimum height=10pt, rounded corners] at (-10pt,-120pt) {\footnotesize{4o. quad}};

\disciplina [yellow!20]{0pt}{-120pt}{bis0505}{63}{Comunicação e Redes\\(3-0-4)}

\disciplina [yellow!20]{63pt}{-120pt}{bis0505}{63}{Introdução à Probabilidade e Estatística\\(3-0-4)}

\disciplina [yellow!20]{126pt}{-120pt}{bis0505}{84}{Introdução às Equações Diferencias Ordinárias\\(4-0-4)}

\disciplina [yellow!20]{210pt}{-120pt}{bis0505}{105}{Fenômenos Eletromagnéticos\\(4-1-6)}

\disciplina [yellow!20]{315pt}{-120pt}{bis0505}{63}{Bases Epistemológicas da Ciência Moderna\\(3-0-4)}

%Q5
\node [draw, rotate=90, black,rectangle, minimum width=40pt, minimum height=10pt, rounded corners] at (-10pt,-160pt) {\footnotesize{5o. quad}};

\disciplina [yellow!20]{0pt}{-160pt}{bis0505}{63}{Física Quântica\\(3-0-4)}

\disciplina [yellow!20]{63pt}{-160pt}{bis0505}{105}{Bioquímica: Estrutura, Propriedade e Funções de Biomoléculas\\(3-2-6)}

\disciplina [yellow!20]{168pt}{-160pt}{bis0505}{63}{Dinâmica e Estrutura Social\\(3-0-4)}

\disciplina [blue!20]{231pt}{-160pt}{bis0505}{84}{Lógica Básica\\(4-0-4)}

\disciplina [blue!20]{315pt}{-160pt}{bis0505}{84}{Programação Estruturada\\(2-2-4)}

%Q6
\node [draw, rotate=90, black,rectangle, minimum width=40pt, minimum height=10pt, rounded corners] at (-10pt,-200pt) {\footnotesize{6o. quad}};

\disciplina [yellow!20]{0pt}{-200pt}{bis0505}{63}{Interações Atômicas e Moleculares\\(3-0-4)}

\disciplina [yellow!20]{63pt}{-200pt}{bis0505}{63}{Ciência, Tecnologia \\e Sociedade\\(3-0-4)}

\disciplina [blue!20]{126pt}{-200pt}{bis0505}{84}{Circuitos Digitais\\(3-1-4)}

\disciplina [blue!20]{210pt}{-200pt}{bis0505}{84}{Algoritmos e Estruturas de Dados I\\(2-2-4)}

\disciplina [blue!20]{294pt}{-200pt}{bis0505}{84}{Matemática Discreta\\(4-0-4)}

%Q7
\node [draw, rotate=90, black,rectangle, minimum width=120pt, minimum height=10pt, rounded corners] at (-25pt,-280pt) {\footnotesize{3o. ANO}};


\node [draw, rotate=90, black,rectangle, minimum width=40pt, minimum height=10pt, rounded corners] at (-10pt,-240pt) {\footnotesize{7o. quad}};

\disciplina [blue!20]{0pt}{-240pt}{bis0505}{84}{Sistemas Digitais\\(2-2-4)}

\disciplina [blue!20]{84pt}{-240pt}{bis0505}{84}{Análise de Algoritmos\\(4-0-4)}

\disciplina [blue!20]{168pt}{-240pt}{bis0505}{84}{Programação Orientada a Objetos\\(2-2-4)}

\disciplina [blue!20]{252pt}{-240pt}{bis0505}{126}{Álgebra Linear\\(6-0-5)}

\disciplina [blue!20]{378pt}{-240pt}{bis0505}{42}{Computadores, Ética e Sociedade\\(2-0-4)}

%Q8
\node [draw, rotate=90, black,rectangle, minimum width=40pt, minimum height=10pt, rounded corners] at (-10pt,-280pt) {\footnotesize{8o. quad}};

\disciplina [blue!20]{0pt}{-280pt}{bis0505}{84}{Arquitetura de Computadores\\(4-0-4)}

\disciplina [blue!20]{84pt}{-280pt}{bis0505}{84}{Algoritmos e Estruturas de Dados II\\(2-2-4)}

\disciplina [blue!20]{168pt}{-280pt}{bis0505}{84}{Teoria dos Grafos\\(3-1-4)}

\disciplina [blue!20]{252pt}{-280pt}{bis0505}{84}{Banco de Dados\\(3-1-4)}

\disciplina [blue!20]{336pt}{-280pt}{bis0505}{84}{Inteligência Artificial\\(3-1-4)}


%Q9
\node [draw, rotate=90, black,rectangle, minimum width=40pt, minimum height=10pt, rounded corners] at (-10pt,-320pt) {\footnotesize{9o. quad}};


\disciplina [blue!20]{0pt}{-320pt}{bis0505}{84}{Redes de Computadores\\(3-1-4)}

\disciplina [blue!20]{84pt}{-320pt}{bis0505}{84}{Sistemas Operacionais\\(3-1-4)}

\disciplina [blue!20]{168pt}{-320pt}{bis0505}{84}{Linguagens Formais e Automata\\(3-1-4)}

\disciplina [blue!20]{252pt}{-320pt}{bis0505}{84}{Engenharia de Software\\(4-0-4)}

\disciplina [yellow!20]{336pt}{-320pt}{bis0505}{42}{Projeto Dirigido\\(0-2-10)}

\disciplina [green!20]{378pt}{-320pt}{bis0505}{84}{Livre\\(4 créditos)}

%Q10
\node [draw, rotate=90, black,rectangle, minimum width=120pt, minimum height=10pt, rounded corners] at (-25pt,-400pt) {\footnotesize{4o. ANO}};

\node [draw, rotate=90, black,rectangle, minimum width=40pt, minimum height=10pt, rounded corners] at (-10pt,-360pt) {\footnotesize{10o. quad}};


\disciplina [blue!20]{0pt}{-360pt}{bis0505}{84}{Sistemas Distribuídos\\(3-1-4)}

\disciplina [blue!20]{84pt}{-360pt}{bis0505}{84}{Compiladores\\(3-1-4)}

\disciplina [blue!20]{168pt}{-360pt}{bis0505}{84}{Paradigmas de Programação\\(2-2-4)}

\disciplina [blue!20]{252pt}{-360pt}{bis0505}{168}{Projeto de Graduação em Computação I\\(0-8-8)}

\disciplina [red!20]{420pt}{-360pt}{bis0505}{168}{Opção Limitada\\(8 créditos)}

%Q11
\node [draw, rotate=90, black,rectangle, minimum width=40pt, minimum height=10pt, rounded corners] at (-10pt,-400pt) {\footnotesize{11o. quad}};


\disciplina [blue!20]{0pt}{-400pt}{bis0505}{84}{Computação Gráfica\\(3-1-4)}

\disciplina [blue!20]{84pt}{-400pt}{bis0505}{84}{Programação Matemática\\(3-1-4)}

\disciplina [blue!20]{168pt}{-400pt}{bis0505}{168}{Projeto de Graduação em Computação II\\(0-8-8)}

\disciplina [red!20]{336pt}{-400pt}{bis0505}{252}{Opção Limitada\\(12 créditos)}

%Q12
\node [draw, rotate=90, black,rectangle, minimum width=40pt, minimum height=10pt, rounded corners] at (-10pt,-440pt) {\footnotesize{12o. quad}};

\disciplina [blue!20]{0pt}{-440pt}{bis0505}{84}{Segurança de Dados\\(3-1-4)}

\disciplina [blue!20]{84pt}{-440pt}{bis0505}{168}{Projeto de Graduação em Computação III\\(0-8-8)}

\disciplina [red!20]{252pt}{-440pt}{bis0505}{210}{Opção Limitada\\(10 créditos)}

\disciplina [green!20]{462pt}{-440pt}{bis0505}{168}{Livre\\(8 créditos)}

\draw [red,thick,dashed] (420pt,20pt) -- (420pt,-480pt); 
\node [text=red] at (420pt,35pt) {20 créditos};


\end{tikzpicture}
}




\subsection{Mapeamento de Habilidades/ Competências e Atividades Pedagógicas}
A organização curricular foi desenhada para atender aos requisitos estruturais e pedagógicos da UFABC, bem como às Diretrizes Curriculares Nacionais dos cursos de graduação em Computação (Res. CNE/CES no. 5, de 16/11/2016). A seguir, indicamos os componentes pedagógicos que contribuem para a formação e consolidação das habilidades e competências dos egressos. As atividades pedagógicas estão classificadas da seguinte forma:

\begin{itemize}
	\item \textcolor{red}{Disciplinas obrigatórias do BC\&T}
	\item \textcolor{blue}{Disciplinas obrigatórias do BCC}
	\item \textcolor{teal}{Disciplinas de opção limitada do BCC}
	\item \textcolor{violet}{Outras ações}
\end{itemize}


%\begin{tabular}{|ccc|}
%	\multicolumn{3}{l}{Identificar problemas que tenham solução algorítmica}\\
%	\hline
%	Bases Computacionais da Ciência  & Algoritmos e Estruturas de Dados I & Teoria dos Grafos\\
%	Processamento da Informação & Algoritmos e Estruturas de Dados II & Inteligência Artificial\\
%	Programação Estruturada & Análise de Algoritmos & Paradigmas de Programação\\
%	Matemática Discreta & Programação Orientada a Objetos & Programação Matemática\\
%	\hline
%\end{tabular}


\begin{longtable}{|p{.35\textwidth}p{.35\textwidth}p{.3\textwidth}|}
	\multicolumn{3}{l}{Identificar problemas que tenham solução algorítmica}\\
	\hline
	\textcolor{red}{Bases Computacionais da Ciência}  & \textcolor{blue}{Algoritmos e Estruturas de Dados I} & \textcolor{blue}{Teoria dos Grafos}\\
	\textcolor{red}{Processamento da Informação} & \textcolor{blue}{Algoritmos e Estruturas de Dados II} & \textcolor{blue}{Inteligência Artificial}\\
	\textcolor{blue}{Programação Estruturada} & \textcolor{blue}{Análise de Algoritmos} & \textcolor{blue}{Paradigmas de Programação}\\
	\textcolor{blue}{Matemática Discreta} & \textcolor{blue}{Programação Orientada a Objetos} & \textcolor{blue}{Programação Matemática}\\
	\hline
	
	\multicolumn{3}{l}{}\\
	
	\multicolumn{3}{l}{Conhecer os limites da computação}\\
	\hline
	\textcolor{blue}{Análise de Algoritmos} & \textcolor{blue}{Linguagens Formais e Automata} & \textcolor{blue}{Teoria dos Grafos}\\
	\hline
	
	\multicolumn{3}{l}{}\\
	\multicolumn{3}{l}{Resolver problemas usando ambientes de programação}\\
	\hline
	\textcolor{red}{Processamento da Informação} &  \textcolor{blue}{Teoria dos Grafos} & \textcolor{blue}{Engenharia de Software}\\
	\textcolor{blue}{Programação Estruturada} &  \textcolor{blue}{Inteligência Artificial} &  \textcolor{blue}{Programação Matemática}\\
	\textcolor{blue}{Algoritmos e Estruturas de Dados I} & \textcolor{blue}{Programação Orientada a Objetos} & \textcolor{blue}{Paradigmas de Programação}\\
	\textcolor{blue}{Algoritmos e Estruturas de Dados II} & \textcolor{blue}{Banco de Dados} &  \textcolor{blue}{Sistemas Digitais}\\
	\textcolor{blue}{Compiladores} && \\
	\hline
	
	\multicolumn{3}{l}{}\\
	\multicolumn{3}{l}{\makecell[l]{Tomar decisões e inovar, com base no conhecimento do funcionamento e das características técnicas de \\
			hardware e da infraestrutura de software dos sistemas de computação consciente dos aspectos éticos, \\
			legais e dos impactos ambientais decorrentes }}\\
	\hline
	\textcolor{red}{Ciência, Tecnologia e Sociedade} & \textcolor{blue}{Redes de Computadores} & \textcolor{blue}{Segurança de Dados}\\
	\textcolor{red}{Comunicação e Redes} &  \textcolor{blue}{Sistemas Operacionais} & \textcolor{blue}{Banco de Dados}\\
	\textcolor{blue}{Arquitetura de Computadores} & \textcolor{blue}{Sistemas Distribuídos} & \textcolor{blue}{Engenharia de Software}\\
	\textcolor{blue}{Sistemas Digitais} &  \textcolor{blue}{Computadores, Ética e Sociedade} & \\
	\hline
	
	\multicolumn{3}{l}{}\\
	\multicolumn{3}{l}{Compreender e explicar as dimensões quantitativas de um problema}\\
	\hline
	\textcolor{red}{Natureza da Informação} & \textcolor{red}{Geometria Analítica} & \textcolor{blue}{Linguagens Formais e Automata}\\
	\textcolor{red}{Intr. à Probabilidade e Estatística} &  \textcolor{blue}{Álgebra Linear} & \textcolor{blue}{Programação Matemática}\\
	\textcolor{red}{Funções de Uma Variável} & \textcolor{blue}{Matemática Discreta} &  \textcolor{blue}{Engenharia de Software}\\
	\textcolor{red}{Funções de Várias Variáveis} &  \textcolor{blue}{Análise de Algoritmos} &\\
	\hline
	
	\multicolumn{3}{l}{}\\
	\multicolumn{3}{l}{\makecell[l]{Gerir a sua própria aprendizagem e desenvolvimento, incluindo a gestão de tempo e competências \\organizacionais}}\\
	\hline
	\textcolor{blue}{Proj. de Grad. Computaçao I} & \textcolor{blue}{Proj. de Grad. Computaçao II} & \textcolor{blue}{Proj. de Grad. Computaçao III}\\
	\hline
	
	\multicolumn{3}{l}{}\\
	\multicolumn{3}{l}{\makecell[l]{Preparar e apresentar seus trabalhos e problemas técnicos e suas soluções para audiências diversas, em \\formatos apropriados (oral e escrito)}}\\
	\hline
	\textcolor{red}{Projeto Dirigido} & \textcolor{blue}{Proj. de Grad. Computaçao I} & \textcolor{blue}{Proj. de Grad. Computaçao III}\\
	\textcolor{blue}{Engenharia de Software} & \textcolor{blue}{Proj. de Grad. Computaçao II} &\\
	\hline
	
	\multicolumn{3}{l}{}\\
	\multicolumn{3}{l}{Avaliar criticamente projetos de sistemas de computação}\\
	\hline
	\textcolor{red}{Ciência, Tecnologia e Sociedade} & \textcolor{blue}{Segurança de Dados} &\textcolor{blue}{Análise de Algoritmos}\\
	\textcolor{blue}{Engenharia de Software} & \textcolor{blue}{Computadores, Ética e Sociedade}& \textcolor{blue}{Sistemas Distribuídos}\\
	\textcolor{blue}{Redes de Computadores} & &\\
	\hline
	
	\multicolumn{3}{l}{}\\
	\multicolumn{3}{l}{Adequar-se rapidamente às mudanças tecnológicas e aos novos ambientes de trabalho}\\
	\hline
	\textcolor{blue}{Computadores, Ética e Sociedade} & & \\
	\hline
	
	\multicolumn{3}{l}{}\\
	\multicolumn{3}{l}{Ler textos técnicos na língua inglesa}\\
	\hline
	\textcolor{red}{Projeto Dirigido} & \textcolor{blue}{Proj. de Grad. Computaçao II} & \textcolor{blue}{Proj. de Grad. Computaçao III}\\
	\textcolor{blue}{Proj. de Grad. Computaçao I} && \\
	\hline
	
	\multicolumn{3}{l}{}\\
	\multicolumn{3}{l}{Empreender e exercer liderança, coordenação e supervisão na sua área de atuação profissional}\\
	\hline
	&& \\
	\hline
	
	\multicolumn{3}{l}{}\\
	\multicolumn{3}{l}{Ser capaz de realizar trabalho cooperativo e entender os benefícios que este pode produzir}\\
	\hline
	\textcolor{red}{Ciência, Tecnologia e Sociedade} & \textcolor{blue}{Computadores, Ética e Sociedade} & \textcolor{blue}{Engenharia de Software}\\
	\hline
	
	\multicolumn{3}{l}{}\\
	\multicolumn{3}{l}{\makecell[l]{Compreender os fatos essenciais, os conceitos, os princípios e as teorias relacionadas à Ciência da \\Computação para o desenvolvimento de software e hardware e suas aplicações}}\\
	\hline
	\textcolor{red}{Bases Computacionais da Ciência} & \textcolor{blue}{Paradigmas de Programação} & \textcolor{blue}{Teoria dos Grafos}\\
	\textcolor{red}{Processamento da Informação} &  \textcolor{blue}{Algoritmos e Estruturas de Dados I} & \textcolor{blue}{Programação Matemática}\\
	\textcolor{blue}{Programação Estruturada} & \textcolor{blue}{Algoritmos e Estruturas de Dados II} & \textcolor{blue}{Circuitos Digitais}\\
	\textcolor{blue}{Análise de Algoritmos} &\textcolor{blue}{Linguagens Formais e Automata} & \textcolor{blue}{Sistemas Digitais}\\
	\textcolor{blue}{Arquitetura de Computadores} & \textcolor{blue}{Matemática Discreta} & \textcolor{blue}{Sistemas Operacionais}\\
	\textcolor{blue}{Lógica Básica} &   &   \\
	\hline
	
	\multicolumn{3}{l}{}\\
	\multicolumn{3}{l}{\makecell[l]{Reconhecer a importância do pensamento computacional no cotidiano e sua aplicação em circunstâncias \\apropriadas e em domínios diversos }}\\
	\hline
	\textcolor{red}{Comunicação e Redes} & \textcolor{red}{Bases Computacionais da Ciência} & \textcolor{blue}{Lógica Básica}\\
	\textcolor{red}{Processamento da Informação} & \textcolor{blue}{Algoritmos e Estruturas de Dados I}& \textcolor{blue}{Matemática Discreta}\\
	\textcolor{red}{Ciência, Tecnologia e Sociedade} & \textcolor{blue}{Algoritmos e Estruturas de Dados II}&  \textcolor{blue}{Teoria dos Grafos}\\
	\textcolor{blue}{Programação Estruturada} &\textcolor{blue}{Computadores, Ética e Sociedade} & \textcolor{blue}{Programação Matemática}\\
	\hline
	
	\multicolumn{3}{l}{}\\
	\multicolumn{3}{l}{\makecell[l]{Identificar e gerenciar os riscos que podem estar envolvidos na operação de equipamentos de computação \\(incluindo os aspectos de dependabilidade e segurança)}}\\
	\hline
	\textcolor{blue}{Segurança de Dados} & \textcolor{blue}{Computadores, Ética e Sociedade} &  \textcolor{blue}{Circuitos Digitais}\\
	\textcolor{blue}{Banco de Dados} & \textcolor{blue}{Redes de Computadores} & \\
	\hline
	
	\multicolumn{3}{l}{}\\
	\multicolumn{3}{l}{\makecell[l]{Identificar e analisar requisitos e especificações para problemas específicos e planejar estratégias para \\suas soluções}}\\
	\hline
	\textcolor{blue}{Engenharia de Software} & \textcolor{blue}{Arquitetura de Computadores} & \textcolor{blue}{Circuitos Digitais}\\
	\textcolor{blue}{Análise de Algoritmos} & \textcolor{blue}{Banco de Dados} & \textcolor{blue}{Sistemas Digitais}\\
	\hline
	
	\multicolumn{3}{l}{}\\
	\multicolumn{3}{l}{\makecell[l]{Especificar, projetar, implementar, manter e avaliar sistemas de computação, empregando teorias, práticas e \\ferramentas adequadas}}\\
	\hline
	\textcolor{blue}{Sistemas Operacionais} &  \textcolor{blue}{Arquitetura de Computadores} &  \textcolor{blue}{Análise de Algoritmos}\\
	\textcolor{blue}{Banco de Dados} & \textcolor{blue}{Redes de Computadores} &  \textcolor{blue}{Compiladores}\\
	\textcolor{blue}{Engenharia de Software} & \textcolor{blue}{Sistemas Distribuídos} & \textcolor{blue}{Sistemas Digitais}\\
	\hline
	
	\multicolumn{3}{l}{}\\
	\multicolumn{3}{l}{\makecell[l]{Conceber soluções computacionais a partir de decisões visando o equilíbrio de todos os fatores envolvidos}}\\
	\hline
	\textcolor{red}{Processamento da Informação} & \textcolor{blue}{Algoritmos e Estruturas de Dados I}&  \textcolor{blue}{Segurança de Dados}\\
	\textcolor{red}{Ciência, Tecnologia e Sociedade} & \textcolor{blue}{Algoritmos e Estruturas de Dados II}& \textcolor{blue}{Engenharia de Software}\\
	\textcolor{blue}{Programação Estruturada} & \textcolor{blue}{Computadores, Ética e Sociedade}& \textcolor{blue}{Compiladores}\\
	\textcolor{blue}{Inteligência Artificial} & \textcolor{blue}{Paradigmas de Programação}& \\
	\hline
	
	\multicolumn{3}{l}{}\\
	\multicolumn{3}{l}{\makecell[l]{Empregar metodologias que visem garantir critérios de qualidade ao longo de todas as etapas de \\desenvolvimento de uma solução computacional}}\\
	\hline
	\textcolor{blue}{Engenharia de Software} & & \\
	\hline
	
	\multicolumn{3}{l}{}\\
	\multicolumn{3}{l}{\makecell[l]{Analisar quanto um sistema baseado em computadores atende os critérios definidos para seu uso corrente e \\futuro (adequabilidade)}}\\
	\hline
	\textcolor{blue}{Engenharia de Software} & \textcolor{blue}{Análise de Algoritmos} & \textcolor{blue}{Banco de Dados}\\
	\hline
	
	\multicolumn{3}{l}{}\\
	\multicolumn{3}{l}{\makecell[l]{Gerenciar projetos de desenvolvimento de sistemas computacionais}}\\
	\hline
	\textcolor{blue}{Compiladores} & \textcolor{blue}{Sistemas Operacionais}&  \textcolor{blue}{Banco de Dados}\\
	\textcolor{blue}{Engenharia de Software} &  \textcolor{teal}{Gestão de Projetos de Software} & \\
	\hline
	
	\multicolumn{3}{l}{}\\
	\multicolumn{3}{l}{\makecell[l]{Aplicar temas e princípios recorrentes, como abstração, complexidade, princípio de localidade de referência \\(caching), compartilhamento de recursos, segurança, concorrência, evolução de sistemas, entre outros, e \\reconhecer que esses temas e princípios são fundamentais à área de Ciência da Computação}}\\
	\hline
	\textcolor{blue}{Análise de Algoritmos} & \textcolor{blue}{Programação Orientada a Objetos} & \textcolor{blue}{Banco de Dados}\\
	\textcolor{blue}{Segurança de Dados} &  \textcolor{blue}{Linguagens Formais e Automata} & \textcolor{blue}{Sistemas Distribuídos}\\
	\textcolor{blue}{Redes de Computadores} & \textcolor{blue}{Arquitetura de Computadores} & \textcolor{blue}{Sistemas Operacionais}\\
	\textcolor{blue}{Inteligência Artificial} & \textcolor{blue}{Engenharia de Software} & \textcolor{blue}{Sistemas Digitais}\\
	\hline
	
	
	\multicolumn{3}{l}{}\\
	\multicolumn{3}{l}{\makecell[l]{Escolher e aplicar boas práticas e técnicas que conduzam ao raciocínio rigoroso no planejamento, na execução \\e no acompanhamento, na medição e gerenciamento geral da qualidade de sistemas computacionais}}\\
	\hline
	\textcolor{blue}{Análise de Algoritmos} & \textcolor{blue}{Computadores, Ética e Sociedade} & \textcolor{blue}{Banco de Dados}\\
	\textcolor{blue}{Engenharia de Software} & \textcolor{teal}{Sistemas de Informação} & \textcolor{blue}{Sistemas Digitais}\\
	\textcolor{blue}{Compiladores} & & \\
	\hline
	
	
	\multicolumn{3}{l}{}\\
	\multicolumn{3}{l}{\makecell[l]{Aplicar os princípios de gerência, organização e recuperação da informação de vários tipos, incluindo texto, \\imagem, som e vídeo}}\\
	\hline
	\textcolor{blue}{Banco de Dados} & \textcolor{blue}{Algoritmos e Estruturas de Dados I} & \textcolor{blue}{Sistemas Distribuídos}\\
	\textcolor{blue}{Computação Gráfica} & \textcolor{blue}{Algoritmos e Estruturas de Dados II} & \textcolor{teal}{Proc. de Sinais Neurais} \\
	\textcolor{blue}{Redes de Computadores} & \textcolor{teal}{Processamento Digital de Imagens} & \\
	\hline
	
	\multicolumn{3}{l}{}\\
	\multicolumn{3}{l}{\makecell[l]{Aplicar os princípios de interação humano-computador para avaliar e construir uma grande variedade de \\produtos, incluindo interface de usuário, páginas WEB, sistemas multimídia e sistemas móveis}}\\
	\hline
	\textcolor{blue}{Computação Gráfica} & \textcolor{teal}{Interação Humano-Computador} & \textcolor{teal}{Sistemas Multimidia}\\
	\textcolor{teal}{Programação para Web} &  \textcolor{teal}{Prog. Av. de Dispositivos Móveis} & \textcolor{teal}{Visão Computacional}\\
	\textcolor{teal}{Sistemas Inteligentes} & & \\
	\hline
	
	\end {longtable}



\newpage
\section {Ações Acadêmicas Complementares à Formação}

São ações complementares à Formação oferecidas pela UFABC:
\begin{itemize}
	\item Projeto de Ensino-Aprendizagem Tutorial – PEAT (maiores informações em: \url{http://prograd.ufabc.edu.br/peat});
	\item Iniciação científica (maiores informações em: \url{http://ic.ufabc.edu.br/images/manual.pdf}) em suas diferentes ofertas: 
	\begin{itemize}
		\item Programa Pesquisando Desde o Primeiro Dia – PDPD;
		\item Programa de Iniciação Científica – PIC/UFABC;
		\item Programa Institucional de Bolsas de Iniciação Científica – PIBIC/CNPq;
		\item Programa Institucional de Bolsas de Iniciação Científica – PIBIC/CNPq nas Ações Afirmativas.
		
	\end{itemize}
	\item Monitoria acadêmica (maiores informações em: \url{http://prograd.ufabc.edu.br/monitoria-academica});
	\item Programa Institucional de Bolsas de Iniciação à Docência – PIBID (maiores informações em: \url{http://pibidufabc.wordpress.com/});
	\item Ações extensionistas (cursos, bolsas, eventos, etc.) (maiores informações em: \url{http://proec.ufabc.edu.br/});
	\item Programa de Educação Tutorial (maiores informações em: \url{http://prograd.ufabc.edu.br/pet});
	\item Cursos de língua estrangeira (maiores informações em \url{http://nte.ufabc.edu.br/});
	\item Mobilidade Acadêmica (maiores informações em: \url{http://ri.ufabc.edu.br/?source=Portal});
	\item Monitoria inclusiva (maiores informações em : \url{http://proap.ufabc.edu.br/acessibilidade-ufabc/servicos-e-recursos/monitoria-inclusiva});
	\item Programa de Apoio ao Desenvolvimento Acadêmico (PADA) da UFABC regulamentado pela Resolução ConsEPE no. 167.
	
\end{itemize}



\newpage
\section {Atividades Complementares}

As atividades complementares são todas as atividades de diversas naturezas, que não se incluem no desenvolvimento regular das disciplinas constantes na matriz curricular do BCC, mas que são relevantes para a formação do aluno.

O objetivo do incentivo à realização de atividades complementares consiste em fornecer ao estudante a oportunidade de enriquecer sua formação com a participação em atividades de natureza diversificada. Como consequência, tem-se a acentuação do caráter interdisciplinar e amplo da formação do aluno, além do fortalecimento do vínculo entre teoria e prática.

Uma vez que o BC\&T é requisito para o BCC, e neste curso já está prevista a realização de 120 horas de atividades complementares, o BCC não exigirá a realização de atividades complementares específicas além das já previstas no BC\&T. 

A forma de validação da carga horária dessas atividades encontra-se na Resolução CG no. 11, de 28 de junho de 2016, publicado pelo Boletim de Serviços no. 568, de 05 de julho de 2016.

\newpage
\section {Estágio Curricular}

O Estágio Curricular não é obrigatório para o curso de Bacharelado em Ciência da Computação.

\newpage
\section {Trabalho de Conclusão de Curso}

\newpage
\section {Avaliação de Processo Ensino-Aprendizagem}

A avaliação do processo de ensino e aprendizagem dos discentes na UFABC é feito por
meio de conceitos, o que permite uma análise mais qualitativa do aproveitamento do aluno.
Segundo a Resolução ConsEPE nº 147, de 19 de março de 2013, os coeficientes de
desempenho utilizados na Instituição consistem em:
\begin{itemize}
	\item [A -] Desempenho excepcional, demonstrando excelente compreensão da disciplina e do uso do conteúdo.
	\item [B -] Bom desempenho, demonstrando boa capacidade de uso dos conceitos da disciplina.
	\item [C -] Desempenho mínimo satisfatório, demonstrando capacidade de uso adequado dos conceitos da disciplina, habilidade para enfrentar problemas relativamente simples e prosseguir em estudos avançados.
	\item[D -] Aproveitamento mínimo não satisfatório dos conceitos da disciplina, com familiaridade parcial do assunto e alguma capacidade para resolver problemas simples, mas demonstrando deficiências que exigem trabalho adicional para prosseguir em estudos avançados. Nesse caso, o aluno é aprovado na expectativa de que obtenha um conceito melhor em outra disciplina, para compensar o conceito D no cálculo do CR. Havendo vaga, o aluno poderá cursar esta disciplina novamente.
	\item [F -] Reprovado. A disciplina deve ser cursada novamente para obtenção de crédito.
	\item [O -] Reprovado por falta. A disciplina deve ser cursada novamente para obtenção de crédito.
\end{itemize}

Os conceitos a serem atribuídos aos estudantes, em uma dada disciplina, não deverão
estar rigidamente relacionados a qualquer nota numérica de provas, trabalhos ou exercícios. Os resultados também considerarão a capacidade do aluno de utilizar os conceitos e material das disciplinas, criatividade, originalidade, clareza de apresentação e participação em sala de aula e/ou laboratórios. O aluno, ao iniciar uma disciplina, será informado sobre as normas e critérios de avaliação que serão considerados.


Não há um limite mínimo de avaliações a serem realizadas, mas, dado o caráter qualitativo
do sistema, é indicado que sejam realizadas ao menos duas em cada disciplina durante o
período letivo. E serão apoiadas e incentivadas as iniciativas de se gerar novos documentos de avaliação, como atividades extraclasse, tarefas em grupo, listas de exercícios, atividades em sala e/ou em laboratório, observações do professor, auto-avaliação, seminários, exposições, projetos, sempre no intuito de se viabilizar um processo de avaliação que não seja apenas qualitativo, mas que se aproxime de uma avaliação contínua.

Assim, propõe-se não apenas a avaliação de conteúdos, mas de estratégias cognitivas e
habilidades e competências desenvolvidas. Esse mínimo de duas sugere a possibilidade de ser feita uma avaliação diagnóstica logo no início do período, que identifique a capacidade do aluno em lidar com conceitos que apoiarão o desenvolvimento de novos conhecimentos e o quanto ele conhece dos conteúdos a serem discutidos na disciplina, e outra no final do período, que possa identificar a evolução do aluno relativamente ao estágio de diagnóstico inicial. De posse do diagnóstico inicial, o próprio professor poderá ser mais eficiente na mediação com os alunos no desenvolvimento da disciplina. Por fim, deverá ser levado em alta consideração o processo evolutivo descrito pelas sucessivas avaliações no desempenho do aluno para que se faça a atribuição de um Conceito a ele.

Cabe ressaltar que os critérios de recuperação do curso da UFABC são atualmente regulamentados pela Resolução ConsEPE Nº. 182 (ou outra resolução que venha a substituí-la).





\newpage
\section {Infraestrutura}

A UFABC é uma universidade multicampi. Tanto o campus de Santo André como o campus de São Bernardo do Campo possuem biblioteca, conexão de internet de alta velocidade, laboratórios didáticos de experimentação, de ensino e computação, laboratórios de pesquisa, biotérios de criação e manutenção de animais de experimentação, setores administrativos, salas de reunião e salas de docentes.

\subsection{Campus Santo André}
  
O 'Bloco A' de edifícios do Campus Santo André mede cerca de 39.000 m$^2$ onde está localizada a maior parte das salas de aula, laboratórios de pesquisa e salas de docentes daquele campus. Esta obra possui três torres principais, cada um relacionado a um centro desta universidade: Centro de Engenharias, Modelagem e Ciências Sociais Aplicadas (CECS), Centro de Ciências Naturais e Humanas (CCNH) e Centro de Matemática, Computação e Cognição (CMCC). As três edificações estão interligadas por áreas comuns nos primeiros três andares de cada prédio. Nessas áreas comuns estão instaladas salas de aula da graduação e setores administrativos. A ideia de continuidade física entre as áreas da UFABC está em consonância com seu projeto de criação que visa a interdisciplinaridade como sua principal meta. Algumas salas de docentes, laboratórios didáticos e de pesquisa, e salas de aula também estão localizados no prédio de 11 andares adjacente ao 'Bloco A', denominado 'Bloco B'. Por fim, O 'Bloco L', com área construída de mais de 16.800 m$^2$ abriga 72 laboratórios didáticos e de pesquisa dos três Centros, além de lanchonetes, almoxarifado entre outros.

\subsection{Campus São Bernardo do Campo} 

O campus de São Bernardo do Campo possui laboratórios didáticos para experimentação e computação nos Blocos 'Alfa' e 'Tau'. O Bloco 'Beta' abriga a biblioteca, anfiteatros e um amplo auditório de 400 lugares. Estão alocados nos laboratórios didáticos do bloco Alfa diversos modelos anatômicos e sistemas de ensino de fisiologia (i-Works). Estão previstos ainda outros edifícios, já em construção, abrigando laboratórios didáticos específicos das Engenharias (Bloco 'Omega'), laboratórios de pesquisa (Bloco 'Zeta') e um biotério de caráter multiusuário de criação e manutenção de animais de experimentação.

\subsection{Laboratórios Didáticos}
A Pró-Reitoria de Graduação possui em sua infraestrutura um grupo de laboratórios compartilhados por todos os cursos de graduação. A Coordenadoria dos Laboratórios Didáticos (CLD), vinculada à PROGRAD, é responsável pela gestão administrativa dos laboratórios didáticos e por realizar a interface entre docentes, discentes e técnicos de laboratório nas diferentes áreas, de forma a garantir o bom andamento dos cursos de graduação, no que se refere às atividades práticas em laboratório.

A CLD é composta por um Coordenador dos Laboratórios Úmidos, um Coordenador dos Laboratórios Secos e um Coordenador dos Laboratórios de Informática e Práticas de Ensino, bem como equipe técnico-administrativa. 

Dentre as atividades da CLD destacam-se o atendimento diário a toda comunidade acadêmica; a elaboração de Política de Uso e Segurança dos Laboratórios Didáticos e a análise e adequação da alocação de turmas nos laboratórios em cada quadrimestre letivo, garantindo a adequação dos espaços às atividades propostas em cada disciplina e melhor utilização de recursos da UFABC, o gerenciamento da infraestrutura dos laboratórios didáticos, materiais, recursos humanos, treinamento, encaminhamento para manutenção preventiva e corretiva de todos os equipamentos. 

Os laboratórios são dedicados às atividades didáticas práticas que necessitam de infraestrutura específica e diferenciada, não atendidas por uma sala de aula convencional. São quatro diferentes categorias de laboratórios didáticos disponíveis para os usos dos cursos de graduação da UFABC: secos, úmidos, de informática e de prática de ensino.

\begin{itemize}
	\item \underline{Laboratórios Didáticos Secos} são espaços destinados às aulas da graduação que necessitem de uma infraestrutura com bancadas e instalação elétrica e/ou instalação hidráulica e/ou gases, uso de kits didáticos e mapas, entre outros.
	\item \underline{Laboratórios Didáticos Úmidos} são espaços destinados às aulas da graduação que necessitem manipulação de agentes químicos ou biológicos, uma infraestrutura com bancadas de granito, com capelas de exaustão e com instalações hidráulica, elétrica e de gases
	\item \underline{Laboratórios Didáticos Práticas de Ensino} são espaços destinados ao suporte dos cursos de licenciatura, desenvolvimento de habilidades e competências para docência da educação básica, podendo ser úteis também para desenvolvimentos das habilidades e competências para docência do ensino superior.
	\item \underline{Laboratórios Didáticos de Informática} são espaços para aulas utilizando recursos de tecnologia de informação como microcomputadores, acesso à internet, linguagens de programação, softwares, hardwares e periféricos.
\end{itemize}

Anexo aos laboratórios há sala de suporte técnico que acomodam quatro técnicos de laboratório, cumprindo as seguintes funções: Nos períodos extra aula, auxiliam a comunidade no que diz respeito à atividades de graduação, pós-graduação e extensão em suas atividades práticas (projetos de disciplinas, iniciação científica, mestrado e doutorado), participam dos processos de compras levantando a minuta dos materiais necessários, fazem controle de estoque de materiais, bem como cooperam com os professores durante a realização testes e experimentos que serão incorporados nas disciplinas e preparação do laboratório para a aula prática. Nos períodos de aula, oferecem apoio para os professores e alunos durante o experimento, repondo materiais, auxiliando no uso de equipamentos e prezando pelo bom uso dos materiais de laboratório. Para isso, os técnicos são alocados previamente em determinadas disciplinas, conforme a sua formação (eletrônica, eletrotécnica, materiais, mecânica, mecatrônica, edificações, química, biologia, informática, etc). Os técnicos trabalham em esquema de horários alternados, possibilitando o apoio às atividades práticas ao longo de todo período de funcionamento da UFABC, das 08 às 23h.

Além dos técnicos, a sala de suporte armazena alguns equipamentos e kits didáticos utilizados nas disciplinas. Há também a sala de suporte técnico, que funciona como almoxarifado, armazenando demais equipamentos e kits didáticos utilizados durante o quadrimestre.

A UFABC dispõe ainda de uma oficina mecânica de apoio, com quatro técnicos especializados na área e atende a demanda de toda comunidade acadêmica (centros, graduação, extensão e prefeitura universitária), para a construção e pequenas reparações de kits didáticos e dispositivos para equipamentos usados na graduação e pesquisa, além do auxílio à discente na construção e montagem de trabalhos de graduação, e pós, projetos acadêmicos como; BAJA, Aerodesign, etc... A oficina mecânica atende no horário das 08h00 horas às 17h00 horas. Esta oficina está equipada com as seguintes máquinas operatrizes: torno CNC, centro de usinagem CNC, torno mecânico horizontal, fresadora universal, retificadora plana, furadeira de coluna, furadeira de bancada, esmeril, serra de fita vertical, lixadeira, serra de fita horizontal, prensa hidráulica, máquina de solda elétrica TIG, aparelho de solda oxi-acetilênica, calandra, curvadora de tubos, guilhotina e dobradora de chapas. Além disso, a oficina mecânica possui duas bancadas e uma grande variedade de ferramentas para trabalhos manuais: chaves para aperto, limas, serras manuais, alicates de diversos tipos, torquímetros, martelos e diversas ferramentas de corte de uso comum em mecânica, como também, ferramentas manuais elétricas: furadeiras manuais, serra tico-tico, grampeadeira, etc. Também estão disponíveis vários tipos de instrumentos de medição comuns em metrologia: paquímetros analógicos e digitais, micrômetros analógicos com batentes intercambiáveis, micrômetros para medição interna, esquadros e goniômetros, traçadores de altura, desempeno, escalas metálicas, relógios comparadores analógicos e digitais e calibradores. Com estes equipamentos e ferramentas, é possível a realização de uma ampla gama de trabalhos de usinagem, ajustes, montagem e desmontagem de máquinas e componentes mecânicos.

A alocação de laboratórios didáticos para as turmas de disciplinas com carga horária prática ou aquelas que necessitem do uso de um laboratório é feita pelo coordenador do curso, a cada quadrimestre, durante o período estipulado pela Pró-Reitoria de Graduação.

O docente da disciplina com carga horária alocada nos laboratórios didáticos é responsável pelas aulas práticas da disciplina, não podendo se ausentar do laboratório durante a aula prática.
Atividades como treinamentos, instalação ou manutenção de equipamentos nos laboratórios didáticos ou aulas pontuais são previamente agendadas com a equipe técnica responsável e acompanhadas por um técnico de laboratório.

Como os laboratórios são compartilhados, todos os cursos podem realizar de diferentes atividades didáticas dentro dos diversos laboratórios, otimizando o uso dos recursos materiais e ampliando as possiblidades didáticas dos docentes da UFABC e a prática da interdisciplinaridade, respeitando as necessidades de cada disciplina ou aula de acordo com a classificação do laboratório e dos materiais e equipamentos disponíveis nele.

\subsection{Sistema De Bibliotecas – SISBI}

O Sistema de Bibliotecas da UFABC, cuja finalidade é atender as demandas informacionais da comunidade universitária e científica interna e externa à Universidade, é formado por unidades de bibliotecas localizadas nos Campi de Santo André e São Bernardo do Campo, responsáveis por atender e apoiar a comunidade universitária em suas atividades de ensino pesquisa e extensão, de forma articulada e pautada na proposta interdisciplinar do projeto pedagógico e de seu plano de desenvolvimento institucional.

As Bibliotecas que compõem o Sistema possuem amplo e diversificado acervo, com aproximadamente 100.000 exemplares de livros físicos e 42.000 títulos de livros eletrônicos, sendo, todas as coleções da editora Springer Nature entre os anos de 2.005 e 2.014, todos os títulos publicados pela editora Wiley em 2.016 e pelos títulos da editora Ebsco referentes a coleção EbscoHost. E, em complemento, títulos resultantes de assinaturas anuais com demais editoras, como: Elsevier, Cengage Learning e Wiley. Além da filmoteca que conta com mais de 1.000 títulos de filmes.
O SisBi ainda, dispõe de sistema (SophiA) que permite o acesso ao seu catálogo e portal na internet para acesso às informações sobre seus serviços e a conteúdos externos, como: sistema Scielo que contempla seleção de periódicos científicos brasileiros, sistema Biblioteca Digital Brasileira de Teses e Dissertações (BDTD); sistema COMUT que permite a obtenção de cópias de documentos técnico-científicos disponíveis nos acervos das principais bibliotecas brasileiras e em serviços de informações internacionais; Portal de Periódicos da CAPES, que oferece uma seleção das mais importantes fontes de informação científica e tecnológica, de acesso gratuito na Web. Atualmente, o portal dispõe de 34.457 periódicos eletrônicos, relacionados às diversas áreas do conhecimento e, ainda, acesso a mais de 2.000 bases de dados; dentre outros.

Convênios também são estabelecidos pelo SisBi, entre os mais significativos o serviço de Empréstimo Entre Bibliotecas (EEB), que estabelece a cooperação e potencializa a utilização do acervo das instituições universitárias participantes, favorecendo a disseminação da informação entre universitários e pesquisadores de todo o país. Outro convênio a ser notado é com o IBGE, que tem por objetivo ampliar para a sociedade, o acesso às informações produzidas por meio de cooperação técnica com o Centro de Documentação e Disseminação de Informações do IBGE. Assim, o SisBi passou a ser depositário das publicações editadas por esse órgão.

As unidades de bibliotecas atendem a comunidade de segunda a sexta, de 8 às 22h, mantendo-se em uma estrutura física com área total de 4.529 m², onde se distribuem 521 assentos; além de terminais de consulta ao acervo. Buscando promover o exercício a reflexão crítica nos espaços universitários, bem como a interação com os diversos públicos, desenvolve ainda, programas e projetos culturais como: CineArte, exibido também ao ar livre; PublicArte; Saraus e Exposições.

\subsection{Tecnologias Digitais}

As Tecnologias de Informação e Comunicação (TIC) têm sido cada vez mais utilizadas no processo de ensino e aprendizagem. Sua importância não está restrita apenas à oferta de disciplinas e cursos semipresenciais, ou totalmente a distância, ocupando um espaço importante também como mediadoras em disciplinas e cursos presenciais.
As salas de aula da UFABC são equipadas com projetor multimídia e um computador, e as disciplinas práticas, que demandam o uso de computadores e internet, são ministradas em laboratórios equipados com 30-48 computadores com acesso à Internet, projetor multimídia e softwares relacionados às atividades desenvolvidas. Estão disponíveis também 10 lousas digitais, distribuídas em salas específicas de cada centro. Para o uso dessas ferramentas e infraestrutura, os docentes contam com o suporte técnico do Núcleo de Tecnologia da Informação (NTI) e da Coordenação de Laboratórios Didáticos (CLD).

\subsection{Ambiente Virtual de Aprendizagem (AVA)}
Com o intuito de estimular a integração das TIC, a UFABC incentiva o uso de um Ambiente Virtual de Aprendizagem (TIDIA 4 ou Moodle) como ferramenta de apoio ao ensino presencial e semipresencial nas diversas disciplinas. O AVA pode possibilitar a interação entre alunos e professores por meio de ferramentas de comunicação síncrona (e.g. bate papo/ chat) e assíncrona (e.g. fórum de discussões, correio eletrônico), além de funcionar como repositório de conteúdo didáticos, e permitir propostas de atividades individuais e colaborativas.

\subsection{Núcleo Educacional de Tecnologias e Línguas (NETEL)}
No âmbito da utilização das TIC nas diferentes modalidades de ensino e aprendizagem (presencial, semipresencial e a distância), a UFABC conta com o apoio do Núcleo Educacional de Tecnologias e Línguas2 (http://netel.ufabc.edu.br//). O NETEL está organizado em seis divisões (Cursos, Design e Inovação Educacional, Tecnologias da Informação, Audiovisual, Comunicação e idiomas), e oferece cursos de extensão e oficinas para capacitação de docentes interessados na integração de novas metodologias e tecnologias digitais nas suas práticas de ensino. Os cursos e oficinas são oferecidos periodicamente, nas modalidades presencial e semipresencial, e possibilitam a formação e a atualização em diferentes domínios, por exemplo: docência com tecnologias, desenvolvimento de objetos de aprendizagem, jogos digitais educacionais, videoaulas, webconferência, lousa digital, metodologias ativas de ensino, ferramentas digitais de apoio à aprendizagem. Para apoiar a oferta de disciplinas na modalidade semipresencial, outras iniciativas formativas do NETEL são os cursos semipresenciais Planejamento de cursos virtuais, que se configura em uma oportunidade de reflexão e compartilhamento de ideias sobre estratégias, ferramentas e métodos que apoiam a criação de espaços virtuais de aprendizagem, e o curso Formação de Tutores para EAD, que têm como objetivo capacitar alunos de graduação e pós-graduação e pessoas interessadas em atuar como tutores/monitores. Para apoiar o docente na criação e oferta de disciplinas na modalidade semipresencial, o NETEL conta com uma equipe de profissionais da área de Design Instrucional e especialistas no desenvolvimento de recursos educacionais, como objetos de aprendizagem e jogos educacionais. O NETEL possui também uma divisão de audiovisual com infraestrutura completa de estúdio e equipamentos para gravação de videoaulas e podcasts. O estúdio proporciona apoio à comunidade acadêmica em diversos projetos de extensão e outras iniciativas que demandam o uso de recursos audiovisuais como filmagem de aulas, palestras etc. Em 2019 o NETEL passou a integrar uma nova divisão, divisão de idiomas, no qual é responsável por desenvolver a política linguística da UFABC através de ofertas de cursos de línguas gratuitos e presenciais como: cursos de inglês; português espanhol; e francês.

Por se tratar de uma instituição que busca excelência no uso das TIC, muitos pesquisadores da UFABC têm desenvolvido pesquisas interdisciplinares nas áreas de Educação, Ensino, Ciência da Computação, Comunicação etc., com o objetivo de compreender as potencialidades de uso das TIC e sua influência nos processos de ensino e aprendizagem. Neste contexto, os docentes envolvidos no núcleo juntamente com parceiros da UFABC desenvolvem pesquisas com a finalidade de renovação e atualização constante das TICs tanto no ensino quanto apoio ao mesmo.

\subsection{Oferta de disciplinas semipresenciais}

Em consonância com a Portaria MEC Nº 2.117, DE 6 DE DEZEMBRO DE 2019 (disponível em \url{https://www.in.gov.br/en/web/dou/-/portaria-n-2.117-de-6-de-dezembro-de-2019-232670913}), que orienta sobre a oferta, por Instituições de Educação Superior (IES), de disciplinas na modalidade a distância em cursos de graduação presencial. Neste sentido, as coordenações dos cursos de graduação juntamente com o seu corpo docente poderão decidir como farão o uso desta portaria no sentido de incluir componentes curriculares que, no todo ou em parte, utilizem a modalidade de ensino semipresencial ou a distância, desde que esta oferta não ultrapasse 40\% (quarenta por cento) da carga horária do curso. Uma mesma disciplina do curso poderá ser ofertada nos formatos presencial e semipresencial, com Planos de Ensino devidamente adequados à sua oferta. O número de créditos atribuídos a um componente curricular será o mesmo em ambos os formatos. Para fins de registros escolares, não existe qualquer distinção entre as ofertas presencial ou semipresencial de um dado componente curricular. Portanto, em ambos os casos, as TICs, o papel dos tutores e dos docentes, a metodologia de ensino, e o material didático a serem utilizados deverão ser detalhados em proposta de Plano de Aula a ser avaliado pela coordenação do curso antes de sua efetiva implantação. O uso desta portaria é de grande importância pois, motiva o uso das TICs nas disciplinas de graduação favorecendo a renovação e modernização do ensino e criando oportunidade para o desenvolvimento das habilidades digitais tanto dos docentes quanto dos alunos da UFABC.

\subsection{Acessibilidade}

A UFABC possui um Núcleo de Acessibilidade, lotado na Pró-Reitoria de Assuntos Comunitários e Políticas Afirmativas (ProAP) responsável por executar as políticas de assistência estudantil direcionadas aos estudantes com deficiência da nossa comunidade. Essas ações e projetos visam eliminar as barreiras arquitetônicas, atitudinais e de comunicação promovendo a inclusão das pessoas com deficiência. É papel da ProAP dar suporte a estudantes com deficiência ou necessidades educacionais específicas, além de orientar a comunidade acadêmica nas questões que envolvem o atendimento educacional destes estudantes. Além disso, a fim de possibilitar à pessoa com deficiência viver de forma autônoma e participar de todos os aspectos da vida acadêmica, a ProAP preza pela disseminação do conceito de desenho universal, conforme disposto na legislação vigente. Orientar o corpo docente, acolher aos estudantes respeitando suas especificidades, difundir e oferecer Tecnologias Assistivas, dar suporte de monitoria acadêmica as disciplinas da graduação, disponibilizar tradução e interpretação de LIBRAS, além da oferta de alguns programas de subsídios financeiros propostos pelo Plano Nacional de Assistência Estudantil - PNAES, também fazem parte dos programas em acessibilidade da UFABC. 




\newpage
\section {Docentes}

\begin{longtable}{|l|l|c|c|}
\hline
Docente & Área de formação & Titulação & dedicação\\
\hline\hline
Alexandre Donizeti Alves & Ciência da Computação & Doutor & DE \\
Alexandre Noma & Ciência da Computação & Doutor & DE \\
Ana Lígia Barbour Scott & Biofísica Molecular & Doutora & DE \\
André Luiz Brandão & Ciência da Computação & Doutor & DE \\
Antonio Sérgio Kimus Braz & Ciências Biológicas - Genética & Doutor & DE \\
Aritanan Borges Garcia Gruber & Ciência da Computação - Pesquisa Operacional & Doutor & DE \\
Bruno Augusto Dorta Marques & Ciência da Computação & Doutor & DE \\
Carla Lopes Rodriguez & Artes Visuais & Doutora & DE \\
Carla Negri Lintzmayer & Ciência da Computação & Doutora & DE \\
Carlo Kleber da Silva Rodrigues & Engenharia de Sistemas e Computação & Doutor & DE \\
Carlos Alberto Kamienski & Ciência da Computação & Doutor & DE \\
Carlos da Silva dos Santos & Ciência da Computação & Doutor & DE \\
Cláudio Nogueira de Meneses & Engenharia Industrial e de Sistemas & Doutor & DE \\
Cristiane Maria Sato & Otimização e Combinatória & Doutora & DE \\
Daniel Morgato Martin & Matemática & Doutor & DE \\
David Corrêa Martins Júnior & Ciência da Computação & Doutor & DE \\
Debora Maria Rossi de Medeiros & Ciência da Computação e Matemática Computacional & Doutora & DE \\
Denis Gustavo Fantinato & Engenharia Elétrica & Doutor & DE \\
Denise Hideko Goya & Ciência da Computação & Doutora & DE \\
Diogo Santana Martins & Ciência da Computação e Matemática Computacional & Doutor & DE \\
Edson Pinheiro Pimentel & Engenharia Eletrônica e Computação & Doutor & DE \\
Emílio de Camargo Francesquini & Ciência da Computação & Doutor & DE \\
Fabrício Olivetti de França & Engenharia Elétrica & Doutor & DE \\
Fedor Pisnitchenko & Matemática Aplicada & Doutor & DE \\
Fernando Teubl Ferreira & Engenharia Elétrica & Doutor & DE \\
Flávio Eduardo Aoki Horita & Ciência da Computação e Matemática Computacional & Doutor & DE \\
Francisco de Assis Zampirolli & Engenharia Elétrica & Doutor & DE \\
Francisco Isidro Massetto & Engenharia Elétrica & Doutor & DE \\
Francisco Javier Ropero Peláez & Neurociências & Doutor & DE \\
Gordana Manic & Ciência da Computação & Doutora & DE \\
Guiou Kobayashi & Engenharia Elétrica & Doutor & DE \\
Gustavo Sousa Pavani & Engenharia Elétrica & Doutor & DE \\
Harlen Costa Batagelo & Engenharia Elétrica & Doutor & DE \\
Itana Stiubiener & Engenharia Elétrica  & Doutora & DE \\
Jair Donadelli Júnior & Matemática Aplicada & Doutor & DE \\
Jerônimo Cordoni Pellegrini & Ciência da Computação & Doutor & DE \\
Jesús Pascual Mena-Chalco & Ciência da Computação & Doutor & DE \\
João Marcelo Borovina Josko & Ciência da Computação & Doutor & DE \\
João Paulo Gois & Ciência da Computação e Matemática Computacional & Doutor & DE \\
João Ricardo Sato & Estatística & Doutor & DE \\
José Artur Quilici Gonzalez & Engenharia Elétrica & Doutor & DE \\
Juliana Cristina Braga & Computação Aplicada & Doutora & DE \\
Karla Vittori & Engenharia Elétrica & Doutora & DE \\
Luiz Carlos da Silva Rozante & Bioinformática & Doutor & DE \\
Márcio Katsumi Oikawa & Ciência da Computação & Doutor & DE \\
Maria das Graças Bruno Marietto & Engenharia Eletrônica & Doutora & DE \\
Maycon Sambinelli &Ciência da Computação  & Doutor & DE \\
Monael Pinheiro Ribeiro & Engenharia Eletrônica & Doutor & DE \\
Nunzio Marco Torrisi & Engenharia Informática & Doutor & DE \\
Paulo Henrique Pisani & Ciência da Computação e Matemática Computacional & Doutor & DE \\
Paulo Roberto Miranda Meirelles & Ciência da Computação & Doutor & DE \\
Raphael Yokoingawa de Camargo & Ciência da Computação & Doutor & DE \\
Rogério Rossi & Engenharia Elétrica & Doutor & DE \\
Ronaldo Cristiano Prati & Ciência da Computação e Matemática Computacional & Doutor & DE \\
Saul de Castro Leite & Modelagem Computacional & Doutor & DE \\
Sílvia Cristina Dotta & Educação & Doutora & DE \\
Valério Ramos Batista & Matemática & Doutor & DE \\
Vera Nagamuta & Ciência da Computação & Doutora & DE \\
Vinicius Cifú Lopes & Matemática & Doutor & DE \\
Vladimir Emiliano Moreira Rocha & Engenharia Elétrica & Doutor & DE \\
Wagner Tanaka Botelho & Engenharia & Doutor & DE \\
\hline

\end{longtable}


\subsection{Núcleo Docente Estruturante}

O Núcleo Docente Estruturante (NDE) do BCC é constituído conforme as orientações da Resolução ConsEPE no 179, de 21 de junho de 2014, que institui o NDE no âmbito dos cursos 
de graduação da UFABC e estabelece suas normas de funcionamento. 

São atribuições do Núcleo Docente Estruturante (NDE).

\begin{enumerate}
\item [I -] contribuir para a consolidação do perfil profissional do egresso do curso;
\item [II -] zelar pela integração curricular interdisciplinar entre as diferentes atividades de ensino
constantes no currículo;
\item [III -] indicar formas de incentivo ao desenvolvimento de linhas de pesquisa e extensão e sua
articulação com a pós-graduação, oriundas das necessidades do curso de graduação, das
exigências do mundo do trabalho, sintonizadas com as políticas públicas próprias à área de
conhecimento; e
\item [IV -] zelar pelo cumprimento das Diretrizes Curriculares Nacionais para o Curso e demais
marcos regulatórios.
\end{enumerate}

A composição atual foi nomeada por meio da Portaria CMCC no. 15/2016, sendo formada 
pelos seguintes docentes:

\begin{itemize}
\item Márcio Katsumi Oikawa (Presidente)
\item Carlos da Silva dos Santos
\item Carlos Alberto Kamienski
\item Cristiane Maria Sato
\item Daniel Morgato Martin
\item Denise Hideko Goya
\item Diogo Santana Martins
\item Jerônimo Cordoni Pellegrini
\item João Marcelo Borovina Josko
\end{itemize}

\newpage
\section {Sistema de Avaliação do Projeto de Curso}

Buscando conhecer, avaliar e aprimorar a qualidade e os compromissos de sua missão, a
UFABC tem implementado mecanismos de avaliação permanente para a efetividade do processo de ensino-aprendizagem, visando compatibilizar a oferta de vagas, os objetivos do curso, o perfil do egresso e a demanda de profissionais no mercado de trabalho para o curso.

Um dos mecanismos adotado pela Coordenação do Curso para avaliação do Projeto
Político Pedagógico do Bacharelado em Ciência da Computação é a análise e o
estabelecimento de ações, a partir dos resultados obtidos pelo Curso e pela Universidade no
Sistema Nacional de Avaliação da Educação Superior (SINAES), regulamentado e instituído
pela Lei n° 10.681, de 14 de abril de 2004.

No Decreto n° 5.773, de 9 de maio de 2006, que dispõe sobre o exercício das funções de
regulação, supervisão e avaliação de Instituições de Educação Superior (IES) e Cursos
superiores de Graduação e Sequenciais no sistema federal de ensino, no seu artigo 1°,
parágrafo 3°, lê-se que a avaliação realizada pelo SINAES constitui referencial básico para os
processos de regulação e supervisão da educação superior, a fim de promover sua qualidade.

No que tange propriamente à estruturação da avaliação estabelecida pelo SINAES, são
considerados três tipos de avaliação:

\begin{enumerate}
	\item Avaliação institucional, que contempla um processo de autoavaliação realizado pela Comissão Própria de Avaliação (CPA) da Instituição de Educação Superior, está já implantada na UFABC, com as seguintes portarias de criação nos últimos anos:
	\begin{enumerate}
		\item Portaria 108, de 28 de fevereiro de 2013, que institui a Comissão Própria de Avaliação e demais portarias correlatas. Disponíveis em
\url{https://www.ufabc.edu.br/administracao/comissoes/cpa/criacao}. Acesso em: 24 jun. 2022.
		\item Regimento interno da CPA - UFABC. Disponível em \url{https://www.ufabc.edu.br/administracao/comissoes/cpa/regimento-interno}. Acesso em 24 jun. 2022.
	\end{enumerate}

	\item Avaliação de curso, que considera um conjunto de avaliações: avaliação dos pares (in loco), avaliação dos estudantes (questionário de Avaliação Discente da Educação Superior – ADES, enviado à amostra selecionada para realização do Exame Nacional de Desempenho de Estudantes - ENADE), avaliação da Coordenação (questionário específico) e dos Professores do Curso e da CPA. Temos os seguintes relatórios produzidos nos últimos anos:
	\begin{enumerate}
		\item Relatório CPA 2022. Disponível em:
		\url{https://www.ufabc.edu.br/images/comissoes/cpa/relatorio_cpa_2022_vfinal_16_04_2022_.pdf}. Acesso em 24 jun. 2022.
		\item Relatório final CPA 2021. Disponível em:
		\url{https://www.ufabc.edu.br/images/comissoes/cpa/relatorio_cpa_2021_final_31_03_2021_entregue.pdf}. Acesso em 24 jun. 2022
		\item Relatório parcial CPA 2021. Disponível em:
\url{https://www.ufabc.edu.br/images/comissoes/cpa/relatorio_cpa_2020.pdf}. Acesso em: 24 jun. 2022.
		\item Relatório do Grupo de Trabalho sobre Problemas e Oportunidades de Melhoria na Infraestrutura Pedagógica da UFABC. Disponível em \url {https://www.ufabc.edu.br/images/comissoes/cpa/relatorio_gt_infraestrutura_pedagogica.pdf}. Acesso em 24 jun. 2022.
		\item Demais relatórios da CPA - UFABC. Disponíveis em  \url{https://www.ufabc.edu.br/administracao/comissoes/cpa}
	\end{enumerate}

	\item Avaliação do Desempenho dos estudantes ingressantes e concluintes, que corresponde à aplicação do ENADE aos estudantes que preenchem os critérios estabelecidos pela legislação vigente (incluem neste exame a prova e os questionários dos alunos, do Coordenador de Curso e da percepção do alunado sobre a prova).
\end{enumerate}

Com o apoio do NDE, os relatórios são utilizados para avaliar a estrutura do curso sob diferentes perspectivas: do discente, do docente, do resultado de exames de acompanhamento externo. Com base nesses elementos, são identificados e discutidos temas levantados sobre pontos positivos e negativos da concepção e execução do curso, como por exemplo:
\begin{itemize}
	\item Adequação da oferta de turmas de disciplinas;
	\item Nível de aproveitamento em disciplinas;
	\item Panorama geral de orientação de alunos para estágios, PGCs, iniciação científica e outras modalidades;
	\item Criação de disciplinas novas com cobertura de assuntos recentes à Computação;
	\item Reformulação de disciplinas;
	\item Adequação de ementas;
	\item Criação de grupos de trabalho;
	\item Outros temas.
\end{itemize}

A aplicação e divulgação dos resultados de discussões realizadas pela coordenação de curso, colegiado de curso e NDE são expostas e deliberadas em reunião plenária, excetuando-se casos em que os temas fogem de seu escopo.

\newpage
\section {Disciplinas}

\subsection{Disciplinas Obrigatórias do BC\&T}
A Tabela \ref{tab:disciplinas_bct}, a seguir, apresenta a lista de todas as disciplinas obrigatórias do BC\&T, que compõe parte do currículo do BCC.

\begin{table}[!h]
\begin{longtable}{|l|p{.6\textwidth}|c|c|}
\hline
Código & Nome da disciplina & Créditos (T-P-I) & Carga-horária\\
\hline\hline
BIS0005-15 & Bases Computacionais da Ciência & 2 (0-2-2) & 24h\\
\hline
BIJ0207-15 & Bases Conceituais da Energia & 2 (2-0-4) & 24h\\
\hline
BIR0004-15 & Bases Epistemológicas da Ciência Moderna & 3 (3-0-4) & 36h\\
\hline
BCS0001-15 & Base Experimental das Ciências Naturais & 3 (0-3-2) & 36h\\
\hline
BIS0003-15 & Bases Matemáticas & 4 (4-0-5) & 48h\\
\hline
BCL0306-15 & Biodiversidade: Interações entre Organismos e Ambiente & 3 (3-0-4) & 36h\\
\hline
BCL0308-15 & Bioquímica: Estrutura, Propriedades e Funções de Biomoléculas & 5 (3-2-6) & 60h\\
\hline
BIR0603-15 & Ciência, Tecnologia e Sociedade & 3 (3-0-4) & 36h\\
\hline
BCM0506-15 & Comunicação e Redes & 3 (3-0-4) & 36h\\
\hline
BIK0102-15 & Estrutura da Matéria & 3 (3-0-4) & 36h\\
\hline
BIQ0602-15 & Estrutura e Dinâmica Social & 3 (3-0-4) & 36h \\
\hline
BIL0304-15 & Evolução e Diversificação da Vida na Terra & 3 (3-0-4) & 36h\\
\hline
BCJ0203-15 & Fenômenos Eletromagnéticos & 5 (4-1-6) & 60h\\
\hline
BCJ0204-15 & Fenômenos Mecânicos & 5 (4-1-6) & 60h \\
\hline
BCJ0205-15 & Fenômenos Térmicos & 4 (3-1-4) & 48h\\
\hline
BCK0103-15 & Física Quântica & 3 (3-0-4) & 36h\\
\hline
BCN0402-15 & Funções de Uma Variável & 4 (4-0-6) & 48h\\
\hline
BCN0407-15 & Funções de Várias Variáveis & 4 (4-0-4) & 48h \\
\hline
BCN0404-15 & Geometria Analítica & 3 (3-0-6) & 36h\\
\hline
BCK0104-15 & Interações Atômicas e Moleculares & 3 (3-0-4) & 36h\\
\hline
BIN0406-15 & Introdução à Probabilidade e Estatística & 3 (3-0-4) & 36h\\
\hline
BCN0405-15 & Introdução às Equações Diferenciais Ordinárias & 4 (4-0-4) & 48h \\
\hline
BCM0504-15 & Natureza da Informação & 3 (3-0-4) & 36h\\
\hline
BCM0505-15 & Processamento da Informação & 5 (3-2-5) & 60h\\
\hline
BCS0002-15 & Projeto Dirigido & 2 (0-2-10) & 24h\\
\hline
BCL0307-15 & Transformações Químicas & 5 (3-2-6) & 60h\\

\hline
\end{longtable}
\caption{Lista de disciplinas obrigatórias do BC\&T}
\label{tab:disciplinas_bct}
\end{table}


\subsection {Disciplinas Obrigatórias do BCC}

\begin{longtable}{|p{.15\textwidth}|p{.55\textwidth}|p{.15\textwidth}|p{.15\textwidth}|}
\hline
Código & Nome da disciplina & Créditos (T-P-I) & Carga-horária\\
\hline\hline
MCTB001-17 & Álgebra Linear & 6 (6-0-5) & 72h \\
\hline
MCTA001-17 & Algoritmos e Estruturas de Dados I & 4 (2-2-4) & 48h \\
\hline
MCTA002-17 & Algoritmos e Estruturas de Dados II & 4 (2-2-4) & 48h \\
\hline
MCTA003-17 & Análise de Algoritmos & 4 (4-0-4) & 48h \\
\hline
MCTA004-17 & Arquitetura de Computadores & 4 (4-0-4) & 48h \\
\hline
MCTA037-17 & Banco de Dados & 4 (3-1-4) & 48h \\
\hline
MCTA006-17 & Circuitos Digitais & 4 (3-1-4) & 48h \\
\hline
MCTA007-17 & Compiladores & 4 (3-1-4) & 48h \\
\hline
MCTA008-17 & Computação Gráfica & 4 (3-1-4) & 48h  \\
\hline
MCTA009-13 & Computadores, Ética e Sociedade & 2 (2-0-4) & 24h \\
\hline
MCTA033-15 & Engenharia de Software & 4 (4-0-4) & 48h \\
\hline
MCTA014-15 & Inteligência Artificial & 4 (3-1-4) & 48h \\
\hline
MCTA015-13 & Linguagens Formais e Automata & 4 (3-1-4) & 48h \\
\hline
NHI2049-13 & Lógica Básica & 4 (4-0-4) & 48h \\
\hline
MCTB019-17 & Matemática Discreta & 4 (4-0-4) & 48h \\
\hline
MCTA016-13 & Paradigmas de Programação & 4 (2-2-4) & 48h \\
\hline
MCTA028-15 & Programação Estruturada & 4 (2-2-4) & 48h \\
\hline
MCTA017-17 & Programação Matemática & 4 (3-1-4) & 48h \\
\hline
MCTA018-13 & Programação Orientada a Objetos & 4 (2-2-4) & 48h \\
\hline
MCTA029-17 & Projeto de Graduação em Computação I & 8 (0-8-8) & 96h \\
\hline
MCTA030-17 & Projeto de Graduação em Computação II & 8 (0-8-8) & 96h \\
\hline
MCTA031-17 & Projeto de Graduação em Computação III & 8 (0-8-8) & 96h \\
\hline
MCTA022-17 & Redes de Computadores & 4 (3-1-4) & 48h \\
\hline
MCTA023-17 & Segurança de Dados & 4 (3-1-4) & 48h \\
\hline
MCTA024-13 & Sistemas Digitais & 4 (2-2-4) & 48h \\
\hline
MCTA025-13 & Sistemas Distribuídos & 4 (3-1-4) & 48h \\
\hline
MCTA026-13 & Sistemas Operacionais & 4 (3-1-4) & 48h\\
\hline
MCTA027-17 & Teoria dos Grafos & 4 (3-1-4) & 48h\\
\hline
\end{longtable}


\newpage
\section {Anexos}

\section {Disciplinas de Opção Limitada}
Lista de disciplinas de opção limitada do Bacharelado em Ciência da Computação atualizado em julho de 2022.

\begin{longtable}{|p{.1\textwidth}|p{.6\textwidth}|p{.15\textwidth}|p{.15\textwidth}|}
\hline
Código & Nome da disciplina & Créditos (TPI) & Carga-horária\\
\hline\hline
\end{longtable}


\end{document}



Disciplina obrigatória/optativa de Libras
(Dec. N° 5.626/2005)

Diretrizes Curriculares Nacionais do Curso.
	

O PPC está coerente com as Diretrizes Curriculares Nacionais? NSA para cursos que não têm Diretrizes Curriculares Nacionais.

2
	

Diretrizes Curriculares Nacionais da Educação Básica (Resolução CNE/CEB 4/2010)
	 O PPC está coerente com as Diretrizes Curriculares Nacionais da Educação Básica?

3
	

Diretrizes Curriculares Nacionais para Educação das Relações Étnico-raciais e para o Ensino de História e Cultura Afro-brasileira e Africana (Resolução CNE/CP N° 01 de 17 de junho de 2004)
	

A Educação das Relações Étnico-Raciais, bem como o tratamento de questões e temáticas que dizem respeito aos afrodescendentes estão inclusas nas disciplinas e atividades curriculares do curso?

4
	

Diretrizes Nacionais para a Educação em Direitos Humanos (Parecer CNE/CP nº 8, de 06/03/2012, que originou a Resolução CNE/CP nº 1, de 30/05/2012)
	 O PPC contempla as Diretrizes Nacionais para a Educação em Direitos Humanos?

5
	

Proteção dos Direitos da Pessoa com Transtorno do Espectro Autista (Lei nº 12.764 de 27 de dezembro de 2012)
	O PPC contempla a Proteção dos Direitos da Pessoa com Transtorno do Espectro Autista?

6
	

Titulação do corpo docente
(Art. 66 da Lei 9.394, de 20 de dezembro de 1996)
	

Todo corpo docente tem formação em pós-graduação?

7
	

Núcleo Docente Estruturante (NDE)
(Resolução CONAES N° 1, de 17/06/2010)
	

O NDE atende à normativa pertinente?

8
	

Denominação dos Cursos Superiores de Tecnologia
(Portaria Normativa N° 12/2006)
	

A denominação do curso está adequada ao Catálogo Nacional dos Cursos Superiores de Tecnologia?

9
	

Carga horária mínima, em horas – para Cursos Superiores de Tecnologia
(Portaria N°10, 28/07/2006; Portaria N° 1024, 11/05/2006; Resolução CNE/CP N°3,18/12/2002)
	

Desconsiderando a carga horária do estágio profissional supervisionado e do Trabalho de Conclusão de Curso – TCC, caso estes estejam previstos, o curso possui carga horária igual ou superior ao estabelecido no Catálogo Nacional dos Cursos Superiores de Tecnologia?

10
	

Carga horária mínima, em horas – para Bacharelados e Licenciaturas
Resolução CNE/CES N° 02/2007 (Graduação, Bacharelado, Presencial). Resolução CNE/CES N° 04/2009 (Área de Saúde, Bacharelado, Presencial). Resolução CNE/CP 2 /2002 (Licenciaturas)
Resolução CNE/CP Nº 1 /2006 (Pedagogia)
	

O curso atende à carga horária mínima em horas estabelecidas nas resoluções?

11
	

Tempo de integralização
Resolução CNE/CES N° 02/2007 (Graduação, Bacharelado, Presencial). Resolução CNE/CES N° 04/2009 (Área de Saúde, Bacharelado, Presencial).
Resolução CNE/CP 2 /2002 (Licenciaturas)
	

O curso atende ao Tempo de Integralização proposto nas resoluções?

12
	

Condições de acesso para pessoas com deficiência e/ou mobilidade reduzida
(Dec. N° 5.296/2004, com prazo de implantação das condições até dezembro de 2008)
	
	

A IES apresenta condições de acesso para pessoas com deficiência e/ou mobilidade reduzida?

13
	

Disciplina obrigatória/optativa de Libras
(Dec. N° 5.626/2005)
	

O PPC prevê a inserção de Libras na estrutura curricular do curso (obrigatória ou optativa, depende do curso)?

14
	

Prevalência de avaliação presencial para EaD
(Dec. N° 5.622/2005 art. 4 inciso II, § 2)
	

Os resultados dos exames presenciais prevalecem sobre os demais resultados obtidos em quaisquer outras formas de avaliação a distância?

15
	

Informações acadêmicas
(Portaria Normativa N° 40 de 12/12/2007, alterada pela Portaria Normativa MEC N° 23 de 01/12/2010, publicada em 29/12/2010)
	

As informações acadêmicas exigidas estão disponibilizadas na forma impressa e virtual?

16
	

Políticas de educação ambiental
(Lei nº 9.795, de 27 de abril de 1999 e Decreto Nº 4.281 de 25 de junho de 2002)
	

Há integração da educação ambiental às disciplinas do curso de modo transversal, contínuo e permanente?

17
	

Diretrizes Curriculares Nacionais para a Formação de Professores da Educação Básica, em nível superior, curso de licenciatura, de graduação plena (Resolução CNE/CP 1/2002 e Resolução CNE/CP 2/2002)
	 O PPC contempla as Diretrizes Curriculares Nacionais para a Formação de Professores da Educação Básica, em nível superior, curso de licenciatura, de graduação plena?
	 
	 

