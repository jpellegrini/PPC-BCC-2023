\section {Objetivos do Curso}

\subsection{Objetivo Geral}
Formação de profissionais com sólido conhecimento científico e tecnológico na área de Computação.

\subsection{Objetivos Específicos}
\begin{itemize}
	\item Incentivar o perfil pesquisador do estudante, visando promover o desenvolvimento científico e tecnológico da Ciência da Computação;
	\item Preparar o estudante para atuar profissionalmente em organizações, com espírito empreendedor e com responsabilidade social;
	\item Proporcionar atividades acadêmicas que estimulem a interdisciplinaridade, bem como a aplicação e renovação dos conhecimentos e habilidades de forma independente e inovadora, nos diversos contextos da atuação profissional;
	\item Formar estudantes que possam estar em sintonia com a nova realidade e necessidade do aprendizado contínuo e autônomo, exigido pela sociedade do conhecimento e organizações dos dias atuais;
	\item Promover no estudante uma postura ética e socialmente comprometida de seu papel e de sua contribuição no avanço científico, tecnológico e social do País.
	
\end{itemize}
Com base nesses objetivos, pode-se definir que o bacharel em Ciência da Computação da
UFABC deverá conhecer os fundamentos de sua ciência, suas raízes históricas e suas
interligações com outras ciências.

