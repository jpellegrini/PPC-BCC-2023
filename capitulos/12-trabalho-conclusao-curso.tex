\section{Trabalho de conclusão de curso}
\label{sec:trabalho_conclusao_curso}


O Trabalho de Conclusão de Curso (TCC) do BCC reger-se-á por norma específica.

O TCC representa o momento em que o estudante demonstra as competências e
habilidades desenvolvidas no curso em um projeto de maior complexidade, no qual
ele possa aplicar de modo integrado todos os conteúdos e técnicas com os quais
teve contato.
O aluno deve mostrar capacidade de avaliar a teoria/tecnologia existente de maneira
crítica, bem como de buscar novas tecnologias de forma independente.
Portanto, o TCC não pode se configurar como uma mera aplicação direta dos
métodos e tecnologias abordadas no curso, mas sim uma experiência na qual o
aluno deve revelar seu domínio da área de Computação e sua capacidade de buscar
soluções criativas para problemas relevantes e não triviais.
