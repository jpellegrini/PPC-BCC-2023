\section{Atividades de Extensão}
\label{sec:extensao}

O BCC obedece à Resolução ConsEPE n° 253/2022, regulamenta a inclusão de carga horária e ações de extensão nos cursos de graduação da UFABC. Entende-se, assim, que a extensão universitária é um processo interdisciplinar, político educacional, cultural, científico ou tecnológico que promove a interação transformadora entre a UFABC e outros setores da sociedade, por meio da produção e aplicação do conhecimento, em articulação permanente com o ensino e/ou a pesquisa.

As ações de extensão na UFABC podem ocorrer como programas, projetos, cursos, eventos, prestação de serviço, divulgação científica, extensão tecnológica, ou outras reconhecidamente regulamentadas pelas instâncias competentes da universidade. 
% No contexto particular do BCC, a carga horária em atividades de extensão deverá ser cumprida por meio dos seguintes recursos curriculares:
% \begin{itemize}
% 	\item Conclusão com aproveitamento de disciplinas que possuam carga horária extensionista, já prevista na ementa e no catálogo de disciplinas;
% 	\item \red{Trabalho de conclusão de curso, com registro de carga horária extensionista reconhecido pela coordenação do curso com limite máximo de 24 horas}; 
% 	\item \red{Estágio curricular, com registro de carga horária extensionista reconhecido pela coordenação do curso com limite máximo de 24 horas};
% 	\item Apresentação ou exposição de trabalhos em palestras, congressos e	seminários técnico-científicos, desde que de cunho extensionista;
% 	\item Publicação como autor de artigo em revista de cunho extensionista resultante de ações de extensão;
% 	\item Publicação de trabalho completo em anais (ou similares) de eventos, palestras, congressos e seminários técnico-científicos, desde que de cunho extensionista;
% 	\item Realização de Componente Curricular Livre (CCL) que tenha caráter extensionista, conforme Resolução ConsEPE n° 242 de 2020.
% 	\item Outras atividades com propósito de extensão reconhecidas pelas instâncias competentes da UFABC;
% \end{itemize}

Para a integralização do curso, cada discente deve cumprir, no mínimo,
10\% (dez por cento) da carga horária total do curso, conforme Tabela~\ref{tab:carga-ext}.

\begin{table}[h!]
  \caption{Distribuição da carga horária extensionista.}
  \label{tab:carga-ext}\centering
  \begin{tabular}{|p{0.3\textwidth}|p{0.1\textwidth}|p{0.2\textwidth}|}\hline
    Descrição & Horas & Carga horária mínima de componentes extensionistas\\\hline
    Carga horária \textbf{total do BCC} & 3280h & 328h\\\hline
    Carga horária \textbf{do BC\&T} & 2400h & 240h \\\hline
    Carga horária \textbf{específica do BCC} & 880h & 88h \\\hline
  \end{tabular}
\end{table}

Os componentes curriculares de extensão contemplados estão listados na
Tabela~\ref{tab:componentes-ext}, com os respectivos indicativos de carga horária a
serem considerados para creditação.

\begin{table}[h!]
   \caption{Creditação de horas extensionistas para componentes curriculares.}
   \label{tab:componentes-ext}\centering
   \begin{tabular}{|p{0.1\textwidth}|p{0.3\textwidth}|p{0.3\textwidth}|}
      \hline
      Art 5°
      &
        Ações  com caráter  extensionista  registradas no  Módulo Extensão  do
        Sistema  de Gestão  Acadêmica em  que o  discente atue  como bolsista,
        voluntário ou  membro da  equipe de  execução em  ações de  extensão e
        cultura
      &
        Número de horas de atividades extensionistas descritas no plano de
        trabalho do projeto e presentes no certificado
      \\
      \hline
      Art 11°
      &
        Metodologia didático-pedagógica extensionista em disciplinas livres ou
        de opção limitada
      &
        Número de horas de extensão no catálogo de disciplinas
      \\
      \hline
      Art. 12°
      &
        Disciplinas com oferecimento excepcional de componente extensionista
      &
        Número de horas de extensão definidas no plano de ensino da oferta
        específica
     \\
           \hline
      Art. 16°
      &
        Trabalho de Conclusão de Curso
      &
        Número de horas de atividades extensionistas convalidadas pela
        Coordenação de Curso, limitados a 24h
      \\

      \hline
      Art 17°
      &
        Estágios
      &
        Número de horas de atividades extensionistas convalidadas pela
        Coordenação de Curso, limitados a 24h
      \\
      \hline
      Art 20°
      &
        Eventos extensionistas periódicos permanentes do curso
      &
        Número  de  horas  de   atividades  extensionistas  convalidadas  pela
        Coordenação de Curso
      \\
      \hline
      Art 22°
      &
        Outras atividades discentes
      &
        Número de horas de atividades extensionistas convalidadas pela
        Coordenação de Curso, limitadas a 26h (30\% da carga extensionista
        total do curso)
      \\
      \hline
    \end{tabular}
  \end{table}

Deve-se destacar que, exceto no caso de disciplinas com carga horária
extensionista antecipadamente definida, o julgamento do aproveitamento
ou não da atividade, bem como a quantidade de horas aproveitadas, com
o caráter extensionista deverá ser definido pela coordenação de curso
ou por grupo avaliador por ela indicado seguindo a regulamentação
vigente na ocasião do pedido de aproveitamento.

Deve-se ainda ressaltar que o cumprimento da carga horária em
atividades de extensão é um requisito para colação de grau do
discente. A carga horária mínima em atividades de extensão é de 328
horas, número equivalente a 10\% da carga horária mínima do curso.
