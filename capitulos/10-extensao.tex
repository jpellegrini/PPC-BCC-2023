\section{Atividades de Extensão}
\label{sec:extensao}

O BCC obedece à Resolução ConsEPE no. 253/2022, regulamenta a inclusão de carga horária e ações de extensão nos cursos de graduação da UFABC. Entende-se, assim, que a extensão universitária é um processo interdisciplinar, político educacional, cultural, científico ou tecnológico que promove a interação transformadora entre a UFABC e outros setores da sociedade, por meio da produção e aplicação do conhecimento, em articulação permanente com o ensino e/ou a pesquisa.

As ações de extensão na UFABC podem ocorrer como programas, projetos, cursos, eventos, prestação de serviço, divulgação científica, extensão tecnológica, ou outras reconhecidamente regulamentadas pelas instâncias competentes da universidade. 
No contexto particular do BCC, a carga horária em atividades de extensão deverá ser cumprida por meio dos seguintes recursos curriculares:
\begin{itemize}
	\item Conclusão com aproveitamento de disciplinas que possuam carga horária extensionista, já prevista na ementa e no catálogo de disciplinas;
	\item \red{Trabalho de conclusão de curso, com registro de carga horária extensionista reconhecido pela coordenação do curso com limite máximo de 24 horas}; 
	\item \red{Estágio curricular, com registro de carga horária extensionista reconhecido pela coordenação do curso com limite máximo de 24 horas};
	\item Apresentação ou exposição de trabalhos em palestras, congressos e	seminários técnico-científicos, desde que de cunho extensionista;
	\item Publicação como autor de artigo em revista de cunho extensionista resultante de ações de extensão;
	\item Publicação de trabalho completo em anais (ou similares) de eventos, palestras, congressos e seminários técnico-científicos, desde que de cunho extensionista;
	\item Realização de Componente Curricular Livre (CCL) que tenha caráter extensionista, conforme Resolução ConsEPE n° 242 de 2020.
	\item Outras atividades com propósito de extensão reconhecidas pelas instâncias competentes da UFABC;
\end{itemize}

Deve-se destacar que, exceto no caso de disciplinas com carga horária
extensionista antecipadamente definida, o julgamento do aproveitamento
ou não da atividade, bem como a quantidade de horas aproveitadas, com
o caráter extensionista deverá ser definido pela coordenação de curso
ou por grupo avaliador por ela indicado seguindo a regulamentação
vigente na ocasião do pedido de aproveitamento.

Deve-se ainda ressaltar que o cumprimento da carga horária em
atividades de extensão é um requisito para colação de grau do
discente. A carga horária mínima em atividades de extensão é de 344
horas, número equivalente a 10\% da carga horária mínima do curso.
