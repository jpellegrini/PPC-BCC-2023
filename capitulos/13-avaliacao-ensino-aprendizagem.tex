\section{Avaliação de processo ensino-aprendizagem}

A avaliação do processo de ensino e aprendizagem dos discentes na UFABC é feito
por meio de conceitos, o que permite uma análise mais qualitativa do
aproveitamento do aluno.
Segundo a Resolução ConsEPE nº 147, de 19 de março de 2013, os coeficientes de
desempenho utilizados na Instituição consistem em:
\begin{itemize}
    \item [A -] Desempenho excepcional, demonstrando excelente compreensão da
    disciplina e do uso do conteúdo.
    \item [B -] Bom desempenho, demonstrando boa capacidade de uso dos
    conceitos da disciplina.
    \item [C -] Desempenho mínimo satisfatório, demonstrando capacidade de uso
    adequado dos conceitos da disciplina, habilidade para enfrentar problemas
    relativamente simples e prosseguir em estudos avançados.
    \item[D -] Aproveitamento mínimo não satisfatório dos conceitos da
    disciplina, com familiaridade parcial do assunto e alguma capacidade para
    resolver problemas simples, mas demonstrando deficiências que exigem
    trabalho adicional para prosseguir em estudos avançados. Nesse caso, o
    aluno é aprovado na expectativa de que obtenha um conceito melhor em outra
    disciplina, para compensar o conceito D no cálculo do CR. Havendo vaga, o
    aluno poderá cursar esta disciplina novamente.
    \item [F -] Reprovado. A disciplina deve ser cursada novamente para
    obtenção de crédito.
    \item [O -] Reprovado por falta. A disciplina deve ser cursada novamente
    para obtenção de crédito.
\end{itemize}

Os conceitos a serem atribuídos aos estudantes, em uma dada disciplina, não
deverão estar rigidamente relacionados a qualquer nota numérica de provas,
trabalhos ou exercícios.
Os resultados também considerarão a capacidade do aluno de utilizar os
conceitos e material das disciplinas, criatividade, originalidade, clareza de
apresentação e participação em sala de aula e/ou laboratórios. 
O aluno, ao iniciar uma disciplina, será informado sobre as normas e critérios
de avaliação que serão considerados.


Não há um limite mínimo de avaliações a serem realizadas, mas, dado o caráter
qualitativo do sistema, é indicado que sejam realizadas ao menos duas em cada
disciplina durante o período letivo.
E serão apoiadas e incentivadas as iniciativas de se gerar novos documentos de
avaliação, como atividades extraclasse, tarefas em grupo, listas de exercícios,
atividades em sala e/ou em laboratório, observações do professor,
auto-avaliação, seminários, exposições, projetos, sempre no intuito de se
viabilizar um processo de avaliação que não seja apenas qualitativo, mas que se
aproxime de uma avaliação contínua.

Assim, propõe-se não apenas a avaliação de conteúdos, mas de estratégias
cognitivas e habilidades e competências desenvolvidas. 
Esse mínimo de duas sugere a possibilidade de ser feita uma avaliação
diagnóstica logo no início do período, que identifique a capacidade do aluno em
lidar com conceitos que apoiarão o desenvolvimento de novos conhecimentos e o
quanto ele conhece dos conteúdos a serem discutidos na disciplina, e outra no
final do período, que possa identificar a evolução do aluno relativamente ao
estágio de diagnóstico inicial. 
De posse do diagnóstico inicial, o próprio professor poderá ser mais eficiente
na mediação com os alunos no desenvolvimento da disciplina. 
Por fim, deverá ser levado em alta consideração o processo evolutivo descrito
pelas sucessivas avaliações no desempenho do aluno para que se faça a
atribuição de um conceito a ele.

Cabe ressaltar que os critérios de recuperação do curso da UFABC são atualmente
regulamentados pela Resolução ConsEPE nº 182 (ou outra resolução que venha a
substituí-la).



