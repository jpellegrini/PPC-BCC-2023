\section{Regras de transição}
\label{sec:regras_transicao}

A matriz curricular de 2023 entrará em vigor assim que aprovada por todos os
órgãos deliberativos de acordo com a Resolução ConsEPE nº 140 e será plenamente
oferecida para os ingressantes a partir do ano de 2023 na Universidade Federal
do ABC.

Os alunos que ingressaram na UFABC anteriormente a 2023 poderão integralizar o
curso de acordo com a matriz curricular de 2023 ou de acordo com a matriz
curricular do projeto pedagógico vigente no seu ano de ingresso.
Caberá ao aluno realizar a análise da sua situação com relação ao coeficiente
de progressão e decidir por qual matriz pretende obter o grau de bacharel em
Ciência da Computação.
Será necessário cumprir em sua totalidade uma das matrizes para ter direito à
integralização do curso.

Os ingressantes anteriores a 2023 que optem pela integralização de acordo com
o PPC 2023 devem simplesmente cursar as disciplinas obrigatórias constantes na
matriz de 2023.
Todas as disciplinas já cursadas na matriz anterior serão convalidadas
automaticamente, com exceção das seguintes:
\begin{itemize}
    \item A disciplina ``NHI2049-13 -- Lógica Básica'' será convalidada como de
    opção livre;
    \item A disciplina ``MCTA027-17 -- Teoria dos Grafos'' será convalidada
    como ``MCTA027-23 -- Algoritmos em Grafos'';
    \item As disciplinas de ``Estágio Supervisionado em Computação'' serão
    convalidadas como de opção limitada. Essas disciplinas deixaram de ser
    ofertadas em 2019 e estão extintas desde a matriz de 2017.
\end{itemize}

Os ingressantes anteriores a 2023 que optem pela integralização de acordo com o
seu PPC de ingresso também devem simplesmente continuar cursando as disciplinas
que serão ofertadas a partir de 2023, uma vez que todas as disciplinas
obrigatórias dos PPCs anteriores permanecem obrigatórias no PPC de 2023, com as
seguintes observações:
\begin{itemize}
    \item A disciplina ``MCTA016-23 -- Programação Funcional'' será convalidada
    como ``MCTA016-13 -- Paradigmas de Programação'';
    \item A disciplina ``MCTA027-23 -- Algoritmos em Grafos'' será convalidada
    como ``MCTA027-17 -- Teoria dos Grafos'';
    \item A disciplina ``MCTAXXX-23 -- Matemática Discreta II'' será
    convalidada como ``NHI2049-13 -- Lógica Básica''.  Alternativamente, a
    disciplina ``NHI2049-13 -- Lógica Básica'' pode ser cursada de acordo com
    oferta pelo curso de Filosofia;
    \item A disciplina ``MCTAXXX-23 -- Metodologia e Escrita Científica para
    Computação'' será convalidada como opção livre/limitada.
\end{itemize}
Com relação às disciplinas de opção limitada, a convalidação é direta, com
observação para duas disciplinas que tiveram o nome modificado:
\begin{itemize}
    \item A disciplina ``MCZA036-17 -- Análise de Algoritmos II'' passou a se
    chamar ``MCZA036-23 -- Análise de Algoritmos III'';
    \item A disciplina ``MCZA014-17 -- Métodos de Otimização'' passou a se
    chamar ``MCZA014-23 -- Otimização Não-linear''.
\end{itemize}

Detalhes das convalidações podem ser encontrados nas
Tabelas~\ref{tab:convalidacoes_obrigatorias}
e~\ref{tab:convalidacoes_limitadas}.

Os casos omissos serão resolvidos pela coordenação do curso, representada pelo
seu coordenador, com o apoio da Pró-Reitoria de Graduação.

\begin{table}[h!]
\caption{Tabela de convalidações das disciplinas obrigatórias.}
\label{tab:convalidacoes_obrigatorias}

\centering
{\footnotesize
\begin{tabular}{|c|p{.25\textwidth}|c||c|p{.25\textwidth}|c|}
    \hline
    \multicolumn{3}{|c||}{\bf 2017} & \multicolumn{3}{|c|}{\bf 2023} \\ 
    \hline
    
    \textbf{Sigla} & \textbf{Disciplina} & \textbf{Créditos} & \textbf{Sigla} & \textbf{Disciplina} & \textbf{Créditos} \\
    \hline\hline
    
    MCTB001-17 & Álgebra Linear & 6 & MCTB001-23 & Álgebra Linear & 6 \\ \hline
    MCTA001-17 & Algoritmos e Estruturas de Dados I & 4 & MCTA001-23 & Algoritmos e Estruturas de Dados I & 4 \\ \hline
    MCTA002-17 & Algoritmos e Estruturas de Dados II & 4 & MCTA002-23 & Algoritmos e Estruturas de Dados II & 4 \\ \hline
    MCTA003-17 & Análise de Algoritmos & 4 & MCTA003-23 & Análise de Algoritmos I & 4 \\
    & & & MCTAXXX-23 & Análise de Algoritmos II & 4 \\ \hline
    MCTA004-17 & Arquitetura de Computadores & 4 & MCTA004-23 & Arquitetura de Computadores & 4 \\ \hline
    MCTA037-17 & Banco de Dados & 4 & MCTA037-23 & Modelagem de Banco de Dados & 4 \\ \hline
    MCTA006-17 & Circuitos Digitais & 4 & MCTA006-23 & Circuitos Digitais & 4 \\ \hline
    MCTA007-17 & Compiladores & 4 & MCTA007-23 & Compiladores e Interpretadores & 4 \\ \hline
    MCTA008-17 & Computação Gráfica & 4 & MCTA008-23 & Computação Gráfica & 4 \\ \hline
    MCTA009-13 & Computadores, Ética e Sociedade & 2 & MCTA009-23 & Computadores, Ética e Sociedade & 2 \\ \hline
    MCTA033-15 & Engenharia de Software & 4 & MCTA033-23 & Engenharia de Software & 4 \\ \hline
    MCTA014-15 & Inteligência Artificial & 4 & MCTA014-23 & Inteligência Artificial & 4 \\ \hline
    MCTA015-13 & Linguagens Formais e Automata & 4 & MCTA015-23 & Linguagens Formais e Autômatos & 4 \\ \hline
    NHI2049-13 & Lógica Básica & 4 & NHI2049-XX & Lógica Básica & 4 \\
    & & & MCTAXXX-23 & ou Matemática Discreta II & 4 \\ \hline
    MCTB019-17 & Matemática Discreta & 4 & MCTB019-23 & Matemática Discreta & 4 \\ \hline
    MCTA016-13 & Paradigmas de Programação & 4 & MCTA016-23 & Programação Funcional & 4 \\ \hline
    MCTA028-15 & Programação Estruturada & 4 & MCTA028-23 & Programação Estruturada & 4 \\ \hline
    MCTA017-17 & Programação Matemática & 4 & MCTA017-23 & Otimização Linear & 4 \\ \hline
    MCTA018-13 & Programação Orientada a Objetos & 4 & MCTA018-23 & Programação Orientada a Objetos & 4 \\ \hline
    MCTA029-17 & Projeto de Graduação em Computação I & 8 & MCTA029-23 & Trabalho de Conclusão de Curso I & 4 \\ \hline
    MCTA030-17 & Projeto de Graduação em Computação II & 8 & MCTA030-23 & Trabalho de Conclusão de Curso II & 4 \\ \hline
    MCTA031-17 & Projeto de Graduação em Computação III & 8 & MCTA031-23 & Trabalho de Conclusão de Curso III & 4 \\ \hline
    MCTA022-17 & Redes de Computadores & 4 & MCTA022-23 & Redes de Computadores & 4 \\ \hline
    MCTA023-17 & Segurança de Dados & 4 & MCTA023-23 & Segurança de Dados & 4 \\ \hline
    MCTA024-13 & Sistemas Digitais & 4 & MCTA024-23 & Sistemas Digitais & 4 \\ \hline
    MCTA025-13 & Sistemas Distribuídos & 4 & MCTA025-23 & Sistemas Distribuídos & 4 \\ \hline
    MCTA026-13 & Sistemas Operacionais & 4 & MCTA026-23 & Sistemas Operacionais & 4 \\ \hline
    MCTA027-17 & Teoria dos Grafos & 4 & MCTA027-23 & Algoritmos em Grafos & 4 \\ \hline
     & Não há & & MCTAXXX-23 & Metodologia e escrita científica para computação & 2 \\ \hline
\end{tabular}
}
\end{table}

{\footnotesize
\begin{longtable}{|c|p{.25\textwidth}|c||c|p{.25\textwidth}|c|}
\caption{Tabela de convalidações das disciplinas de opção limitada.}
\label{tab:convalidacoes_limitadas} \\

\hline
\multicolumn{3}{|c||}{\bf 2017} & \multicolumn{3}{|c|}{\bf 2023} \\ 
\hline

\textbf{Sigla} & \textbf{Disciplina} & \textbf{Créditos} & \textbf{Sigla} & \textbf{Disciplina} & \textbf{Créditos} \\
\hline\hline

MCZA035-17 & Algoritmos Probabilísticos & 4 & MCZA035-17 & Algoritmos Probabilísticos & 4\\ \hline
MCZA036-17 & Análise de Algoritmos II & 4 & MCZA036-23 & Análise de Algoritmos III & 4\\ \hline
MCZA001-13 & Análise de Projetos & 2 & MCZA001-13 & Análise de Projetos & 2\\ \hline
MCTB007-17 & Anéis e Corpos & 4 & MCTB007-17 & Anéis e Corpos & 4\\ \hline
MCZA002-17 & Aprendizado de Máquina & 4 & MCZA002-17 & Aprendizado de Máquina & 4\\ \hline
MCZA003-17 & Arquitetura de Computadores de Alto Desempenho & 4 & MCZA003-17 & Arquitetura de Computadores de Alto Desempenho & 4\\ \hline
MCZA004-13 & Avaliação de Desempenho de Redes & 4 & MCZA004-13 & Avaliação de Desempenho de Redes & 4\\ \hline
MCZA005-17 & Banco de Dados de Apoio à Tomada de Decisão & 4 & MCZA005-17 & Banco de Dados de Apoio à Tomada de Decisão & 4\\ \hline
MCTB009-17 & Cálculo Numérico & 4 & MCTB009-17 & Cálculo Numérico & 4\\ \hline
MCZA037-17 & Combinatória Extremal & 4 & MCZA037-17 & Combinatória Extremal & 4\\ \hline
MCZA006-17 & Computação Evolutiva e Conexionista & 4 & MCZA006-17 & Computação Evolutiva e Conexionista & 4\\ \hline
ESZG013-17 & Empreendedorismo & 4 & ESZG013-17 & Empreendedorismo & 4\\ \hline
MCZA007-13 & Empreendedorismo e Desenvolvimento de Negócios & 4 & MCZA007-13 & Empreendedorismo e Desenvolvimento de Negócios & 4\\ \hline
MCZA051-17 & Estágio Supervisionado em Computação & 4 & MCZA051-17 & Estágio Supervisionado em Computação & 4\\ \hline
ESZI030-17 & Gerenciamento e Interoperabilidade de Redes & 4 & ESZI030-17 & Gerenciamento e Interoperabilidade de Redes & 4\\ \hline
MCZA016-17 & Gestão de projetos de software & 4 & MCZA016-17 & Gestão de projetos de software & 4\\ \hline
ESZG019-17 & Gestão Estratégica e Organizacional & 2 & ESZG019-17 & Gestão Estratégica e Organizacional & 2\\ \hline
MCTB018-17 & Grupos & 4 & MCTB018-17 & Grupos & 4\\ \hline
MCZB012-13 & Inferência Estatística & 4 & MCZB012-13 & Inferência Estatística & 4\\ \hline
ESZI013-17 & Informática Industrial & 4 & ESZI013-17 & Informática Industrial & 4\\ \hline
MCZA008-17 & Interação Humano-Computador & 4 & MCZA008-17 & Interação Humano-Computador & 4\\ \hline
ESZB022-17 & Introdução à Bioinformática & 4 & ESZB022-17 & Introdução à Bioinformática & 4\\ \hline
MCZB015-13 & Introdução à Criptografia & 4 & MCZB015-13 & Introdução à Criptografia & 4\\ \hline
MCZB018-13 & Introdução à Modelagem e Processos Estocásticos & 4 & MCZB018-13 & Introdução à Modelagem e Processos Estocásticos & 4\\ \hline
MCTC021-15 & Introdução à Neurociência Computacional & 4 & MCTC021-15 & Introdução à Neurociência Computacional & 4\\ \hline
MCZA032-17 & Introdução à Programação de Jogos & 4 & MCZA032-17 & Introdução à Programação de Jogos & 4\\ \hline
ESZI034-17 & Jogos Digitais: Aspectos Técnicos e Aplicações & 4 & ESZI034-17 & Jogos Digitais: Aspectos Técnicos e Aplicações & 4\\ \hline
MCZA010-13 & Laboratório de Engenharia de Software & 4 & MCZA010-13 & Laboratório de Engenharia de Software & 4\\ \hline
MCZA011-17 & Laboratório de Redes & 4 & MCZA011-17 & Laboratório de Redes & 4\\ \hline
MCZA012-13 & Laboratório de Sistemas Operacionais & 4 & MCZA012-13 & Laboratório de Sistemas Operacionais & 4\\ \hline
MCZA013-13 & Lógicas não Clássicas & 4 & MCZA013-13 & Lógicas não Clássicas & 4\\ \hline
MCZA014-17 & Métodos de Otimização & 4 & MCZA014-23 & Otimização não-linear & 4\\ \hline
MCZA015-13 & Mineração de Dados & 4 & MCZA015-13 & Mineração de Dados & 4\\ \hline
MCZA052-22 & Vizualização de Dados e Informações & 4 & MCZA052-22 & Vizualização de Dados e Informações & 4\\ \hline
ESTG013-17 & Pesquisa Operacional & 4 & ESTG013-17 & Pesquisa Operacional & 4\\ \hline
ESZI022-17 & Planejamento de Redes de Informação & 4 & ESZI022-17 & Planejamento de Redes de Informação & 4\\ \hline
MCZA038-17 & Prática Avançada de Programação A & 4 & MCZA038-17 & Prática Avançada de Programação A & 4\\ \hline
MCZA038-17 & Prática Avançada de Programação B & 4 & MCZA038-17 & Prática Avançada de Programação B & 4\\ \hline
MCZA038-17 & Prática Avançada de Programação C & 4 & MCZA038-17 & Prática Avançada de Programação C & 4\\ \hline
MCZA041-17 & Processamento de Imagens Utilizando GPU & 4 & MCZA041-17 & Processamento de Imagens Utilizando GPU & 4\\ \hline
MCZA017-13 & Processamento de Linguagem Natural & 4 & MCZA017-13 & Processamento de Linguagem Natural & 4\\ \hline
MCTC022-15 & Processamento de Sinais Neurais & 4 & MCTC022-15 & Processamento de Sinais Neurais & 4\\ \hline
MCZA018-17 & Processamento Digital de Imagens & 4 & MCZA018-17 & Processamento Digital de Imagens & 4\\ \hline
MCZA042-17 & Processo e Desenvolvimento de Softwares Educacionais & 4 & MCZA042-17 & Processo e Desenvolvimento de Softwares Educacionais & 4\\ \hline
MCZA033-17 & Programação Avançada para Dispositivos Móveis & 4 & MCZA033-17 & Programação Avançada para Dispositivos Móveis & 4\\ \hline
ESZI033-17 & Programação de Dispositivos Móveis & 2 & ESZI033-17 & Programação de Dispositivos Móveis & 2\\ \hline
MCZA019-17 & Programação para Web & 4 & MCZA019-17 & Programação para Web & 4\\ \hline
MCZA020-13 & Programação Paralela & 4 & MCZA020-13 & Programação Paralela & 4\\ \hline
MCZA034-17 & Programação Segura & 4 & MCZA034-17 & Programação Segura & 4\\ \hline
MCZA021-17 & Projeto de Redes & 4 & MCZA021-17 & Projeto de Redes & 4\\ \hline
MCZA022-17 & Projeto Interdisciplinar & 4 & MCZA022-17 & Projeto Interdisciplinar & 4\\ \hline
MCZA023-17 & Redes Convergentes & 4 & MCZA023-17 & Redes Convergentes & 4\\ \hline
ESZI029-17 & Redes WAN de Banda Larga & 4 & ESZI029-17 & Redes WAN de Banda Larga & 4\\ \hline
MCZA024-17 & Redes sem Fio & 4 & MCZA024-17 & Redes sem Fio & 4\\ \hline
MCZA044-17 & Robótica e Sistemas Inteligentes & 4 & MCZA044-17 & Robótica e Sistemas Inteligentes & 4\\ \hline
MCZA045-17 & Robótica Educacional & 4 & MCZA045-17 & Robótica Educacional & 4\\ \hline
MCZA025-13 & Segurança em Redes & 4 & MCZA025-13 & Segurança em Redes & 4\\ \hline
MCZA046-17 & Semântica de Linguagem de Programação & 4 & MCZA046-17 & Semântica de Linguagem de Programação & 4\\ \hline
MCZA026-17 & Sistemas de Gerenciamento de Banco de Dados & 4 & MCZA026-17 & Sistemas de Gerenciamento de Banco de Dados & 4\\ \hline
MCZA027-17 & Sistemas de Informação & 4 & MCZA027-17 & Sistemas de Informação & 4\\ \hline
ESZI014-17 & Sistemas Inteligentes & 4 & ESZI014-17 & Sistemas Inteligentes & 4\\ \hline
MCZA028-13 & Sistemas Multiagentes & 4 & MCZA028-13 & Sistemas Multiagentes & 4\\ \hline
MCZA029-13 & Sistemas Multimídia & 4 & MCZA029-13 & Sistemas Multimídia & 4\\ \hline
MCZA047-17 & Sistemas Multi-Robôs Sociais & 4 & MCZA047-17 & Sistemas Multi-Robôs Sociais & 4\\ \hline
MCZA050-15 & Técnicas Avançadas de Programação & 4 & MCZA050-15 & Técnicas Avançadas de Programação & 4\\ \hline
NHZ5019-15 & Tecnologias da Informação e Comunicação na Educação & 4 & NHZ5019-15 & Tecnologias da Informação e Comunicação na Educação & 4\\ \hline
MCZB033-17 & Teoria da Recursão e Computabilidade & 4 & MCZB033-17 & Teoria da Recursão e Computabilidade & 4\\ \hline
MCZA048-17 & Teoria Espectral de Grafos & 4 & MCZA048-17 & Teoria Espectral de Grafos & 4\\ \hline
MCZA049-17 & Tópicos Emergentes em Banco de Dados & 4 & MCZA049-17 & Tópicos Emergentes em Banco de Dados & 4\\ \hline
MCZA030-17 & Vida Artificial na Computação & 4 & MCZA030-17 & Vida Artificial na Computação & 4\\ \hline
ESZA019-17 & Visão Computacional & 4 & ESZA019-17 & Visão Computacional & 4\\ \hline
MCZA031-13 & Web Semântica & 4 & MCZA031-13 & Web Semântica & 4\\ \hline

\end{longtable}
}

