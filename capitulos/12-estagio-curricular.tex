\section{Estágio Supervisionado}
\label{sec:estagio}

Segundo o Parecer CNE/CES No 136/2012, \textit{"os cursos de bacharelado na área de Computação são orientados para que seus egressos assumam funções no mercado de trabalho, incluindo a área acadêmica. Algumas das funções dos egressos dos cursos de bacharelados e de licenciatura da área de Computação são predominantemente orientadas para realizar atividades de processos e outras para transformar processos, com o desenvolvimento de novas tecnologias."}

O estágio supervisionado curricular \textit{não é obrigatório}. O BCC, por se identificar como um bacharelado em Ciências, não exige o cumprimento de estágio para integralização do curso. Apesar disso, o BCC incentiva e apoia os alunos interessados nessa atividade como instrumento complementar de formação e mecanismo facilitador de profissionalização. O estágio curricular é formalizado por meio de disciplinas específicas de opção limitada, que podem ser aproveitadas por estudantes regularmente matriculados no curso e que desenvolvam atividades de estágio que reconhecidamente contribuam para sua profissionalização na área de Computação.

Os critérios de aproveitamento, documentação, renovação (quando cabível) e orientação de estágio supervisionado são definidos pela Resolução CG no. 17 de 09 de outubro de 2017 ou documento mais recente que a substitua.


