\section{Atividades complementares}
\label{sec:atividades_complementares}

As atividades complementares são todas as atividades acadêmicas de natureza diversa, que
não se incluem no desenvolvimento regular das disciplinas constantes na matriz
curricular do BCC, mas que são relevantes para a formação do aluno.

De acordo com as DCNs da área de Computação, as Atividades Complementares são componentes curriculares enriquecedores e implementadores do próprio perfil do formando e deverão possibilitar o desenvolvimento de habilidades, conhecimentos, competências e atitudes do estudante, inclusive as adquiridas fora do ambiente acadêmico, que serão reconhecidas mediante processo de avaliação 

O BCC e a UFABC fornecem ao estudante várias oportunidades de enriquecer sua formação com a
participação em atividades de natureza diversificada e interdisciplinar, fortalecendo assim a formação em diferentes áreas e estabelecendo diferentes oportunidades para consolidar o vínculo entre teoria e prática.

O aproveitamento de carga horária dessas atividades seguirá a Resolução CG nº 30, de 19 de outubro de 2022 (ou resolução mais recente que a substitua), publicado pelo Boletim de Serviços nº 1188, de 21 de outubro de 2022.
