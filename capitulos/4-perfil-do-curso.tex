\section{Perfil do Curso}

O curso de Bacharelado em Ciência da Computação (BCC) da Universidade Federal
do ABC propõe formar profissionais com carácter interdisciplinar e
multidisciplinar, com formação teórica consistente e vivência prática que
permita contribuir para o desenvolvimento científico e tecnológico da Ciência
da Computação, atuando profissionalmente em empresas de tecnologia, em pesquisa
científica ou em ações empreendedoras.

Além de uma formação básica sólida e uma proposta de desenvolvimento ético e
científico, o curso promove fortemente uma construção interdisciplinar, em
consonância com projeto pedagógico institucional da UFABC.
Os egressos do curso podem atuar em nível regional, nacional e internacional,
atendendo à crescente demanda por profissionais qualificados nas diversas
áreas em que a Ciência da Computação pode atuar.

A Computação está presente na rotina da população em praticamente todas as suas
atividades sociais, econômicas e científicas.
Podemos facilmente identificar a influência de algoritmos e recursos
computacionais em diversas atividades comuns, tais como ler notícias,
comunicar-se com outras pessoas, viajar, trabalhar, estudar, etc.
Dispositivos computacionais estão presentes em eletrodomésticos, veículos,
telefones celulares, televisores e computadores, entre outros.
A Ciência da Computação é certamente uma das áreas de futuro mais promissor,
abrindo várias oportunidades de desenvolvimento tecnológico e alimentando
iniciativas empreendedoras que buscam soluções para problemas gerais e
específicos da sociedade.
A demanda por profissionais é reconhecidamente alta e com tendência de
expansão, necessitando de cursos de formação que contribuam para atender de
forma qualificada a essa perspectiva de crescimento.

A estrutura curricular do BCC se baseia em vários documentos de referência:
\begin{itemize}
    \item Diretrizes curriculares nacionais dos cursos da área de Computação;
    \item Proposta curricular das associações:
    \begin{itemize}
        \item ACM (Association for Computing Machinery);
        \item IEEE-CS (IEEE Computer Society);
        \item SBC (Sociedade Brasileira de Computação).
    \end{itemize}
\end{itemize}

O BC\&T contribui com a formação básica e divide-se em seis eixos
didáticopedagógicos estruturantes:
\begin{itemize}
    \item Estrutura da Matéria;
    \item Energia;
    \item Processos de Transformação;
    \item Representação e Simulação;
    \item Informação e Comunicação;
    \item Humanidades.
\end{itemize}


\subsection{Justificativa de oferta do curso}

A UFABC está localizada na região conhecida como ABC Paulista, apelido que faz
referência às cidades de Santo \textbf{A}ndré, São \textbf{B}ernardo do Campo e
São \textbf{C}aetano do Sul e parte da região metropolitana de São Paulo
(RMSP).
A RMSP é altamente urbanizada (98\%) formada por 39 municípios e uma população
próxima de 22 milhões de habitantes (2021), que a faz figurar entre as dez mais
populosas do mundo. 

Do ponto de vista econômico, a RMSP é considerada o maior pólo de riqueza do
Brasil, com PIB per capita no valor de R\$ 56.649,03 (2018).
A atividade econômica está fortemente ligada à prestação de serviços (85,5\%),
embora o setor industrial também tenha relevância (14,3\%), sendo grande a
contribuição do ABC Paulista. 
Do ponto de vista educacional, é uma região em que mais da metade (57,5\%) da
população jovem entre 18 e 24 anos possui, no mínimo, o Ensino Médio completo
(censo 2010).
É também uma região com grande número de escolas e faculdades, públicas e
privadas. 

A Computação é uma das áreas mais promissoras em termos crescimento e
desenvolvimento.
Praticamente todos os setores utilizam recursos computacionais para automatizar
tarefas, desenvolver produtos, otimizar a utilização e monitoramento de
recursos, inovar, planejar políticas de expansão, controlar atividades, etc.
Durante a pandemia de COVID-19, foi uma das poucas áreas que apresentou
crescimento e permitiu que muitas atividades econômicas e sociais pudessem ser
preservadas, apesar das dificuldades e restrições sanitárias.
Segundo levantamento da Associação Brasileira das Empresas de Tecnologia da
Informação e Comunicação (Brasscom) realizado em 2021, a demanda não atendida
por profissionais no Brasil deve atingir 420 mil vagas até 2024, sendo que
formam-se aproximadamente 46 mil por ano.

Nesse contexto, o ABC Paulista pode ser visto como uma região estratégica para
o apoio ao desenvolvimento tecnológico local e nacional.
O ABC é uma região com forte participação industrial, conurbada em uma área com
forte demanda por serviços.
Além disso, é uma região com alto índice educacional, integrada à RMSP e ao
Brasil por meio de grandes rodovias, grandes aeroportos, ferrovias e o porto de
Santos, o maior da América Latina.
É uma região estratégica para implantação de empresas nacionais e
internacionais, de diversos setores sociais e econômicos.

A Computação é uma das áreas de conhecimento mais presente e influente na vida
de empresas e pessoas.
Encontramos técnicas, teorias, produtos e metodologias associadas à Ciência da
Computação em diversas iniciativas empresariais, políticas, sociais e
tecnológicas.
A busca por profissionais qualificados é uma necessidade de diversas entidades
que buscam inovação, otimização de recursos, pesquisa e desenvolvimento.

Outra característica da Ciência da Computação é sua aplicabilidade, capaz de
contribuir com diversas áreas de conhecimento, o que lhe garante alta
capacidade interdisciplinar e integradora.
A implantação do BCC, sob essa visão, é uma naturalmente identificada aos
princípios norteadores da UFABC e às necessidades das comunidades local,
regional e nacional.


Referências:
\begin{itemize}
    \item \url{https://perfil.seade.gov.br/}
\end{itemize}
