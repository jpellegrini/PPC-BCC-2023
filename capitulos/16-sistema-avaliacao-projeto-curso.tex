\section {Sistema de Avaliação do Projeto de Curso}

Buscando conhecer, avaliar e aprimorar a qualidade e os compromissos de sua missão, a
UFABC tem implementado mecanismos de avaliação permanente para a efetividade do processo de ensino-aprendizagem, visando compatibilizar a oferta de vagas, os objetivos do curso, o perfil do egresso e a demanda de profissionais no mercado de trabalho para o curso.

Um dos mecanismos adotado pela Coordenação do Curso para avaliação do Projeto
Político Pedagógico do Bacharelado em Ciência da Computação é a análise e o
estabelecimento de ações, a partir dos resultados obtidos pelo Curso e pela Universidade no
Sistema Nacional de Avaliação da Educação Superior (SINAES), regulamentado e instituído
pela Lei n° 10.681, de 14 de abril de 2004.

No Decreto n° 5.773, de 9 de maio de 2006, que dispõe sobre o exercício das funções de
regulação, supervisão e avaliação de Instituições de Educação Superior (IES) e Cursos
superiores de Graduação e Sequenciais no sistema federal de ensino, no seu artigo 1°,
parágrafo 3°, lê-se que a avaliação realizada pelo SINAES constitui referencial básico para os
processos de regulação e supervisão da educação superior, a fim de promover sua qualidade.

No que tange propriamente à estruturação da avaliação estabelecida pelo SINAES, são
considerados três tipos de avaliação:

\begin{enumerate}
	\item Avaliação institucional, que contempla um processo de autoavaliação realizado pela Comissão Própria de Avaliação (CPA) da Instituição de Educação Superior, está já implantada na UFABC, com as seguintes portarias de criação nos últimos anos:
	\begin{enumerate}
		\item Portaria 108, de 28 de fevereiro de 2013, que institui a Comissão Própria de Avaliação e demais portarias correlatas. Disponíveis em
		\url{https://www.ufabc.edu.br/administracao/comissoes/cpa/criacao}. Acesso em: 24 jun. 2022.
		\item Regimento interno da CPA - UFABC. Disponível em \url{https://www.ufabc.edu.br/administracao/comissoes/cpa/regimento-interno}. Acesso em 24 jun. 2022.
	\end{enumerate}
	
	\item Avaliação de curso, que considera um conjunto de avaliações: avaliação dos pares (in loco), avaliação dos estudantes (questionário de Avaliação Discente da Educação Superior – ADES, enviado à amostra selecionada para realização do Exame Nacional de Desempenho de Estudantes - ENADE), avaliação da Coordenação (questionário específico) e dos Professores do Curso e da CPA. Temos os seguintes relatórios produzidos nos últimos anos:
	\begin{enumerate}
		\item Relatório CPA 2022. Disponível em:
		\url{https://www.ufabc.edu.br/images/comissoes/cpa/relatorio_cpa_2022_vfinal_16_04_2022_.pdf}. Acesso em 24 jun. 2022.
		\item Relatório final CPA 2021. Disponível em:
		\url{https://www.ufabc.edu.br/images/comissoes/cpa/relatorio_cpa_2021_final_31_03_2021_entregue.pdf}. Acesso em 24 jun. 2022
		\item Relatório parcial CPA 2021. Disponível em:
		\url{https://www.ufabc.edu.br/images/comissoes/cpa/relatorio_cpa_2020.pdf}. Acesso em: 24 jun. 2022.
		\item Relatório do Grupo de Trabalho sobre Problemas e Oportunidades de Melhoria na Infraestrutura Pedagógica da UFABC. Disponível em \url {https://www.ufabc.edu.br/images/comissoes/cpa/relatorio_gt_infraestrutura_pedagogica.pdf}. Acesso em 24 jun. 2022.
		\item Demais relatórios da CPA - UFABC. Disponíveis em  \url{https://www.ufabc.edu.br/administracao/comissoes/cpa}
	\end{enumerate}
	
	\item Avaliação do Desempenho dos estudantes ingressantes e concluintes, que corresponde à aplicação do ENADE aos estudantes que preenchem os critérios estabelecidos pela legislação vigente (incluem neste exame a prova e os questionários dos alunos, do Coordenador de Curso e da percepção do alunado sobre a prova).
\end{enumerate}

Com o apoio do NDE, os relatórios são utilizados para avaliar a estrutura do curso sob diferentes perspectivas: do discente, do docente, do resultado de exames de acompanhamento externo. Com base nesses elementos, são identificados e discutidos temas levantados sobre pontos positivos e negativos da concepção e execução do curso, como por exemplo:
\begin{itemize}
	\item Adequação da oferta de turmas de disciplinas;
	\item Nível de aproveitamento em disciplinas;
	\item Panorama geral de orientação de alunos para estágios, PGCs, iniciação científica e outras modalidades;
	\item Criação de disciplinas novas com cobertura de assuntos recentes à Computação;
	\item Reformulação de disciplinas;
	\item Adequação de ementas;
	\item Criação de grupos de trabalho;
	\item Outros temas.
\end{itemize}

A aplicação e divulgação dos resultados de discussões realizadas pela coordenação de curso, colegiado de curso e NDE são expostas e deliberadas em reunião plenária, excetuando-se casos em que os temas fogem de seu escopo.
