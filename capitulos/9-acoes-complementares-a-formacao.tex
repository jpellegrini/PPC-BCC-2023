\section{Ações acadêmicas complementares à formação}
\label{sec:acoes_complementares}

A UFABC possui diversos programas e ações para promover a qualidade do ensino
de graduação, dentre os quais citamos:
\begin{description}
    \item[Projeto de Ensino-Aprendizagem Tutorial -- PEAT:] tem como objetivo
    promover a adaptação do aluno ao projeto acadêmico da UFABC, orientando-o
    para uma transição tranquila e organizada do Ensino Médio para o Superior.
    Mais informações em: \url{http://prograd.ufabc.edu.br/peat};

    \item[Programas de iniciação científica:] têm como objetivo introduzir os
    alunos de graduação na pesquisa científica, visando fundamentalmente
    colocar o aluno desde cedo em contato direto com a atividade científica e
    engajá-lo na pesquisa. A UFABC possui as seguintes modalidades:
    \begin{itemize}
        \item Programa Pesquisando Desde o Primeiro Dia -- PDPD;
        \item Programa de Iniciação Científica -- PIC/UFABC;
        \item Programa Institucional de Bolsas de Iniciação Científica -- PIBIC/CNPq;
        \item Programa Institucional de Bolsas de Iniciação Científica -- PIBIC/CNPq nas Ações Afirmativas.
    \end{itemize}
    Mais informações em:
    \url{https://propes.ufabc.edu.br/perfis-de-acesso/aluno};

    \item[Programa de monitoria acadêmica:] têm como objetivo selecionar alunos
    para desenvolverem atividades de monitoria.
    Mais informações em:
    \url{http://prograd.ufabc.edu.br/monitoria-academica};

    \item[Programa Institucional de Bolsas de Iniciação à Docência -- PIBID:] é
    um programa da Coordenação de Aperfeiçoamento de Pessoal de Nível Superior
    (CAPES) que tem por finalidade fomentar a iniciação à docência,
    contribuindo para o aperfeiçoamento da formação de docentes em nível
    superior e para a melhoria da qualidade da educação básica pública
    brasileira.
    Mais informações em: \url{https://pibid.ufabc.edu.br/};

    \item[Ações extensionistas:] esse tipo de atividade ultrapassa o âmbito
    específico de atuação do Instituto no que se refere ao Ensino (Graduação e
    Pós-Graduação) e Pesquisa. A Extensão é uma das funções sociais da
    Universidade, realizada por meio de um conjunto de ações dirigidas à
    sociedade, as quais devem estar indissociavelmente vinculadas ao Ensino e à
    Pesquisa.
    Mais informações em: \url{http://proec.ufabc.edu.br/};

    \item[Programa de Educação Tutorial -- PET:] tem como proposta desenvolver
    atividades que propiciem a ciência, tecnologia e inovação de dentro para
    fora da Universidade, conscientizando seus discentes da sua importância e
    de como fazer, assim como proporcionar ao corpo docente um ambiente
    favorável ao seu desenvolvimento e dar acesso a qualquer comunidade a esse
    recurso tanto acadêmica quanto externamente.
    Mais informações em:
    \url{http://prograd.ufabc.edu.br/pet};

    \item[Cursos de língua estrangeira:] oferecidos pelo Núcleo Educacional de
    Tecnologias e Línguas. Mais informações em:
    \url{https://netel.ufabc.edu.br/};

    \item[Mobilidade acadêmica:] consiste em um período de estudos, em regra de
    1 semestre, em uma universidade estrangeira ou nacional, com o objetivo de
    oferecer ao aluno experiências enriquecedoras capazes de agregar
    positivamente sua vida acadêmica, profissional e pessoal.
    Mais informações em: \url{https://ri.ufabc.edu.br/mobilidade-academica};

    \item[Monitoria inclusiva:] os monitores inclusivos são alunos de graduação
    que se dedicam 10 horas semanais em atividades de ações afirmativas ao
    aluno com deficiência, dando suporte como ledor, transcritor,
    audiodescritora de figuras, imagens, desenhos e vídeos. Outra atividade que
    também demanda atenção do monitor inclusivo é a adaptação de materiais e
    livros usados por alunos com deficiência visual.
    Mais informações em:
    \url{https://proap.ufabc.edu.br/acessibilidade-ufabc/servicos-e-recursos/monitoria-inclusiva};

    \item[Programa de Apoio ao Desenvolvimento Acadêmico -- PADA:] realiza
    atividades de orientação pedagógica a discentes de graduação nas áreas de:
    planejamento dos estudos junto a estudantes dos Bacharelados
    Interdisciplinares (BIs) e Licenciaturas Interdisciplinares (LIs);
    requisitos para integralização dos BIs e LIs; prazos para conclusão dos
    cursos interdisciplinares; prevenção ao desligamento dos cursos
    interdisciplinares.
    Mais informações em: \url{https://prograd.ufabc.edu.br/pada}.
\end{description}
