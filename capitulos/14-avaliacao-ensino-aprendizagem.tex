\section{Avaliação de processo ensino-aprendizagem}
\label{sec:avaliacao_ensino_aprendizagem}

A avaliação do processo de ensino e aprendizagem na UFABC é realizada por meio
de conceitos, o que permite uma análise qualitativa do aproveitamento do discente.
É recomendado que os conceitos a serem atribuídos aos estudantes, em
uma dada disciplina, não sejam estar rigidamente relacionados a
qualquer nota numérica de provas, trabalhos ou exercícios.  Os
resultados também considerarão a capacidade do aluno de utilizar os
conceitos e material das disciplinas, criatividade, originalidade,
clareza de apresentação e participação em sala de aula e/ou
laboratórios.  O aluno, ao iniciar uma disciplina, será informado
sobre as normas e critérios de avaliação que serão considerados.

Serão apoiadas e incentivadas as iniciativas de se gerar novos documentos de
avaliação, como atividades extraclasse, tarefas em grupo, listas de exercícios,
atividades em sala e/ou em laboratório, observações do professor,
auto-avaliação, seminários, exposições, projetos, sempre no intuito de se
viabilizar um processo de avaliação que não seja apenas qualitativo, mas que se
aproxime de uma avaliação contínua.
Assim, propõe-se não apenas a avaliação de conteúdos, mas de estratégias
cognitivas e habilidades e competências desenvolvidas. 

Por fim, deverá ser levada em alta consideração o processo evolutivo descrito
pelas sucessivas avaliações no desempenho do aluno para que se faça a
atribuição de um conceito a ele.

\subsection{Conceitos}

Segundo a Resolução ConsEPE nº 147, de 19 de março de 2013, os coeficientes de
desempenho utilizados na Instituição consistem em:
\begin{description}
    \item[A -] Desempenho excepcional, demonstrando excelente compreensão da
    disciplina e do uso do conteúdo.
    \item[B -] Bom desempenho, demonstrando boa capacidade de uso dos conceitos
    da disciplina.
    \item[C -] Desempenho mínimo satisfatório, demonstrando capacidade de uso
    adequado dos conceitos da disciplina, habilidade para enfrentar problemas
    relativamente simples e prosseguir em estudos avançados.
    \item[D -] Aproveitamento mínimo não satisfatório dos conceitos da
    disciplina, com familiaridade parcial do assunto e alguma capacidade para
    resolver problemas simples, mas demonstrando deficiências que exigem
    trabalho adicional para prosseguir em estudos avançados. Nesse caso, o
    aluno é aprovado na expectativa de que obtenha um conceito melhor em outra
    disciplina, para compensar o conceito D no cálculo do CR. Havendo vaga, o
    aluno poderá cursar esta disciplina novamente.
    \item[F -] Reprovado. A disciplina deve ser cursada novamente para obtenção
    de crédito.
    \item[O -] Reprovado por falta. A disciplina deve ser cursada novamente
    para obtenção de crédito.
\end{description}


Cabe ressaltar que os critérios de recuperação do curso da UFABC são atualmente
regulamentados pela Resolução ConsEPE nº 182 (ou outra resolução que venha a
substituí-la).

\subsection{Frequência}
Nas disciplinas presenciais, a frequência mínima obrigatória para aprovação é de
75\% das aulas ministradas e/ou atividades realizadas

\subsection{Mecanismos Susbtitutivos de Avaliação}
O discente que faltar à avaliação presencial poderá realizá-la sob a forma de
mecanismos de avaliação substitutivos, conforme critérios estabelecidos pelo docente
em seu Plano de Ensino. Além dos critérios estabelecidos pelo docente, fica
assegurado ao discente o direito a mecanismos de avaliação substitutivos nos casos
contemplados pelo Art. 2° da Resolução ConsEPE n° 227, de 23 de abril de 2018.

\subsection{Vista e Revisão de Instrumentos Avaliativos}
O estudante matriculado em disciplinas dos cursos de graduação terá direito a
vistas das correções de avaliações por ele realizadas durante o quadrimestre vigente. O
discente que discordar da correção realizada deverá pronunciar-se no momento da
vista, solicitando ao professor a revisão imediata, à luz dos objetivos e critérios
esclarecidos antes da avaliação. No prazo máximo de 7 (sete) dias letivos após o início
do quadrimestre subsequente, o discente poderá recorrer da revisão da correção do
instrumento avaliativo e/ou do conceito final conforme Resolução ConsEPE n° 120, de
4 de outubro de 2011.

\subsection{Recuperação}
De acordo com a Resolução ConsEPE n° 182, de 23 de outubro de 2014, além dos
critérios estabelecidos pelo docente em seu Plano de Ensino, fica garantido ao discente
que for aprovado com conceito D ou reprovado com conceito F numa disciplina, o
direito a fazer uso de mecanismos de recuperação.
A data e os critérios dos mecanismos de recuperação deverão ser definidos pelo
docente responsável pela disciplina e explicitados no Plano de Ensino, o qual deverá
ser disponibilizado aos discentes no início do quadrimestre letivo. Além disso, o
mecanismo de recuperação não poderá ser aplicado em período inferior a 72 horas
após a divulgação dos conceitos das avaliações regulares e poderá ser aplicado até a
terceira semana após o início do quadrimestre subsequente.
Por fim, a critério do docente e nos casos em que seja possível a sua aplicação, o
mecanismo de avaliação substitutivo poderá ser o mecanismo de recuperação, desde
que garantido o direito ao mecanismo de recuperação para o estudante que fez uso do
mecanismo de avaliação substitutivo.

\subsection{Coeficientes de Desempenho}

\newcommand{\cpk}{CP$_k$\xspace}

De acordo com a Resolução ConsEPE n° 147, de 19 de março de 2013, o
desempenho de discentes será avaliado por meio dos seguintes coeficientes:
Coeficiente de Rendimento (CR), Coeficiente de Aproveitamento (CA) e Coeficiente de
Progressão (\cpk).

\begin{itemize}
\item O \textbf{Coeficiente de Rendimento (CR)} é um número indicativo do desenvolvimento do
estudante no curso, cujo cálculo considera os conceitos obtidos em todas as disciplinas
por ele cursadas, incluindo repetições. O cálculo do CR leva em conta a média
ponderada dos conceitos obtidos em todas as disciplinas cursadas pelo estudante,
considerando seus respectivos créditos, conforme expressão abaixo,
\[
  CR = \dfrac{\sum_{i=1}^{NC}C_if(N_i)}{\sum_{i=1}^{NC}{C_i}},
\]
sendo $NC$ o número de disciplinas cursadas até o momento pelo discente, $C_i$ o número
de créditos da disciplina $i$, $N_i$ o conceito obtido pelo estudante na disciplina $i$, $f(A) = 4$,
$f(B) = 3$, $f(C) = 2$, $f(D) = 1$, e $f(F) = f(O) = 0$.

{\em Observação:} Todos os conceitos de todas as disciplinas cursadas entram no cálculo do
CR, independentemente do resultado obtido pelo discente. Somente as disciplinas com
cancelamento de matrícula deferido e as disciplinas em que o estudante obteve
dispensa por equivalência não entram no cálculo do CR.

\item O \textbf{Coeficiente de Aproveitamento (CA)} é um número indicativo da média dos
melhores conceitos obtidos em todas as disciplinas cursadas pelo estudante. O cálculo
do CA é similar ao do CR; entretanto, no caso de disciplina realizada mais de uma vez,
somente se contabiliza o melhor conceito obtido, conforme disposto na expressão
abaixo,
\[
  CA = \dfrac{\sum_{i=1}^{ND}C_if(M_i)}{\sum_{i=1}^{ND}C_i},
\]
sendo $ND$ o número de disciplinas diferentes cursadas até o momento pelo estudante,
$C_i$ o número de créditos da disciplina $i$, $M_i$ o melhor conceito obtido pelo discente na
disciplina $i$ considerando todas as vezes em que ele a tenha cursado, $f(A) = 4$, $f(B) = 3$,
$f(C) = 2$, $f(D) = 1$, e $f(F) = f(O) = 0$.

\item O \textbf{Coeficiente de Progressão (\cpk)} para um determinado curso k é um número que
informa a razão entre os créditos das disciplinas aprovadas e o número total de
créditos exigidos para integralização desse curso, seja esse um bacharelado ou
licenciatura interdisciplinar ou qualquer curso de formação específica. O valor do \cpk,
calculado conforme expressão abaixo, cresce à medida que o discente é aprovado nas
disciplinas cursadas, de acordo com suas categorias (obrigatória, opção limitada ou
livre) para o curso considerado. Quando o \cpk alcançar o valor 1, o estudante terá
concluído os créditos correspondentes às disciplinas do curso $k$ considerado.
\[
  CP_k = \frac{n^k_{obr} + \min\{
      (N^k_{lim} + N^k_{livre}),
      n^k_{lim} + \min\{n^k_{livre},N^k_{livre}\}
    \} }{NC_k}
\]
sendo
\begin{itemize}

\item $n^k_{obr}$ o número de créditos em disciplinas obrigatórias do curso cumpridos com
aprovação;
\item $n^k_{lim}$ o número de créditos em disciplinas de opção limitada do curso cumpridos
com aprovação;
\item $n^k_{livre}$ o número de créditos em disciplinas livres do curso cumpridos com
aprovação;
\item $N^k_{obr}$ o número de créditos exigidos em disciplinas obrigatórias do curso;
\item $N^k_{lim}$ o número de créditos exigidos em disciplinas de opção limitada do curso;
\item $N^k_{livre}$ o número de créditos propostos em disciplinas livres do curso; e
\item $NC_k = N^k_{obr} + N^k_{lim} + N^k_{livre}$.
\end{itemize}
\end{itemize}
