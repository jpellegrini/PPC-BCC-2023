\section{Requisito de acesso}

\subsection{Formas de acesso ao curso}
O processo seletivo para acesso aos Cursos de Graduação da Universidade Federal
do ABC é anual, e inicialmente realizado pelo Sistema de Seleção Unificado
(SISU), do MEC, onde as vagas oferecidas serão preenchidas em uma única fase,
baseado no resultado do Exame Nacional do Ensino Médio (ENEM).

O ingresso nos cursos de formação específica, após a conclusão dos bacharelados
interdisciplinares, se dá por seleção interna, segundo a Resolução ConsEPE, nº
31, de 4 de agosto de 2009.
Sendo assim, o ingresso ao Bacharelado em Ciência da Computação é realizado
após o ingresso no Bacharelado em Ciência e Tecnologia.

Existe ainda a possibilidade de transferência, facultativa ou obrigatória, de
alunos de outras Instituições de Ensino (IES) para o BCC.
No primeiro caso, mediante transferência de alunos de cursos afins, quando da
disponibilidade de vagas, através de processo seletivo interno (art. 49 da Lei
nº 9.394, de 1996 e Resolução ConsEPE nº 174 de 24 de abril de 2014); para o
segundo, por \textit{transferências ex officio} previstas em normas específicas
(art. 99 da Lei 8.112 de 1990, art. 49 da Lei 9.394 de 1996 regulamentada pela
Lei 9.536 de 1997 e Resolução ConsEPE nº 10 de 2008).

\subsection{Regime de matrícula}

O processo de matrículas em disciplinas é conduzido de forma unificada pela
Pró-Reitoria de Graduação (Prograd) da UFABC.
Antes do início de cada quadrimestre letivo, cada aluno(a) deve solicitar a sua
matrícula, indicando as disciplinas que deseja cursar no quadrimestre
correspondente.
O período de matrícula é determinado pelo calendário da UFABC definido
anualmente pela Comissão de Graduação.

A matrícula de alunos ingressantes é realizada de forma automática e
obrigatória, obedecendo à matriz curricular do bacharelado interdisciplinar de
ingresso.
A partir do quadrimestre letivo seguinte, o(a) aluno(a) entra no regime de
matrícula regular, obedecendo ao procedimento citado anteriormente.
Alunos ingressantes devem cursar, obrigatoriamente, o mínimo de nove créditos
no quadrimestre de ingresso. 

Por não apresentarem pré-requisitos, todas as disciplinas podem ser solicitadas
livremente e a qualquer momento no processo de matrícula.
Apesar disso, deve-se ressaltar que cada disciplina possui uma lista de
recomendações, que expõe disciplinas que desejavelmente deveriam ter sido
cursadas anteriormente.
Embora não exista o bloqueio formal do pré-requisito, é importante que cada
estudante considere a lista de recomendações como um elemento orientador que
busca auxiliar o cumprimento bem-sucedido da matriz curricular.

É essencial ressaltar que o número de vagas e turmas é limitado, e o
preenchimento de vagas na matrícula segue os critérios de seleção adotados pela
Prograd.
Em casos particulares (como disciplinas de Trabalho de Conclusão de Curso,
Estágios ou outras), os pedidos de matrícula são ainda analisados pela
coordenação do BCC, que poderá autorizá-los, ou não, dentro de seus critérios
de adequação e viabilidade pedagógica.

É importante ainda que o(a) estudante observe os critérios de permanência do
curso e jubilação (desligamento), regulados pela Resolução ConsEPE nº 166, de
08 de outubro de 2013.
