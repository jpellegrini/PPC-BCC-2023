\section{Atividades complementares}

As atividades complementares são todas as atividades de diversas naturezas, que
não se incluem no desenvolvimento regular das disciplinas constantes na matriz
curricular do BCC, mas que são relevantes para a formação do aluno.

O objetivo do incentivo à realização de atividades complementares consiste em
fornecer ao estudante a oportunidade de enriquecer sua formação com a
participação em atividades de natureza diversificada. 
Como consequência, tem-se a acentuação do caráter interdisciplinar e amplo da
formação do aluno, além do fortalecimento do vínculo entre teoria e prática.

Uma vez que o BC\&T é requisito para o BCC, e neste curso já está prevista a
realização de 120 horas de atividades complementares, o BCC não exigirá a
realização de atividades complementares específicas além das já previstas no
BC\&T. 

A forma de validação da carga horária dessas atividades encontra-se na
Resolução CG nº 11, de 28 de junho de 2016, publicado pelo Boletim de Serviços
nº 568, de 05 de julho de 2016.
