\section {Apresentação}

A Fundação Universidade Federal do ABC (UFABC) é uma fundação pública, criada pela Lei nº 11.145 de 26 de julho de 2005 sancionada pelo Presidente da República e publicada no Diário Oficial da União em 27 de julho de 2005 e alterada pela Lei nº 13.110, de 25 de março de 2015, publicada no D.O.U em 26 de março de 2015. É uma instituição de ensino superior, extensão e pesquisa, vinculada ao Ministério da Educação, com sede e foro no Município de Santo André, situada na Avenida dos Estados, 5001, bairro Santa Terezinha, Santo André, CEP 09210-580, no Estado de São Paulo e com limite territorial de atuação multicampi na região do ABC paulista, nos termos do Artigo 2\textsuperscript{\underline{o}} da lei mencionada.

A UFABC possui autonomia administrativa, didático-científica, gestão financeira e
disciplinar, rege-se pela legislação federal que lhe é pertinente, pelo Regimento dos Órgãos da Administração Superior e das Unidades Universitárias e pelas Resoluções de seus Órgãos.

A instituição foi criada para atender a um anseio antigo da região do ABC paulista por uma universidade pública e de qualidade. A UFABC busca ser reconhecida como uma referência no panorama nacional e internacional, por meio de sua atenção às demandas regionais, produzindo pesquisas e formando profissionais de alta qualidade para enfrentá-las. Sua missão é \textit{``facilitar e induzir a interdisciplinaridade, promovendo a visão sistêmica e a apropriação do conhecimento pela sociedade, sem esmorecimento da rigorosa cultura disciplinar''}. Para esse propósito, a UFABC procura ter um olhar voltado para o mundo e, ao mesmo tempo, procura caminhar lado a lado com a sociedade e o setor produtivo.

Nesse propósito, a atuação acadêmica da UFABC se dá através de cursos de graduação, pós-graduação e extensão, visando à formação e o aperfeiçoamento de recursos humanos solicitados para o progresso da sociedade brasileira. Além disso, a instituição promove e estimula a pesquisa científica, tecnológica e a produção de pensamento original no campo da ciência e da tecnologia.

A UFABC oferece atualmente um total de 1.960 (um mil, novecentas e sessenta) vagas de ingresso anuais, destinadas aos seus cursos de Bacharelado Interdisciplinar. São eles:

\begin{itemize}
	\item Câmpus Santo André:
	\begin{itemize}
		\item Bacharelado em Ciência e Tecnologia (BC\&T): 1.125 vagas;
	\end{itemize}
	\item Câmpus São Bernardo do Campo:
	\begin{itemize}
		\item Bacharelado em Ciência e Tecnologia (BC\&T): 435 vagas;
		\item Bacharelado em Ciências e Humanidades (BC\&H): 400 vagas.
	\end{itemize}
\end{itemize}

Todos os alunos de graduação da UFABC ingressam por meio de um Bacharelado Interdisciplinar, que deve ser concluído em até três anos.


\subsection{O CURSO DE BACHARELADO EM CIÊNCIA DA COMPUTAÇÃO DA UFABC}

O curso de Bacharelado em Ciência da Computação (BCC), previsto no Projeto Pedagógico Institucional da UFABC, faz parte do planejamento global da instituição, que tem entre seus objetivos tornar-se um pólo produtor de conhecimento de nível nacional e internacional  no âmbito das ciências, cultura e artes.

O BCC está sediado no câmpus Santo André e iniciou seu funcionamento a partir do Edital de Vestibular ocorrido em 02 de maio de 2006, publicado no D.O.U, Seção 3, Nº 83, 03 de maio de 2006, pág. 25.

O BCC tem a duração mínima de quatro anos, podendo ser concluído em prazo menor a depender do desempenho do aluno e do regime de matrículas da UFABC. A duração máxima do curso é de oito anos, conforme a Resolução ConsEP No. 166, de 08 de outubro de 2013. Deve-se atentar ao prazo máximo de 18 quadrimestres para integralização do BC\&T, conforme Resolução ConsEPE No. 166, de 08 de outubro de 2013.

A admissão no Bacharelado em Ciência da Computação pode ser realizada por discentes que estão cursando ou já concluíram o BC\&T. As disciplinas sugeridas na matriz curricular do BC\&T podem ser cursadas paralelamente às disciplinas sugeridas na matriz curricular do BCC. Apesar disso, a colação de grau no BCC está vinculada à colação de grau no BC\&T, de modo que o aluno que desejar colar grau no BCC já deve possuir o grau de Bacharel em Ciência e Tecnologia.  A colação de grau de ambos os cursos também pode ser realizada de forma conjunta.

Além de garantir aos egressos uma sólida e abrangente formação em Ciência da
Computação por meio de suas disciplinas obrigatórias e de opção limitada, o curso se compromete com atividades complementares à sua formação, tais como monitoria acadêmica, iniciação científica e atividades extensionistas.

Este projeto pedagógico baseia seu conteúdo na integração dos seguintes documentos reguladores:

\begin{itemize}
	\item Projeto Pedagógico Institucional (PPI) da UFABC (2017);
	\item Projeto pedagógico do Bacharelado em Ciência e Tecnologia - BC\&T (2015)
	\item Diretrizes curriculares nacionais para os cursos de graduação da Computação (2016)
	\item Referenciais de formação para cursos de Graduação em Computação (2017).
	\item Plano de Desenvolvimento Institucional (PDI) da UFABC (2013-2022)
\end{itemize}




No 3o. quadrimestre de 2010, formou-se a primeira turma do Bacharelado em Ciência da Computação e, em março de 2011, a comissão designada pelo INEP/MEC emitiu parecer favorável ao reconhecimento do curso, atribuindo ao mesmo o conceito máximo 5 (cinco).
Na aplicação do Exame Nacional de Desempenho de Estudantes (Enade) realizado em 2017, o curso também obteve conceito máximo 5 (cinco).
