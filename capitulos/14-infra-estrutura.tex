\section{Infraestrutura}

A UFABC é uma universidade multicampi.
Tanto o campus de Santo André como o campus de São Bernardo do Campo possuem
biblioteca, conexão de internet de alta velocidade, laboratórios didáticos de
experimentação, de ensino e computação, laboratórios de pesquisa, biotérios de
criação e manutenção de animais de experimentação, setores administrativos,
salas de reunião e salas de docentes.

\subsection{Campus Santo André}

O `Bloco A' de edifícios do Campus Santo André mede cerca de 39.000 m² e é
onde está localizada a maior parte das salas de aula, laboratórios de pesquisa
e salas de docentes daquele campus.
Esta obra possui três torres principais, cada uma relacionada a um centro desta
universidade: Centro de Engenharias, Modelagem e Ciências Sociais Aplicadas
(CECS), Centro de Ciências Naturais e Humanas (CCNH) e Centro de Matemática,
Computação e Cognição (CMCC).
As três edificações estão interligadas por áreas comuns nos primeiros três
andares de cada prédio.
Nessas áreas comuns estão instaladas salas de aula da graduação e setores
administrativos.
A ideia de continuidade física entre as áreas da UFABC está em consonância com
seu projeto de criação, que visa a interdisciplinaridade como sua principal
meta.

Algumas salas de docentes, laboratórios didáticos e de pesquisa, e salas de
aula também estão localizados no prédio de 11 andares adjacente ao `Bloco A',
denominado `Bloco B'.
O `Bloco L', com área construída de mais de 16.800 m², abriga 72
laboratórios didáticos e de pesquisa dos três Centros, além de lanchonetes,
almoxarifado entre outros.
Esse campus conta ainda com uma biblioteca, restaurante universitário e quadras
para esportes.

\subsection{Campus São Bernardo do Campo} 

O campus de São Bernardo do Campo possui laboratórios didáticos para
experimentação e computação nos Blocos `Alfa' e `Tau'.
O Bloco `Beta' abriga a biblioteca, anfiteatros e um amplo auditório de 400
lugares.
Estão alocados nos laboratórios didáticos do bloco Alfa diversos modelos
anatômicos e sistemas de ensino de fisiologia (i-Works).
Além disso, contamos também com os laboratórios didáticos específicos das
Engenharias no `Bloco Omega' e laboratórios de pesquisa no `Bloco Zeta' e um
Biotério de caráter multiusuário de criação e manutenção de animais de
experimentação.

\subsection{Laboratórios didáticos}

A Pró-Reitoria de Graduação (PROGRAD) possui em sua infraestrutura um grupo de
laboratórios compartilhados por todos os cursos de graduação. 
A Coordenadoria dos Laboratórios Didáticos (CLD), vinculada à PROGRAD, é
responsável pela gestão administrativa dos laboratórios didáticos e por
realizar a interface entre docentes, discentes e técnicos de laboratório nas
diferentes áreas, de forma a garantir o bom andamento dos cursos de graduação,
no que se refere às atividades práticas em laboratório.

A CLD é composta por um Coordenador dos Laboratórios Úmidos, um Coordenador dos
Laboratórios Secos e um Coordenador dos Laboratórios de Informática e Práticas
de Ensino, bem como equipe técnico-administrativa. 
Dentre as atividades da CLD destacam-se o atendimento diário a toda comunidade
acadêmica, a elaboração de Política de Uso e Segurança dos Laboratórios
Didáticos e a análise e adequação da alocação de turmas nos laboratórios em
cada quadrimestre letivo, garantindo a adequação dos espaços às atividades
propostas em cada disciplina e melhor utilização de recursos da UFABC, o
gerenciamento da infraestrutura dos laboratórios didáticos, materiais, recursos
humanos, treinamento, encaminhamento para manutenção preventiva e corretiva de
todos os equipamentos. 

Os laboratórios são dedicados às atividades didáticas práticas que necessitam
de infraestrutura específica e diferenciada, não atendidas por uma sala de aula
convencional. 
São quatro diferentes categorias de laboratórios didáticos disponíveis para os
usos dos cursos de graduação da UFABC: 
\begin{description}
    \item[Laboratórios Didáticos Secos:] são espaços destinados às aulas da
    graduação que necessitem de uma infraestrutura com bancadas e instalação
    elétrica e/ou instalação hidráulica e/ou gases, uso de kits didáticos e
    mapas, entre outros;
    \item[Laboratórios Didáticos Úmidos:] são espaços destinados às aulas da
    graduação que necessitem manipulação de agentes químicos ou biológicos, uma
    infraestrutura com bancadas de granito, com capelas de exaustão e com
    instalações hidráulica, elétrica e de gases;
    \item[Laboratórios Didáticos Práticas de Ensino:] são espaços destinados ao
    suporte dos cursos de licenciatura, desenvolvimento de habilidades e
    competências para docência da educação básica, podendo ser úteis também
    para desenvolvimentos das habilidades e competências para docência do
    ensino superior;
    \item[Laboratórios Didáticos de Informática:] são espaços para aulas
    utilizando recursos de tecnologia de informação como microcomputadores,
    acesso à internet, linguagens de programação, softwares, hardwares e
    periféricos.
\end{description}

Anexo a cada laboratório há uma sala de suporte técnico que acomoda quatro
técnicos de laboratório, cumprindo as seguintes funções:
\begin{itemize}
    \item nos períodos extra aula, auxiliam a comunidade no que diz respeito a
    atividades de graduação, pós-graduação e extensão em suas atividades
    práticas (projetos de disciplinas, iniciação científica, mestrado e
    doutorado), participam dos processos de compras levantando a minuta dos
    materiais necessários, fazem controle de estoque de materiais, bem como
    cooperam com os professores durante a realização de testes e experimentos
    que serão incorporados nas disciplinas e preparação do laboratório para a
    aula prática;
    \item nos períodos de aula, oferecem apoio para os professores e alunos
    durante o experimento, repondo materiais, auxiliando no uso de equipamentos
    e prezando pelo bom uso dos materiais de laboratório. Para isso, os
    técnicos são alocados previamente em determinadas disciplinas, conforme a
    sua formação (eletrônica, eletrotécnica, materiais, mecânica, mecatrônica,
    edificações, química, biologia, informática, etc).
\end{itemize}
Os técnicos trabalham em esquema de horários alternados, possibilitando o apoio
às atividades práticas ao longo de todo período de funcionamento da UFABC, das
8h às 23h.

Além dos técnicos, a sala de suporte armazena alguns equipamentos e kits
didáticos utilizados nas disciplinas. 
Há também a sala de suporte técnico, que funciona como almoxarifado,
armazenando demais equipamentos e kits didáticos utilizados durante o
quadrimestre.

A UFABC dispõe ainda de uma oficina mecânica de apoio, com quatro técnicos
especializados na área, e atende a demanda de toda comunidade acadêmica
(centros, graduação, extensão e prefeitura universitária) para a construção e
pequenas reparações de kits didáticos e dispositivos para equipamentos usados
na graduação e pesquisa, além do auxílio a discentes na construção e montagem de
trabalhos de graduação e pós, projetos acadêmicos como BAJA, Aerodesign, etc.
A oficina mecânica atende no horário das 8h às 17h.
Esta oficina está equipada com as seguintes máquinas operatrizes: torno CNC,
centro de usinagem CNC, torno mecânico horizontal, fresadora universal,
retificadora plana, furadeira de coluna, furadeira de bancada, esmeril, serra
de fita vertical, lixadeira, serra de fita horizontal, prensa hidráulica,
máquina de solda elétrica TIG, aparelho de solda oxi-acetilênica, calandra,
curvadora de tubos, guilhotina e dobradora de chapas. 
Além disso, a oficina mecânica possui duas bancadas e uma grande variedade de
ferramentas para trabalhos manuais: chaves para aperto, limas, serras manuais,
alicates de diversos tipos, torquímetros, martelos e diversas ferramentas de
corte de uso comum em mecânica, como também, ferramentas manuais elétricas como
furadeiras manuais, serra tico-tico, grampeadeira, etc. 
Também estão disponíveis vários tipos de instrumentos de medição comuns em
metrologia: paquímetros analógicos e digitais, micrômetros analógicos com
batentes intercambiáveis, micrômetros para medição interna, esquadros e
goniômetros, traçadores de altura, desempeno, escalas metálicas, relógios
comparadores analógicos e digitais e calibradores.
Com estes equipamentos e ferramentas, é possível a realização de uma ampla gama
de trabalhos de usinagem, ajustes, montagem e desmontagem de máquinas e
componentes mecânicos.

A alocação de laboratórios didáticos para as turmas de disciplinas com carga
horária prática ou aquelas que necessitem do uso de um laboratório é feita pelo
coordenador do curso, a cada quadrimestre, durante o período estipulado pela
PROGRAD.

O docente da disciplina com carga horária alocada nos laboratórios didáticos é
responsável pelas aulas práticas da disciplina, não podendo se ausentar do
laboratório durante a aula prática.
Atividades como treinamentos, instalação ou manutenção de equipamentos nos
laboratórios didáticos ou aulas pontuais são previamente agendadas com a equipe
técnica responsável e acompanhadas por um técnico de laboratório.

Como os laboratórios são compartilhados, todos os cursos podem realizar de
diferentes atividades didáticas dentro dos diversos laboratórios, otimizando o
uso dos recursos materiais e ampliando as possibilidades didáticas dos docentes
da UFABC e a prática da interdisciplinaridade, respeitando as necessidades de
cada disciplina ou aula de acordo com a classificação do laboratório e dos
materiais e equipamentos disponíveis nele.

\subsection{Sistema de bibliotecas}

O Sistema de Bibliotecas (SisBi) da UFABC, cuja finalidade é atender as demandas
informacionais da comunidade universitária e científica interna e externa à
Universidade, é formado por unidades de bibliotecas localizadas nos campi de
Santo André e São Bernardo do Campo, responsáveis por atender e apoiar a
comunidade universitária em suas atividades de ensino pesquisa e extensão, de
forma articulada e pautada na proposta interdisciplinar do projeto pedagógico e
de seu plano de desenvolvimento institucional.

As bibliotecas que compõem o sistema possuem amplo e diversificado acervo, com
aproximadamente 100.000 exemplares de livros físicos e 42.000 títulos de livros
eletrônicos, sendo todas as coleções da editora Springer Nature entre os anos
de 2005 e 2014, todos os títulos publicados pela editora Wiley em 2016 e
pelos títulos da editora Ebsco referentes a coleção EbscoHost. 
Além disso, conta com títulos resultantes de assinaturas anuais com demais
editoras, como Elsevier, Cengage Learning e Wiley, além de ter uma filmoteca,
que conta com mais de 1.000 títulos de filmes.

O SisBi ainda dispõe do sistema SophiA, que permite o acesso ao seu catálogo e
portal na internet para acesso às informações sobre seus serviços e a conteúdos
externos, como: sistema Scielo, que contempla seleção de periódicos científicos
brasileiros; sistema Biblioteca Digital Brasileira de Teses e Dissertações;
sistema COMUT, que permite a obtenção de cópias de documentos
técnico-científicos disponíveis nos acervos das principais bibliotecas
brasileiras e em serviços de informações internacionais; Portal de Periódicos
da CAPES, que oferece uma seleção das mais importantes fontes de informação
científica e tecnológica, de acesso gratuito na Web.
Atualmente, o portal dispõe de 34.457 periódicos eletrônicos, relacionados às
diversas áreas do conhecimento e, ainda, acesso a mais de 2.000 bases de dados;
dentre outros.

Convênios também são estabelecidos pelo SisBi.
Entre os mais significativos está o serviço de Empréstimo Entre Bibliotecas,
que estabelece a cooperação e potencializa a utilização do acervo das
instituições universitárias participantes, favorecendo a disseminação da
informação entre universitários e pesquisadores de todo o país. 
Outro convênio a ser notado é com o IBGE, que tem por objetivo ampliar para a
sociedade o acesso às informações produzidas por meio de cooperação técnica
com o Centro de Documentação e Disseminação de Informações do IBGE.
Assim, o SisBi passou a ser depositário das publicações editadas por esse órgão.

As unidades de bibliotecas atendem a comunidade de segunda a sexta, das 8h às
22h, mantendo-se em uma estrutura física com área total de 4.529 m², onde se
distribuem 521 assentos, além de terminais de consulta ao acervo.
Buscando promover o exercício a reflexão crítica nos espaços universitários,
bem como a interação com os diversos públicos, desenvolve ainda programas e
projetos culturais como CineArte, exibido também ao ar livre, PublicArte,
Saraus e Exposições.

\subsection{Tecnologias digitais}

As Tecnologias de Informação e Comunicação (TIC) têm sido cada vez mais
utilizadas no processo de ensino e aprendizagem.
Sua importância não está restrita apenas à oferta de disciplinas e cursos
semipresenciais, ou totalmente a distância, ocupando um espaço importante
também como mediadoras em disciplinas e cursos presenciais.
As salas de aula da UFABC são equipadas com projetor multimídia e um
computador, e as disciplinas práticas, que demandam o uso de computadores e
internet, são ministradas em laboratórios equipados com 30-48 computadores com
acesso à Internet, projetor multimídia e softwares relacionados às atividades
desenvolvidas. 
Estão disponíveis também 10 lousas digitais, distribuídas em salas específicas
de cada centro. 
Para o uso dessas ferramentas e infraestrutura, os docentes contam com o
suporte técnico do Núcleo de Tecnologia da Informação (NTI) e da Coordenação de
Laboratórios Didáticos (CLD).

\subsection{Ambiente virtual de aprendizagem}

Com o intuito de estimular a integração das TIC, a UFABC incentiva o uso de um
Ambiente Virtual de Aprendizagem (AVA), TIDIA 4 ou Moodle, como ferramenta de
apoio ao ensino presencial e semipresencial nas diversas disciplinas. 
O AVA pode possibilitar a interação entre alunos e professores por meio de
ferramentas de comunicação síncrona (e.g. bate papo/chat) e assíncrona (e.g.
fórum de discussões, correio eletrônico), além de funcionar como repositório de
conteúdo didáticos, e permitir propostas de atividades individuais e
colaborativas.

\subsection{Núcleo educacional de tecnologias e línguas}

No âmbito da utilização das TIC nas diferentes modalidades de ensino e
aprendizagem (presencial, semipresencial e a distância), a UFABC conta com o
apoio do Núcleo Educacional de Tecnologias e
Línguas\footnote{\url{http://netel.ufabc.edu.br/}} (NETEL).
O NETEL está organizado em seis divisões (Cursos, Design e Inovação
Educacional, Tecnologias da Informação, Audiovisual, Comunicação e idiomas), e
oferece cursos de extensão e oficinas para capacitação de docentes interessados
na integração de novas metodologias e tecnologias digitais nas suas práticas de
ensino. 
Os cursos e oficinas são oferecidos periodicamente, nas modalidades presencial
e semipresencial, e possibilitam a formação e a atualização em diferentes
domínios, por exemplo: docência com tecnologias, desenvolvimento de objetos de
aprendizagem, jogos digitais educacionais, videoaulas, webconferência, lousa
digital, metodologias ativas de ensino, ferramentas digitais de apoio à
aprendizagem. 

Para apoiar a oferta de disciplinas na modalidade semipresencial, outras
iniciativas formativas do NETEL são os cursos semipresenciais ``Planejamento de
cursos virtuais'', que se configura em uma oportunidade de reflexão e
compartilhamento de ideias sobre estratégias, ferramentas e métodos que apoiam
a criação de espaços virtuais de aprendizagem, e o curso ``Formação de Tutores
para EAD'', que tem como objetivo capacitar alunos de graduação e pós-graduação
e pessoas interessadas em atuar como tutores/monitores. 

Para apoiar o docente na criação e oferta de disciplinas na modalidade
semipresencial, o NETEL conta com uma equipe de profissionais da área de Design
Instrucional e especialistas no desenvolvimento de recursos educacionais, como
objetos de aprendizagem e jogos educacionais. 
O NETEL possui também uma divisão de audiovisual com infraestrutura completa de
estúdio e equipamentos para gravação de videoaulas e podcasts. 
O estúdio proporciona apoio à comunidade acadêmica em diversos projetos de
extensão e outras iniciativas que demandam o uso de recursos audiovisuais como
filmagem de aulas, palestras etc. 
Em 2019 o NETEL passou a integrar uma nova divisão, de idiomas, a qual é
responsável por desenvolver a política linguística da UFABC através de ofertas
de cursos de línguas gratuitos e presenciais como cursos de inglês, português,
espanhol e francês.

Por se tratar de uma instituição que busca excelência no uso das TIC, muitos
pesquisadores da UFABC têm desenvolvido pesquisas interdisciplinares nas áreas
de Educação, Ensino, Ciência da Computação, Comunicação etc., com o objetivo de
compreender as potencialidades de uso das TIC e sua influência nos processos de
ensino e aprendizagem. 
Neste contexto, os docentes envolvidos no núcleo juntamente com parceiros da
UFABC desenvolvem pesquisas com a finalidade de renovação e atualização
constante das TICs tanto no ensino quanto apoio ao mesmo.


\subsection{Oferta de disciplinas semipresenciais}

A Portaria MEC nº 2.117, de 6 de dezembro de 2019 (disponível em
\url{https://www.in.gov.br/en/web/dou/-/portaria-n-2.117-de-6-de-dezembro-de-2019-232670913}),
orienta sobre a oferta, por IES, de disciplinas na modalidade à distância em
cursos de graduação presencial. 
Neste sentido, as coordenações dos cursos de graduação juntamente com o seu
corpo docente poderão decidir como farão o uso desta portaria no sentido de
incluir componentes curriculares que, no todo ou em parte, utilizem a
modalidade de ensino semipresencial ou a distância, desde que esta oferta não
ultrapasse 40\% (quarenta por cento) da carga horária do curso. 
Uma mesma disciplina do curso poderá ser ofertada nos formatos presencial e
semipresencial, com Planos de Ensino devidamente adequados à sua oferta.
O número de créditos atribuídos a um componente curricular será o mesmo em
ambos os formatos. 

Para fins de registros escolares, não existe qualquer distinção entre as
ofertas presencial ou semipresencial de um dado componente curricular.
Portanto, em ambos os casos, as TICs, o papel dos tutores e dos docentes, a
metodologia de ensino, e o material didático a serem utilizados deverão ser
detalhados em proposta de Plano de Aula a ser avaliado pela coordenação do
curso antes de sua efetiva implantação. 
O uso desta portaria é de grande importância, pois motiva o uso das TICs nas
disciplinas de graduação, favorecendo a renovação e modernização do ensino e
criando oportunidade para o desenvolvimento das habilidades digitais tanto dos
docentes quanto dos alunos da UFABC.

\subsection{Acessibilidade}

A UFABC possui ainda um Núcleo de Acessibilidade, lotado na Pró-Reitoria de
Assuntos Comunitários e Políticas Afirmativas (ProAP), responsável por executar
as políticas de assistência estudantil direcionadas aos estudantes com
deficiência da nossa comunidade.
Essas ações e projetos visam eliminar as barreiras arquitetônicas, atitudinais
e de comunicação, promovendo a inclusão das pessoas com deficiência. 

É papel da ProAP dar suporte a estudantes com deficiência ou necessidades
educacionais específicas, além de orientar a comunidade acadêmica nas questões
que envolvem o atendimento educacional destes estudantes. 
Além disso, a fim de possibilitar à pessoa com deficiência viver de forma
autônoma e participar de todos os aspectos da vida acadêmica, a ProAP preza
pela disseminação do conceito de desenho universal, conforme disposto na
legislação vigente. 

Orientar o corpo docente, acolher aos estudantes respeitando suas
especificidades, difundir e oferecer Tecnologias Assistivas, dar suporte de
monitoria acadêmica as disciplinas da graduação, disponibilizar tradução e
interpretação de LIBRAS, além da oferta de alguns programas de subsídios
financeiros propostos pelo Plano Nacional de Assistência Estudantil - PNAES,
também fazem parte dos programas em acessibilidade da UFABC.
