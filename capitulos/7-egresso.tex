\section{Perfil do Egresso}

O egresso no BCC deve estar preparado para atuar no mercado de trabalho, propondo
soluções adequadas para situações já conhecidas, bem como atuar de maneira inovadora em contextos e problemas ainda não explorados. Sendo assim, este profissional pode continuar suas atividades na pesquisa, promovendo o desenvolvimento científico, ou aplicando os conhecimentos científicos, promovendo o desenvolvimento tecnológico.

O egresso deverá ainda ter a autonomia intelectual para desenvolver-se em um processo constante de educação continuada.
O bacharel em Ciência da Computação da UFABC poderá atuar nas seguintes áreas:
\begin{itemize}
	\item Organizações públicas, privadas e do terceiro setor;
	\item Empreendedorismo;
	\item Atividades de pesquisa;
	\item Consultorias.
\end{itemize}

Do egresso do curso de Bacharelado em Ciência da Computação espera-se uma
predisposição e aptidões para a área, assim como competências relacionadas às atividades profissionais. Entende-se o termo competência como a capacidade de exercer aptidões, obtidas principalmente através dos conhecimentos e práticas adquiridos no decorrer do curso.

A seguir são apresentadas as competências a serem adquiridas pelos egressos, com suas respectivas habilidades:
