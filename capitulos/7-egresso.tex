\section{Perfil do egresso}
\label{sec:perfil_egresso}

O BCC baseia-se em dois conjuntos fundamentais de documentos para a composição de sua proposta pedagógica e curricular e formação do perfil de egresso:
\begin{itemize}
	\item Diretrizes curriculares nacionais (DCNs) para cursos na área de Computação (ver Seção \ref{subsec:fund_legal});
	\item Projeto pedagógico institucional da UFABC (PPI).
\end{itemize}

Em relação às DCNs da área de Computação, a estrutura curricular se orienta nas exigências quanto ao "Perfil Geral de Egressos de Cursos na Área de Computação" e o "Perfil Específico de Egressos de Cursos de Bacharelado em Ciência da Computação", listados a seguir. 

O BCC trabalha a formação de seus egressos para que:
\begin{itemize}
	\item possuam sólida formação em Ciência da Computação e Matemática que os capacitem a construir aplicativos de propósito geral, ferramentas e infraestrutura de software de sistemas de computação e de sistemas embarcados, gerando conhecimento científico e inovação;
	\item desenvolvam visão global e interdisciplinar de sistemas e entendam que esta visão transcende os detalhes de implementação dos vários componentes e os conhecimentos	dos domínios de aplicação;
	\item conheçam a estrutura dos sistemas de computação e os processos envolvidos na sua construção e análise;
	\item dominem os fundamentos teóricos da área de Computação e como eles influenciam a prática profissional;
	\item sejam capazes de agir de forma reflexiva na construção de sistemas de computação, compreendendo o seu impacto direto ou indireto sobre as pessoas e a sociedade;
	\item sejam capazes de criar soluções, individualmente ou em equipe, para problemas complexos caracterizados por relações entre domínios de conhecimento e de aplicação;
	\item reconheçam o caráter fundamental da inovação e da criatividade e compreendam as perspectivas de negócios e oportunidades relevantes.
\end{itemize}

    Além disso e de forma mais ampla, o BCC trabalha seus egressos para desenvolver a capacidade de: 
\begin{itemize}
	\item identificar problemas que tenham solução algorítmica;
	\item conhecer os limites da computação;
	\item resolver problemas usando ambientes de programação;
	\item tomar decisões e inovar, com base no conhecimento do funcionamento e
	\item características técnicas de hardware e da infraestrutura de software dos sistemas de computação consciente dos aspectos éticos, legais e dos impactos ambientais decorrentes;
	\item compreender e explicar as dimensões quantitativas de um problema;
	\item gerir a sua própria aprendizagem e desenvolvimento, incluindo a gestão de tempo e competências organizacionais;
	\item preparar e apresentar seus trabalhos e problemas técnicos e suas soluções para audiências diversas, em formatos apropriados (oral e escrito);
	\item avaliar criticamente projetos de sistemas de computação;
	\item adequar-se rapidamente às mudanças tecnológicas e aos novos ambientes de trabalho;
	\item ler textos técnicos na língua inglesa;
	\item empreender e exercer liderança, coordenação e supervisão na sua área de atuação profissional;
	\item realizar trabalho cooperativo e entender os benefícios que este pode produzir.
	
\end{itemize}


De forma complementar, o BCC trabalha sua estrutura curricular e suas ações em conformidade com as três principais políticas institucionais previstas no PPI da UFABC: interdisciplinaridade, excelência e inclusão social. Por meio das iniciativas institucionais da UFABC, o aluno possui ampla liberdade para complementar sua formação em diversas áreas de conhecimento trabalhadas pelos cursos da universidade. Com suas várias oportunidades de integração com iniciativas de pesquisa e extensão, os alunos também encontram oportunidades para participar de projetos de pesquisa científica e tecnológica, além de acesso a equipamentos e técnicas avançadas de pesquisa moderna. Utilizando diversas iniciativas de inclusão social, a UFABC, dentro de suas possibilidades orçamentárias e legais, busca ampliar seu alcance para a comunidade local, promovendo ações que buscam democratizar o acesso ao ensino superior, compartilhar os resultados de iniciativas científicas, e abrir oportunidades para alunos em situação de vulnerabilidade.

Com isso e em consonância com as DCNs da área de Computação, também é papel do curso garantir que seu egresso se dotado:

\begin{itemize}
	\item de conhecimento das questões sociais, profissionais, legais, éticas, políticas e humanísticas;
	\item da compreensão do impacto da computação e suas tecnologias na sociedade no que concerne ao atendimento e à antecipação estratégica das necessidades da sociedade;
	\item de visão crítica e criativa na identificação e resolução de problemas contribuindo para o desenvolvimento de sua área;
	\item da capacidade de atuar de forma empreendedora, abrangente e cooperativa no atendimento às demandas sociais da região onde atua, do Brasil e do mundo;
	\item de utilizar racionalmente os recursos disponíveis de forma transdisciplinar;
	\item da compreensão das necessidades da contínua atualização e aprimoramento de suas competências e habilidades;
	\item da capacidade de reconhecer a importância do pensamento computacional na vida cotidiana, como também sua aplicação em outros domínios e ser capaz de aplicá-lo em circunstâncias apropriadas; e
	\item da capacidade de atuar em um mundo de trabalho globalizado.
	
\end{itemize}

