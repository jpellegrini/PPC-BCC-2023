\section{Perfil do egresso}
\label{sec:perfil_egresso}

O egresso no BCC deve estar preparado para atuar no mercado de trabalho,
propondo soluções adequadas para situações já conhecidas, bem como atuar de
maneira inovadora em contextos e problemas ainda não explorados.
Sendo assim, este profissional pode continuar suas atividades na pesquisa,
promovendo o desenvolvimento científico, ou aplicando os conhecimentos
científicos, promovendo o desenvolvimento tecnológico.

O egresso deverá ainda ter a autonomia intelectual para desenvolver-se em um
processo constante de educação continuada.
O bacharel em Ciência da Computação da UFABC poderá atuar em áreas como:
\begin{itemize}
    \item Organizações públicas, privadas e do terceiro setor;
    \item Empreendedorismo;
    \item Atividades de pesquisa;
    \item Consultorias.
\end{itemize}

Do egresso do BCC espera-se uma predisposição e aptidões para a área, assim
como competências relacionadas às atividades profissionais.
Entende-se o termo \textit{competência} como a capacidade de exercer aptidões,
obtidas principalmente através dos conhecimentos e práticas adquiridos no
decorrer do curso.

A seguir são apresentadas as competências a serem adquiridas pelos egressos,
com suas respectivas habilidades \textcolor{red}{Talvez aqui deveria ter algo com as novas DNCs??? O texto abaixo é idêntico ao do PPC 2017.}:
\begin{enumerate}
    \item \textbf{Forte embasamento conceitual nas áreas de formação básica, e
    na formação de uma visão holística da área de Computação.}

    Pretende-se com esta competência desenvolver o raciocínio lógico e abstrato
    do estudante, tendo como arcabouço a área de formação básica e suas
    matérias: Ciência da Computação, Matemática e Física.
    As habilidades a serem desenvolvidas nos alunos são:
    \begin{itemize}
        \item Visão sistêmica e integrada da área de Computação;
        \item Forte conhecimento dos aspectos científicos e tecnológicos
        relacionados à área de Computação.
    \end{itemize}

    \item \textbf{Domínio do processo de projeto e implementação de sistemas
    computacionais, envolvendo o conhecimento do conceito de software e
    hardware.}

    O cientista de Computação tem como uma das principais atividades projetar
    sistemas computacionais em seu aspecto mais amplo, o que envolve elementos
    de hardware e de software. A ele cabe analisar a aplicação a que se destina
    o sistema computacional, escolhendo as configurações, estruturas e funções
    mais adequadas para a aplicação em questão.
    A seguir são descritas as habilidades relacionadas nesse grupo:
    \begin{itemize}
        \item Iniciar, projetar, desenvolver, implementar, validar, gerenciar e
        avaliar projetos de software;
        \item Projetar e desenvolver sistemas que integrem hardware e software;
        \item Pesquisar e viabilizar soluções de software para várias áreas de
        conhecimento e aplicação;
        \item Conhecer aspectos relacionados à evolução da área de Computação,
        de forma a poder compreender a situação presente e projetar a evolução
        futura.
    \end{itemize}

    \item \textbf{Aplicação dos conhecimentos específicos de diversas áreas da
    Computação.}

    Dentro deste domínio, pretende-se aprimorar os conhecimentos e habilidades
    dos estudantes em disciplinas específicas nas seguintes áreas:
    \begin{itemize}
        \item Estruturas Discretas;
        \item Fundamentos da Programação;
        \item Algoritmos e Complexidade;
        \item Organização e Arquitetura dos Computadores;
        \item Sistemas Operacionais;
        \item Computação Centrada em Redes;
        \item Linguagem de Programação;
        \item Interação Humano-Computador;
        \item Computação Gráfica e Visual;
        \item Sistemas Inteligentes;
        \item Gestão e Administração da Informação;
        \item Questões Sociais e Profissionais;
        \item Engenharia de Software, e;
        \item Ciência Computacional.
    \end{itemize}

    O estudante deve considerar que as atuais tecnologias, métodos e
    ferramentas para cada uma destas áreas são passíveis de renovação e
    evolução.

    \item \textbf{Atuação em empresas e como empreendedores.}

    Esta competência está refletida na disciplina de Empreendedorismo e no
    estágio supervisionado.
    Ela envolve planejar, ordenar atividades e metas, tomar decisões
    identificando e dimensionando riscos.
    A tomada de decisão deve analisar e definir o uso apropriado, a eficácia e
    o custo-efetividade de recursos humanos, equipamentos, de materiais, de
    procedimentos e de práticas.
    As habilidades a serem desenvolvidas são as seguintes:
    \begin{itemize}
        \item Utilizar o conhecimento sobre a área de Computação, e sua
        familiarização com as tecnologias correntes, para a solução de
        problemas nas organizações para o desenvolvimento de novos
        conhecimentos, ferramentas, produtos, processos e negócios;
        \item Organizar, coordenar e participar de equipes multi e
        interdisciplinares;
        \item Desenvolver a capacidade empreendedora;
        \item Considerar aspectos de negócio no processo de gerenciamento de um
        projeto.
    \end{itemize}

    \item \textbf{Desenvolvimento de atividades de pesquisa.}

    Esta competência está relacionada ao desenvolvimento de pesquisa científica
    e tecnológica, que permita ao aluno ingressar em cursos de pós-graduação,
    ou realizar estas pesquisas na indústria ou em organizações especializadas.
    Pela característica da rápida evolução da Computação, o futuro profissional
    tem que estar em um processo de contínuo aprendizado. 
    As habilidades a serem desenvolvidas são as seguintes:
    \begin{itemize}
        \item Aprofundamento do conhecimento em área (ou áreas) específica(s)
        da Computação, visando possibilitar uma contribuição para o
        desenvolvimento da área;
        \item Ser capaz de identificar e especificar problemas para
        investigação, bem como planejar procedimentos adequados para testar
        suas hipóteses; 
        \item Conhecer e aplicar o método científico de produção e difusão do
        conhecimento na sociedade.
    \end{itemize}

    Neste sentido, a UFABC possui diversos programas ligados ao estímulo da
    pesquisa científica, dentre os quais citamos: i) Pesquisando Desde o
    Primeiro Dia (PDPD), voltado para estudantes ingressantes; ii) Programas de
    Iniciação Científica PIC/UFABC e PIBIC/UFABC-CNPq, que possui bolsas de
    auxílio da própria UFABC e do CNPq, além do regime de voluntariado; iii)
    Programa de Iniciação Científica em Desenvolvimento Tecnológico e Inovação
    (PIBITI/CNPq/UFABC), e; iv) Programa de Iniciação Científica PIBIC Ações
    Afirmativas, voltado para alunos que ingressaram na universidade por meio
    de ações afirmativas.
    Além disso, a UFABC realiza um Simpósio de Iniciação Científica anualmente,
    para divulgar os trabalhos dos alunos inscritos em seus programas.

    \item \textbf{Formação integral do estudante.}

    Com a rápida e constante evolução na área da Computação, o BCC da UFABC
    deve preparar egressos para o processo de educação continuada, que os
    permitirá avançar além das tecnologias atuais, vencendo desta forma os
    desafios do futuro.
    Os egressos do curso devem apresentar um bom nível de comunicação, tanto
    oral quanto escrita, em uma variedade de contextos.
    Também, o egresso deve ser capaz de liderar e ser liderado com espírito de
    equipe, resolvendo situações com flexibilidade e adaptabilidade diante de
    problemas e desafios.
    A visão da importância em pautar seu trabalho pela ética profissional e
    pelo respeito humano deve ser uma característica marcante do futuro
    profissional.
    A seguir são descritas as habilidades relacionadas a esta competência:
    \begin{itemize}
        \item Desenvolver aprendizagem contínua e autônoma;
        \item Apresentar um bom nível de comunicação oral e escrita;
        \item Trabalho em grupo e com equipes inter e multidisciplinares;
        \item Domínio de regras básicas que regem a ética profissional da área
        de Computação, bem como a ética social;
        \item Compreender a atuação profissional como uma forma de intervenção
        do indivíduo na sociedade, devendo esta intervenção refletir uma
        atitude crítica, de respeito aos indivíduos, à legislação, à ética, ao
        meio ambiente, tendo em vista contribuir para a construção da sociedade
        presente e futura.
    \end{itemize}

    Estas habilidades podem ser desenvolvidas na disciplina de Computadores,
    Ética e Sociedade.
\end{enumerate}
