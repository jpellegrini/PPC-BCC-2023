\section{Disciplinas obrigatórias}
\label{sec:disciplinas_obrigatorias}

\subsection{Disciplinas obrigatórias do BC\&T}

A Tabela~\ref{tab:disciplinas_bct} apresenta a lista de todas as
disciplinas obrigatórias do BC\&T, que compõem parte do currículo do BCC.

\begin{table}[h!]
    \caption{Disciplinas obrigatórias do BC\&T, com siglas, número de créditos e carga horária total.}
    \label{tab:disciplinas_bct}

    \centering
    \begin{tabular}{|l|p{.4\textwidth}|c|c|}
        \hline
        \textbf{Código} & \textbf{Nome da disciplina} & \textbf{Créditos T+P (T-P-E-I)} & \textbf{Carga horária}\\
        \hline\hline
        BIS0005-15 & Bases Computacionais da Ciência & 2 (0-2-0-2) & 24h\\
        \hline
        BIR0004-15 & Bases Epistemológicas da Ciência Moderna & 3 (3-0-0-4) & 36h\\
        \hline
        BCS0001-15 & Base Experimental das Ciências Naturais & 3 (0-3-0-2) & 36h\\
        \hline
        BIS0003-15 & Bases Matemáticas & 4 (4-0-0-5) & 48h\\
        \hline
        BCL0306-15 & Biodiversidade: Interações entre Organismos e Ambiente & 3 (3-0-0-4) & 36h\\
        \hline
        BCL0308-15 & Bioquímica: estrutura, propriedade e funções de biomoléculas & 5 (3-2-0-6) & 60h\\
        \hline
        BIR0603-15 & Ciência Tecnologia e Sociedade & 3 (3-0-0-4) & 36h\\
        \hline
        BCM0506-15 & Comunicação e Redes & 3 (3-0-0-4) & 36h\\
        \hline
        BIK0102-15 & Estrutura da Matéria & 3 (3-0-0-4) & 36h\\
        \hline
        BIQ0602-15 & Estrutura e Dinâmica Social & 3 (3-0-0-4) & 36h \\
        \hline
        BIL0304-15 & Evolução e Diversificação da Vida na Terra & 3 (3-0-0-4) & 36h\\
        \hline
        BCJ0203-15 & Fenômenos Eletromagnéticos & 5 (4-1-0-6) & 60h\\
        \hline
        BCJ0204-15 & Fenômenos Mecânicos & 5 (4-1-0-6) & 60h \\
        \hline
        BCJ0205-15 & Fenômenos Térmicos & 4 (3-1-0-4) & 48h\\
        \hline
        BCK0103-15 & Física Quântica & 3 (3-0-0-4) & 36h\\
        \hline
        BCN0402-15 & Funções de Uma Variável & 4 (4-0-0-6) & 48h\\
        \hline
        BCN0407-15 & Funções de Várias Variáveis & 4 (4-0-0-6) & 48h \\
        \hline
        BCN0404-15 & Geometria Analítica & 3 (3-0-0-6) & 36h\\
        \hline
        BIN0406-15 & Introdução à Probabilidade e à Estatística & 3 (3-0-0-4) & 36h\\
        \hline
        BCN0405-15 & Introdução às Equações Diferenciais Ordinárias & 4 (4-0-0-4) & 48h \\
        \hline
        BCM0504-15 & Natureza da Informação & 3 (3-0-0-4) & 36h\\
        \hline
        BCM0505-22 & Processamento da Informação & 4 (0-4-0-4) & 48h\\
        \hline
        BCS0002-15 & Projeto Dirigido & 2 (0-2-0-10) & 24h\\
        \hline
        BCL0307-15 & Transformações Químicas & 5 (3-2-0-6) & 60h\\
        \hline
    \end{tabular}
\end{table}

As ementas das disciplinas são mantidas no Catálogo de Disciplinas da UFABC.

\subsection{Disciplinas obrigatórias do BCC}
\label{sec:disciplinas_obrigatorais_ementas}

A Tabela~\ref{tab:disciplinas_obrigatorias_bcc} apresenta a lista de todas as
disciplinas obrigatórias do BCC.
% A descrição das ementas e bibliografia dessas disciplinas encontra-se nas
% seções a seguir.
% %As regras de transição entre a matriz sugerida do PPC anterior e a matriz
% %sugerida do PPC atual são apresentadas na Seção~\ref{sec:regras_transicao}.


\begin{table}[h!]
    \caption{Lista de disciplinas obrigatórias do BCC, com siblas, número de créditos e carga horária total.}
    \label{tab:disciplinas_obrigatorias_bcc}

    \centering
    \begin{tabular}{|l|p{.4\textwidth}|c|c|}
        \hline
        \textbf{Código} & \textbf{Nome da disciplina} & \textbf{Créditos T+P (T-P-E-I)} & \textbf{Carga horária}\\
        \hline\hline
        MCTB001-17 & \hyperref[disc:alge_lin]{Álgebra Linear} & 6 (6-0-0-5) & 72h \\
        \hline
        MCCC001-23 & \hyperref[disc:aedI]{Algoritmos e Estruturas de Dados I} & 4 (2-2-0-6) & 48h \\
        \hline
        MCCC002-23 & \hyperref[disc:aedII]{Algoritmos e Estruturas de Dados II} & 4 (4-0-0-6) & 48h \\
        \hline
        MCCC003-23 & \hyperref[disc:ag]{Algoritmos em Grafos} & 4 (4-0-0-4) & 48h\\
        \hline
        MCCC004-23 & \hyperref[disc:aaI]{Análise de Algoritmos I} & 4 (4-0-0-4) & 48h \\
        \hline
        MCCC005-23 & \hyperref[disc:aaII]{Análise de Algoritmos II} & 4 (4-0-0-4) & 48h \\
        \hline
        MCTA004-17 & \hyperref[disc:arq]{Arquitetura de Computadores} & 4 (4-0-0-4) & 48h \\
        \hline
        MCTA006-17 & \hyperref[disc:circ_dig]{Circuitos Digitais} & 4 (3-1-0-4) & 48h \\
        \hline
        MCCC006-23 & \hyperref[disc:compi]{Compiladores e Interpretadores} & 4 (4-0-0-4) & 48h \\
        \hline
        MCCC007-23 & \hyperref[disc:cg]{Computação Gráfica} & 4 (0-4-0-4) & 48h  \\
        \hline
        MCTA009-13 & \hyperref[disc:ces]{Computadores, Ética e Sociedade} & 2 (2-0-0-4) & 24h \\
        \hline
        MCTA033-15 & \hyperref[disc:es]{Engenharia de Software} & 4 (4-0-0-4) & 48h \\
        \hline
        MCCC008-23 & \hyperref[disc:ia]{Inteligência Artificial} & 4 (4-0-0-4) & 48h \\
        \hline
        MCZA008-17 & Interação Humano-Computador& 4 (4-0-0-4) & 48h \\
        \hline
        MCCC009-23 & \hyperref[disc:lfa]{Linguagens Formais e Autômatos} & 4 (4-0-0-4) & 48h \\
        \hline
        MCBM006-23 & \hyperref[disc:mdI]{Matemática Discreta} & 4 (4-0-0-4) & 48h \\
        \hline
        MCCC010-23 & \hyperref[disc:mdII]{Matemática Discreta II} & 4 (4-0-0-4) & 48h \\
        \hline
        MCCC011-23 & \hyperref[disc:metod]{Metodologia e Escrita Científica para Ciência da Computação} & 2 (2-0-0-4) & 48h \\
        \hline
        MCCC012-23 & \hyperref[disc:mbd]{Modelagem de Banco de Dados} & 4 (4-0-0-4) & 48h \\
        \hline
       %MCCC013-23 & \hyperref[disc:ol]{Otimização Linear} & 4 (4-0-0-4) & 48h \\
        \hline
        MCTA028-15& \hyperref[disc:pe]{Programação Estruturada} & 4 (2-2-0-4) & 48h \\
       %MCCC014-23 & \hyperref[disc:pe]{Programação Estruturada} & 4 (2-2-0-6) & 48h \\
        \hline
        MCCC015-23 & \hyperref[disc:pf]{Programação Funcional} & 4 (4-0-0-4) & 48h \\
        \hline
        MCTA018-13 & \hyperref[disc:poo]{Programação Orientada a Objetos} & 4 (2-2-0-4) & 48h \\
        \hline
        %MCTA029-23 & \hyperref[disc:tccI]{Trabalho de Conclusão de Curso I} & 4 (0-4-0-4) & 48h \\
        %\hline
        %MCTA030-23 & \hyperref[disc:tccII]{Trabalho de Conclusão de Curso II} & 4 (0-4-0-6) & 48h \\
        %\hline
        MCTA031-23 & \hyperref[disc:tccIII]{Trabalho de Conclusão de Curso} & 4 (12-0-0-12) & 144h \\
        \hline
        MCTA022-17 & \hyperref[disc:redes]{Redes de Computadores} & 4 (3-1-0-4) & 48h \\
        \hline
        MCTA023-17 & \hyperref[disc:seg]{Segurança de Dados} & 4 (3-1-0-4) & 48h \\
        \hline
        MCTA024-13 & \hyperref[disc:sist_dig]{Sistemas Digitais} & 4 (2-2-0-4) & 48h \\
        \hline
        MCTA025-13 & \hyperref[disc:sist_distr]{Sistemas Distribuídos} & 4 (3-1-0-4) & 48h \\
        \hline
        MCTA026-13 & \hyperref[disc:so]{Sistemas Operacionais} & 4 (3-1-0-4) & 48h\\
        \hline
    \end{tabular}
\end{table}

% \incluircurso{obrigatorias/MCTB001-AlgeLin.tex}
% \incluircurso{obrigatorias/MCTA001-AEDI.tex}
% \incluircurso{obrigatorias/MCTA002-AEDII.tex}
% \incluircurso{obrigatorias/MCTAxxx-AG.tex}
% \incluircurso{obrigatorias/MCTA003-AAI.tex}
% \incluircurso{obrigatorias/MCTAxxx-AAII.tex}
% \incluircurso{obrigatorias/MCTA004-ARQ.tex}
% \incluircurso{obrigatorias/MCTA006-CircuitosDigitais.tex}
% \incluircurso{obrigatorias/MCTA007-Compiladores.tex}
% \incluircurso{obrigatorias/MCTA009-CompEticaSoc.tex}
% \incluircurso{obrigatorias/MCTA008-CG.tex}
% \incluircurso{obrigatorias/MCTA033-ES.tex}
% \incluircurso{obrigatorias/MCTA014-IA.tex}
% \incluircurso{obrigatorias/MCTA015-LFA.tex}
% \incluircurso{obrigatorias/MCTB019-MatDiscr.tex}
% \incluircurso{obrigatorias/MCTAxxx-MatDiscrII.tex}
% \incluircurso{obrigatorias/MCTAxxx-Metod.tex}
% \incluircurso{obrigatorias/MCTA037-ModelagemBD.tex}
% \incluircurso{obrigatorias/MCTA017-OtimLinear.tex}
% \incluircurso{obrigatorias/MCTA028-PE.tex}
% \incluircurso{obrigatorias/MCTA016-PF.tex}
% \incluircurso{obrigatorias/MCTA018-POO.tex}
% \incluircurso{obrigatorias/MCTA022-Redes.tex}
% \incluircurso{obrigatorias/MCTA023-SegDados.tex}
% \incluircurso{obrigatorias/MCTA024-SistDigitais.tex}
% \incluircurso{obrigatorias/MCTA025-SistDistrib.tex}
% \incluircurso{obrigatorias/MCTA026-SistOper.tex}
