\section {Disciplinas}

\subsection{Disciplinas Obrigatórias do BC\&T}

A Tabela~\ref{tab:disciplinas_bct} apresenta a lista de todas as
disciplinas obrigatórias do BC\&T, que compõem parte do currículo do BCC.

\begin{table}[!h]
    \caption{Lista de disciplinas obrigatórias do BC\&T.}
    \label{tab:disciplinas_bct}

    \centering
    \begin{longtable}{|l|p{.4\textwidth}|c|c|}
        \hline
        \textbf{Código} & \textbf{Nome da disciplina} & \textbf{Créditos (T-P-E-I)} & \textbf{Carga-horária}\\
        \hline\hline
        BIS0005-15 & Bases Computacionais da Ciência & 2 (0-2-2) & 24h\\
        \hline
        BIJ0207-15 & Bases Conceituais da Energia & 2 (2-0-4) & 24h\\
        \hline
        BIR0004-15 & Bases Epistemológicas da Ciência Moderna & 3 (3-0-4) & 36h\\
        \hline
        BCS0001-15 & Base Experimental das Ciências Naturais & 3 (0-3-2) & 36h\\
        \hline
        BIS0003-15 & Bases Matemáticas & 4 (4-0-5) & 48h\\
        \hline
        BCL0306-15 & Biodiversidade: Interações entre Organismos e Ambiente & 3 (3-0-4) & 36h\\
        \hline
        BCL0308-15 & Bioquímica: Estrutura, Propriedades e Funções de Biomoléculas & 5 (3-2-6) & 60h\\
        \hline
        BIR0603-15 & Ciência, Tecnologia e Sociedade & 3 (3-0-4) & 36h\\
        \hline
        BCM0506-15 & Comunicação e Redes & 3 (3-0-4) & 36h\\
        \hline
        BIK0102-15 & Estrutura da Matéria & 3 (3-0-4) & 36h\\
        \hline
        BIQ0602-15 & Estrutura e Dinâmica Social & 3 (3-0-4) & 36h \\
        \hline
        BIL0304-15 & Evolução e Diversificação da Vida na Terra & 3 (3-0-4) & 36h\\
        \hline
        BCJ0203-15 & Fenômenos Eletromagnéticos & 5 (4-1-6) & 60h\\
        \hline
        BCJ0204-15 & Fenômenos Mecânicos & 5 (4-1-6) & 60h \\
        \hline
        BCJ0205-15 & Fenômenos Térmicos & 4 (3-1-4) & 48h\\
        \hline
        BCK0103-15 & Física Quântica & 3 (3-0-4) & 36h\\
        \hline
        BCN0402-15 & Funções de Uma Variável & 4 (4-0-6) & 48h\\
        \hline
        BCN0407-15 & Funções de Várias Variáveis & 4 (4-0-4) & 48h \\
        \hline
        BCN0404-15 & Geometria Analítica & 3 (3-0-6) & 36h\\
        \hline
        BCK0104-15 & Interações Atômicas e Moleculares & 3 (3-0-4) & 36h\\
        \hline
        BIN0406-15 & Introdução à Probabilidade e Estatística & 3 (3-0-4) & 36h\\
        \hline
        BCN0405-15 & Introdução às Equações Diferenciais Ordinárias & 4 (4-0-4) & 48h \\
        \hline
        BCM0504-15 & Natureza da Informação & 3 (3-0-4) & 36h\\
        \hline
        BCM0505-15 & Processamento da Informação & 5 (3-2-5) & 60h\\
        \hline
        BCS0002-15 & Projeto Dirigido & 2 (0-2-10) & 24h\\
        \hline
        BCL0307-15 & Transformações Químicas & 5 (3-2-6) & 60h\\
        \hline
    \end{longtable}
\end{table}


\subsection{Disciplinas obrigatórias do BCC}

A Tabela~\ref{tab:disciplinas_bcc} apresenta a lista de todas as
disciplinas obrigatórias do BCC.

\begin{table}[!h]
    \caption{Lista de disciplinas obrigatórias do BCC.}
    \label{tab:disciplinas_bcc}

    \centering
    \begin{longtable}{|l|p{.4\textwidth}|c|c|}
        \hline
        \textbf{Código} & \textbf{Nome da disciplina} & \textbf{Créditos (T-P-E-I)} & \textbf{Carga-horária}\\
        \hline\hline
        MCTB001-23 & Álgebra Linear & 6 (6-0-0-5) & 72h \\
        \hline
        MCTA001-23 & Algoritmos e Estruturas de Dados I & 4 (2-2-0-6) & 48h \\
        \hline
        MCTA002-23 & Algoritmos e Estruturas de Dados II & 4 (4-0-0-6) & 48h \\
        \hline
        MCTA027-23 & Algoritmos em Grafos & 4 (4-0-0-4) & 48h\\
        \hline
        MCTA003-23 & Análise de Algoritmos I & 4 (4-0-0-4) & 48h \\
        \hline
        MCTA???-23 & Análise de Algoritmos II & 4 (4-0-0-4) & 48h \\
        \hline
        MCTA004-23 & Arquitetura de Computadores & 4 (4-0-0-4) & 48h \\
        \hline
        MCTA006-23 & Circuitos Digitais & 4 (3-1-0-4) & 48h \\
        \hline
        MCTA007-23 & Compiladores e Interpretadores & 4 (4-0-0-4) & 48h \\
        \hline
        MCTA008-23 & Computação Gráfica & 4 (3-1-0-4) & 48h  \\
        \hline
        MCTA009-23 & Computadores, Ética e Sociedade & 2 (2-0-0-4) & 24h \\
        \hline
        MCTA033-23 & Engenharia de Software & 4 (4-0-0-4) & 48h \\
        \hline
        MCTA014-23 & Inteligência Artificial & 4 (4-0-0-4) & 48h \\
        \hline
        MCTA015-23 & Linguagens Formais e Autômatos & 4 (4-0-0-4) & 48h \\
        \hline
        MCTB019-23 & Matemática Discreta & 4 (4-0-0-4) & 48h \\
        \hline
        MCTA???-23 & Matemática Discreta II & 4 (4-0-0-4) & 48h \\
        \hline
        MCTA037-23 & Modelagem de Banco de Dados & 4 (4-0-0-4) & 48h \\
        \hline
        MCTA017-23 & Otimização Linear & 4 (4-0-0-4) & 48h \\
        \hline
        MCTA028-23 & Programação Estruturada & 4 (2-2-0-4) & 48h \\
        \hline
        MCTA016-23 & Programação Funcional & 4 (4-0-0-4) & 48h \\
        \hline
        MCTA018-23 & Programação Orientada a Objetos & 4 (2-2-0-4) & 48h \\
        \hline
        MCTA029-23 & Trabalho de Conclusão de Curso I & 4 (0-4-0-4) & 48h \\
        \hline
        MCTA030-23 & Trabalho de Conclusão de Curso II & 4 (0-4-0-6) & 48h \\
        \hline
        MCTA031-23 & Trabalho de Conclusão de Curso III & 4 (0-4-0-6) & 48h \\
        \hline
        MCTA022-23 & Redes de Computadores & 4 (3-1-0-4) & 48h \\
        \hline
        MCTA023-23 & Segurança de Dados & 4 (3-1-0-4) & 48h \\
        \hline
        MCTA024-23 & Sistemas Digitais & 4 (2-2-0-4) & 48h \\
        \hline
        MCTA025-23 & Sistemas Distribuídos & 4 (3-1-0-4) & 48h \\
        \hline
        MCTA026-23 & Sistemas Operacionais & 4 (3-1-0-4) & 48h\\
        \hline
    \end{longtable}
\end{table}
