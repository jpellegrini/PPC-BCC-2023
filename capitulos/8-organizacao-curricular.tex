\section {Organização Curricular}

\subsection{Fundamentação Legal}

A seguir são elencados os documentos legais externos (Diretrizes Curriculares Nacionais, Leis, Decretos, Resoluções, Pareceres, Portarias, Normativas etc.), de ordem federal, estadual, de órgão de classe, dentre outros, bem como os internos (Projeto Pedagógico, Plano de Desenvolvimento Institucional) que fundamentam a estrutura curricular do curso de
Bacharelado em Ciência da Computação da UFABC.

\begin{itemize}
	
	\item BRASIL. Presidência da República. Casa Civil. Subchefia para Assuntos Jurídicos. Lei 9.394, de 20 de dezembro de 1996. Estabelece as diretrizes e bases da educação nacional. Disponível em: \url{https://www.planalto.gov.br/ccivil_03/Leis/L9394.htm}. Acesso em: 07 jul. 2016.
	
	\item BRASIL. Presidência da República. Casa Civil. Subchefia para Assuntos Jurídicos. Lei 10.639, de 9 de janeiro de 2003. Altera a Lei 9.394, de 20 de dezembro de 1996, que estabelece as diretrizes e bases da educação nacional, para incluir no currículo oficial da Rede de Ensino a obrigatoriedade da temática ``História e Cultura Afro-Brasileira'', e dá outras providências. Disponível em: \url{http://www.planalto.gov.br/ccivil_03/leis/2003/l10.639.htm}. Acesso em: 07 jul. 2016.
	
	\item BRASIL. Presidência da República. Casa Civil. Subchefia para Assuntos Jurídicos. Lei 11.645, de 10 de março de 2008. Altera a Lei 9.394, de 20 de dezembro de 1996, modificada pela Lei 10.639, de 9 de janeiro de 2003, que estabelece as diretrizes e bases da educação nacional, para incluir no currículo oficial da rede de ensino a obrigatoriedade da temática ``História e Cultura Afro-Brasileira e Indígena''. Disponível em: \url{http://www.planalto.gov.br/ccivil_03/_ato2007-2010/2008/lei/l11645.htm}. Acesso em:
	07 jul. 2016.
	
	\item BRASIL. Ministério da Educação. Conselho Nacional de Educação. Conselho Pleno. Resolução 1, de 17 de junho de 2004. Institui Diretrizes Curriculares Nacionais para a Educação das Relações Étnico Raciais e para o Ensino de História e Cultura Afro-Brasileira e Africana. Disponível em:
	\url{http://portal.mec.gov.br/cne/arquivos/pdf/res012004.pdf}. Acesso em: 07 jul. 2016.
	
	\item BRASIL. Ministério da Educação. Secretaria da Educação Superior. Referenciais Orientadores para os Bacharelados Interdisciplinares e Similares. 2010. Disponível em: \url{http://www.ufabc.edu.br/images/stories/comunicacao/bacharelados-interdisciplinares_referenciais-orientadores-novembro_2010-brasilia.pdf}. Acesso em: 07 jul. 2016.
	
	
	\item BRASIL. Ministério da Educação. Conselho Nacional de Educação. Referenciais orientadores para os Bacharelados Interdisciplinares e Similares das Universidades Federais. Câmara de Educação Superior. Parecer CNE/CES 266, de 5 jul. 2011.Disponível em: \url{http://portal.mec.gov.br/index.php?option=com_content&view=article&id=16418&Itemid=866} Acesso em: 07 jul. 2016.
	
	\item BRASIL. Presidência da República. Casa Civil. Subchefia para Assuntos Jurídicos. Decreto 5.626, de 22 de dezembro de 2005. Regulamenta a Lei  10.436, de 24 de abril de 2002, que dispõe sobre a Língua Brasileira de Sinais - Libras, e o art. 18 da Lei 10.098, de 19 de dezembro de 2000. Disponível em: \url{https://www.planalto.gov.br/ccivil_03/_Ato2004-2006/2005/Decreto/D5626.htm}. Acesso em: 07 jul. 2016.
	
	\item  BRASIL. Presidência da República. Casa Civil. Subchefia para Assuntos Jurídicos. Lei 9.795, de 27 de abril de 1999. Dispõe sobre a educação ambiental, institui a Política Nacional de Educação Ambiental e dá outras providências. Disponível em: \url{http://www.planalto.gov.br/ccivil_03/leis/l9795.htm}. Acesso em: 07 jul. 2016.
	
	\item BRASIL. Presidência da República. Casa Civil. Subchefia para Assuntos Jurídicos. Decreto 4.281, de 25 de junho de 2002. Regulamenta a Lei 9.795, de 27 de abril de 1999, que institui a Política Nacional de Educação Ambiental, e dá outras providências. Disponível em: \url{http://www.planalto.gov.br/ccivil_03/decreto/2002/D4281.htm}. Acesso em: 07 jul. 2016.
	
	\item BRASIL. Ministério da Educação. Conselho Nacional de Educação. Câmara de Educação Superior. Resolução 2, de 18 de junho de 2007. Dispõe sobre carga horária mínima e procedimentos relativos à integralização e duração dos cursos de graduação, bacharelados, na modalidade presencial. Disponível em: \url{http://portal.mec.gov.br/cne/arquivos/pdf/2007/rces002_07.pdf}. Acesso em: 07 jul. 2016.
	
	\item BRASIL. Ministério da Educação. Gabinete do Ministro. Portaria Normativa 40, de 12 de dezembro de 2007. Institui o e-MEC, sistema eletrônico de fluxo de trabalho e gerenciamento de informações relativas aos processos de regulação, avaliação e supervisão da educação superior no sistema federal de educação, e o Cadastro e-MEC de Instituições e Cursos Superiores e consolida disposições sobre indicadores de qualidade, banco de avaliadores (Basis) e o Exame Nacional de Desempenho de Estudantes (ENADE) e outras disposições. Disponível em: \url{http://portal.mec.gov.br/cne/arquivos/pdf/2007/port40_07.pdf}. Acesso em: 07 jul. 2016.
	
	\item BRASIL. Comissão Nacional de Avaliação da Educação Superior. Resolução 1, de 17 de junho de 2010. Normatiza o Núcleo Docente Estruturante e dá outras providências. Disponível em: \url{http://portal.mec.gov.br/index.php?option=com_docman&task=doc_download&gid=6885&Itemid}. Acesso em: 07 jul. 2016.
	
	\item BRASIL. Presidência da República. Casa Civil. Subchefia para Assuntos Jurídicos. Decreto 5.622. Regulamenta o art. 80 da Lei 9.394, de 20 de dezembro de 1996, que estabelece as diretrizes e bases da educação nacional. Disponível em: \url{http://www.planalto.gov.br/ccivil_03/_ato2004-2006/2005/Decreto/D5622compilado.htm}. Acesso em: 07 jul. 2016.
	
	\item BRASIL. Ministério da Educação. Portaria n° 4.059, de 10 de dezembro de 2004. Regulamentação de disciplinas na modalidade semipresencial. Disponível em: \url{http://portal.mec.gov.br/sesu/arquivos/pdf/nova/acs_portaria4059.pdf}. Acesso em: 07 jul. 2016.
	
	\item BRASIL. Ministério da Educação. Conselho Nacional de Educação. Câmara de Educação Superior Parecer CNE/CES nº 136/2012, aprovado em 8 de março de 2012 - Diretrizes Curriculares Nacionais para os cursos de graduação em Computação. Disponível em: \url{http://portal.mec.gov.br/component/content/article?id=12991}. Acesso em: 07 jul. 2016. Publicado no DOU no 134, de 12 de julho de 2012.
	
	\item BRASIL. Ministério da Educação. Conselho Nacional de Educação. Conselho Pleno. Parecer CNE/CP n° 003, de 10 mar. 2004. Disponível em:
	\url{http://portal.mec.gov.br/cne/arquivos/pdf/003.pdf}. Acesso em: 07 jul. 2016.
	
	\item BRASIL. Ministério da Educação. Conselho Nacional de Educação. Conselho Pleno. Resolução n° 1, de 30 de maio de 2012. Estabelece Diretrizes Nacionais para a Educação em Direitos Humanos. Disponível em: \url{http://portal.mec.gov.br/index.php?option=com_docman&view=download&alias=10889-
		rcp001-12&category_slug=maio-2012-pdf&Itemid=30192}. Acesso em: 07 jul. 2016.
	
	\item BRASIL. Presidência da República. Casa Civil. Subchefia para Assuntos Jurídicos. Lei 12.764, de 27 de dezembro de 2012. Institui a Política Nacional de Proteção dos Direitos da Pessoa com Transtorno do Espectro Autista; e altera o § 3o do art. 98 da Lei 8.112, de 11 de dezembro de 1990. Disponível em: \url{http://www.planalto.gov.br/ccivil_03/_ato2011-2014/2012/lei/l12764.htm}. Acesso em: 07 jul. 2016.
	
	\item FUNDAÇÃO UNIVERSIDADE FEDERAL DO ABC. Projeto Pedagógico. Santo André,
	2006. Disponível em: \url{http://www.ufabc.edu.br/images/stories/pdfs/institucional/projetopedagogico.pdf}. Acesso em: 07 jul. 2016.
	
	\item FUNDAÇÃO UNIVERSIDADE FEDERAL DO ABC. Plano de Desenvolvimento
	Institucional. Santo André, 2013. Disponível em: \url{http://www.ufabc.edu.br/index.php?option=com_content&view=article&id=7880%3Areso
		lucao-consuni-no-112-aprova-o-plano-de-desenvolvimento-institucional-2013-
		2022&catid=226%3Aconsuni-resolucoes&Itemid=42}. Acesso em: 07 jul. 2016.
	
	\item BRASIL. Ministério da Educação. Conselho Nacional de Educação. Câmara de Educação Superior. Parecer CNE/CES nº 136/2012, aprovado em 8 de março de
	2012 - Diretrizes Curriculares Nacionais para os cursos de graduação em Computação. Disponível em: \url{http://portal.mec.gov.br/index.php?option=com_docman&view=download&alias=11205-pces136-11-pdf&category_slug=julho-2012-pdf&Itemid=30192}.
	
	\item BRASIL. Ministério da Educação. Conselho Nacional de Educação. Câmara de Educação Superior. Resolução CNE/CES 5, de 16 de novembro de 2016 - Institui as Diretrizes Curriculares Nacionais para os cursos de graduação na área da Computação, abrangendo os cursos de bacharelado em Ciência da Computação, em Sistemas de Informação, em Engenharia de Computação, em Engenharia de Software e de licenciatura em Computação, e dá outras providências. Disponível em: \url{http://portal.mec.gov.br/index.php?option=com_docman&view=download&alias=52101-rces005-16-pdf&category_slug=novembro-2016-pdf&Itemid=30192}.
	
\end{itemize}


\subsection{Regime de Ensino}

\subsubsection{Estrutura Curricular}
A distribuição da quantidade de créditos e da carga-horária a serem cumpridas em cada uma das categorias de disciplinas para a obtenção do grau de Bacharel em Ciência da Computação é dada a seguir:

\begin{center}
	\begin{tabular}{|l|c|c|}
		\hline
		Categoria & Créditos & Carga horária (horas)\\
		\hline\hline
		Disciplinas obrigatórias do BC\&T & 90 & 1080\\
		\hline
		Disciplinas obrigatórias do BCC & 124 & 1488 \\
		\hline
		Disciplinas de opção limitada do BCC & 30 & 360 \\
		\hline
		Disciplinas livres & 12 & 144\\
		\hline
		Total & 256 & \\
		\hline\hline
		Atividades complementares do BC\&T & & 120\\
		\hline
		Total de horas & & 3192\\
		\hline
	\end{tabular}
\end{center}


O currículo do curso tem um eixo central de disciplinas, obrigatório para todos os alunos (excetuando as obrigatórias do BC\&T), que padroniza a formação dos acadêmicos da UFABC. Este eixo totaliza 124 créditos, que corresponde a 48,4\% do curso.

Há um conjunto de disciplinas que podem ser selecionadas pelos estudantes, oferecendo autonomia para projetarem esta carga horária de acordo com seus interesses e aptidões. Tais disciplinas são oferecidas em dois grupos: disciplinas de opção limitada e disciplinas livres. As disciplinas de opção limitada devem ser selecionadas dentre aquelas constantes da Tabela 8 e
totalizam 30 créditos da matriz curricular.

As disciplinas livres objetivam a formação complementar do acadêmico, permitindo a escolha das disciplinas dentre as oferecidas nos cursos de graduação da UFABC. Totalizam 12 créditos da matriz curricular.



\subsubsection{Interdisciplinaridade}
O Bacharelado em Ciência e Tecnologia (BC\&T) é a base da matriz curricular do BCC, de maneira que a formação proposta proporciona interdisciplinaridade e flexibilidade curricular. As disciplinas obrigatórias do BC\&T organizam o conhecimento em eixos (Energia, Processos de Transformação, Representação e Simulação, Informação e Comunicação, Estrutura da Matéria e Humanidades), visando despertar o interesse dos alunos para a investigação de cunho interdisciplinar. Os cursos de graduação da UFABC estão estruturados em um sistema de créditos que permite diferentes organizações curriculares, de acordo com os interesses e aptidões dos alunos. Através das disciplinas livres, os alunos poderão se aprofundar em quaisquer áreas do conhecimento, partindo para especificidades curriculares de cursos de formação profissional ou explorando a interdisciplinaridade e estabelecendo um currículo individual de formação.

É importante destacar que a interdisciplinaridade do presente projeto pedagógico e a possibilidade de escolher disciplinas livres, permite que o discente formado no BCC da UFABC esteja alinhado com as seguintes diretrizes legais:

\begin{itemize}
	
	\item Decreto 5.626 de 22 de Dezembro de 2005: a disciplina de LIBRAS, cuja
	ementa faz parte do rol de disciplinas dos cursos de licenciatura da UFABC, pode ser cursada pelos alunos do BCC.
	
	\item Lei no 11.64, sobre a obrigatoriedade da temática ``História e Cultura Afro-Brasileira e Indígena'' e Resolução 01/2004, de 17 de junho de 2004: o aluno do BCC pode escolher cursar disciplinas livres que fazem parte do rol de disciplinas da UFABC e que envolvem a temática da História e Cultura Afro-Brasileira e Indígenas.
	
	\item Política Nacional de Educação Ambiental (Lei nº 9795/1999 e decreto  4.281, de 25/06/2002): muitas disciplinas livres oferecidas no rol de disciplinas de engenharia ambiental podem ser cursadas pelos alunos do BCC, permitindo assim a integração desse projeto pedagógico com a educação ambiental.
\end{itemize}



\subsection{Estratégias Pedagógicas}

A UFABC foi concebida definindo a interdisciplinaridade como uma referência pedagógica. Sendo assim,  o desenvolvimento
do perfil do egresso é trabalhado com uma formação interdisciplinar com alto grau de liberdade para incorporar  
componentes curriculares à sua formação. Além de cobrir os assuntos pertinentes à formação definida pelas DCNs de Computação,
esse modelo possibilita que o aluno desenvolva competências em outras áreas de seu próprio interesse.

Seguindo a recomendação da matriz curricular, os primeiros quadrimestres letivos de curso são preenchidos por disciplinas do BC\&T, onde o(a) aluno(a) tem contato com várias áreas da Ciência, fortalecendo sua base científica e humanística, além de experimentar os primeiros contatos
com disciplinas da área de Computação (Bases Computacionais da Ciência, Natureza da Informação, Comunicação e Redes, Processamento da Informação). Aos poucos, o(a) aluno(a) vai encontrar janelas de horários para incluir disciplinas específicas de Computação enquanto finaliza sua formação no BC\&T. O projeto pedagógico prevê, ainda, 25\% de carga horária em disciplinas livres e de opção limitada, em que o(a) aluno(a) poderá escolher os componentes curriculares que completarão a sua formação.

Na UFABC as disciplinas não possuem pré-requisitos entre disciplinas. Mesmo assim, a estrutura da matriz curricular sugere uma composição
que favorece o desenvolvimento contínuo das competências e habilidades do egresso durante o desenvolvimento do curso, concentrando 
disciplinas que abordam temas avançados e específicos no final do curso e disciplinas fundamentais em seu início.

O uso de Tecnologias de Informação e Comunicação (TICs) é uma realidade próxima dos estudantes na UFABC. Muitas disciplinas
utilizam ambientes virtuais de aprendizagem (AVAs) para gestão de conteúdo em disciplinas presenciais e semipresenciais. Durante
a pandemia, algumas disciplinas foram também ofertadas na modalidade online com sucesso. Todos os cursos possuem páginas 
específicas em que seus conteúdos e documentos ficam acessíveis à comunidade (projeto pedagógico, informações gerais, documentos, links para outras páginas de recursos, etc.). Uma importante parcela da carga horária total é trabalhada em aulas práticas, ofertadas em laboratórios de informática com computadores ou laboratórios de {\it hardware} com dispositivos eletrônicos.


Em termos de acessibilidade, a UFABC tem se preocupa com a garantia de acesso às
pessoas com deficiência e/ou com mobilidade reduzida. Seguindo as determinações do
Decreto n° 5.296/2004 47 e da Lei 10.098/2000 48, os dois câmpus da UFABC possuem
acessibilidade arquitetônica, garantindo o uso autônomo dos espaços por pessoas com
deficiência e/ou com mobilidade reduzida. Através do Núcleo de Acessibilidade da Pró-Reitoria
de Assuntos Comunitários e Políticas Afirmativas (PROAP), a UFABC tem procurado a
excelência no quesito inclusão. Nesse sentido, dentre as disciplinas oferecidas pela UFABC,
destacamos o oferecimento da disciplina NHI5010-15 - LIBRAS.

Políticas de educação ambiental e de educação em direitos humanos são tratadas 
nas disciplinas ofertadas pela UFABC relacionadas à Educação Ambiental: 
ESZU025-17 - Educação Ambiental; ESHC034-17 - Economia e Meio Ambiente; 
ESZU006-17 – Economia; Sociedade e Meio Ambiente e
ESTE004-17 – Energia, Meio Ambiente e Sociedade.

Dentre as disciplinas ofertadas pela UFABC relacionadas à Educação em Direitos
Humanos citamos: ESHR028-14 - Regime Internacional dos Direitos Humanos e a Atuação
Brasileira; ESZP029-13 - Movimentos Sindicais, Sociais e Culturais; ESZP014-13 - Diversidade
Cultural, Conhecimento Local e Políticas Públicas; BHQ0001-15 - Identidade e Cultura e
ESHP004-13 - Cidadania, Direitos e Desigualdades; ESZR002-13 - Cultura, Identidade e
Política na América Latina, e; ESHR027-14 - Trajetórias Internacionais do Continente Africano.




\newpage
\subsection{Matriz Curricular Recomendada}

\scalebox{.7}{
	\begin{tikzpicture}
		
		%Q1
		\node [draw, rotate=90, black,rectangle, minimum width=120pt, minimum height=10pt, rounded corners] at (-25pt,-40pt) {\footnotesize{1o. ANO}};
		
		\node [draw, rotate=90, black,rectangle, minimum width=40pt, minimum height=10pt, rounded corners] at (-10pt,0pt) {\footnotesize{1o. quad}};
		
		\disciplina {0pt}{0pt}{a}{63}{Base Experimental das Ciências Naturais\\ (0-3-2)}
		
		\disciplina {63pt}{0pt}{b}{42}{Bases Computacionais da Ciência \\(0-2-2)} 
		
		\disciplina {105pt}{0pt}{c}{84}{Bases Matemáticas\\(4-0-5)}
		
		\disciplina {189pt}{0pt}{d}{42}{Bases Conceituais da Energia \\(2-0-4)}
		
		\disciplina {231pt}{0pt}{e}{63}{Estrutura da Matéria \\(3-0-4)}
		
		\disciplina {294pt}{0pt}{f}{63}{Evolução e Diversificação da Vida na Terra \\(3-0-4)}
		
		
		%Q2
		\node [draw, rotate=90, black,rectangle, minimum width=40pt, minimum height=10pt, rounded corners] at (-10pt,-40pt) {\footnotesize{2o. quad}};
		
		\disciplina [yellow!20]{0pt}{-40pt}{bis0505}{63}{Natureza da Informação\\(3-0-4)}
		
		\disciplina [yellow!20]{63pt}{-40pt}{bis0505}{63}{Geometria Analítica\\(3-0-6)}
		
		\disciplina [yellow!20]{126pt}{-40pt}{bis0505}{84}{Funções de Uma Variável\\(4-0-6)}
		
		\disciplina [yellow!20]{210pt}{-40pt}{bis0505}{84}{Fenômenos Mecânicos\\(3-1-4)}
		
		\disciplina [yellow!20]{294pt}{-40pt}{bis0505}{63}{Biodiversidade: Int. Organismos e Ambiente\\(3-0-4)}
		
		%Q3
		\node [draw, rotate=90, black,rectangle, minimum width=40pt, minimum height=10pt, rounded corners] at (-10pt,-80pt) {\footnotesize{3o. quad}};
		
		\disciplina [yellow!20]{0pt}{-80pt}{bis0505}{105}{Processamento da Informação\\(3-2-5)}
		
		\disciplina [yellow!20]{105pt}{-80pt}{bis0505}{84}{Funções de Várias Variáveis\\(4-0-4)}
		
		\disciplina [yellow!20]{189pt}{-80pt}{bis0505}{84}{Fenômenos Térmicos\\(3-1-4)}
		
		\disciplina [yellow!20]{273pt}{-80pt}{bis0505}{105}{Transformações Químicas\\(3-2-6)}
		
		
		%Q4
		\node [draw, rotate=90, black,rectangle, minimum width=120pt, minimum height=10pt, rounded corners] at (-25pt,-160pt) {\footnotesize{2o. ANO}};
		
		\node [draw, rotate=90, black,rectangle, minimum width=40pt, minimum height=10pt, rounded corners] at (-10pt,-120pt) {\footnotesize{4o. quad}};
		
		\disciplina [yellow!20]{0pt}{-120pt}{bis0505}{63}{Comunicação e Redes\\(3-0-4)}
		
		\disciplina [yellow!20]{63pt}{-120pt}{bis0505}{63}{Introdução à Probabilidade e Estatística\\(3-0-4)}
		
		\disciplina [yellow!20]{126pt}{-120pt}{bis0505}{84}{Introdução às Equações Diferencias Ordinárias\\(4-0-4)}
		
		\disciplina [yellow!20]{210pt}{-120pt}{bis0505}{105}{Fenômenos Eletromagnéticos\\(4-1-6)}
		
		\disciplina [yellow!20]{315pt}{-120pt}{bis0505}{63}{Bases Epistemológicas da Ciência Moderna\\(3-0-4)}
		
		%Q5
		\node [draw, rotate=90, black,rectangle, minimum width=40pt, minimum height=10pt, rounded corners] at (-10pt,-160pt) {\footnotesize{5o. quad}};
		
		\disciplina [yellow!20]{0pt}{-160pt}{bis0505}{63}{Física Quântica\\(3-0-4)}
		
		\disciplina [yellow!20]{63pt}{-160pt}{bis0505}{105}{Bioquímica: Estrutura, Propriedade e Funções de Biomoléculas\\(3-2-6)}
		
		\disciplina [yellow!20]{168pt}{-160pt}{bis0505}{63}{Dinâmica e Estrutura Social\\(3-0-4)}
		
		\disciplina [blue!20]{231pt}{-160pt}{bis0505}{84}{Lógica Básica\\(4-0-4)}
		
		\disciplina [blue!20]{315pt}{-160pt}{bis0505}{84}{Programação Estruturada\\(2-2-4)}
		
		%Q6
		\node [draw, rotate=90, black,rectangle, minimum width=40pt, minimum height=10pt, rounded corners] at (-10pt,-200pt) {\footnotesize{6o. quad}};
		
		\disciplina [yellow!20]{0pt}{-200pt}{bis0505}{63}{Interações Atômicas e Moleculares\\(3-0-4)}
		
		\disciplina [yellow!20]{63pt}{-200pt}{bis0505}{63}{Ciência, Tecnologia \\e Sociedade\\(3-0-4)}
		
		\disciplina [blue!20]{126pt}{-200pt}{bis0505}{84}{Circuitos Digitais\\(3-1-4)}
		
		\disciplina [blue!20]{210pt}{-200pt}{bis0505}{84}{Algoritmos e Estruturas de Dados I\\(2-2-4)}
		
		\disciplina [blue!20]{294pt}{-200pt}{bis0505}{84}{Matemática Discreta\\(4-0-4)}
		
		%Q7
		\node [draw, rotate=90, black,rectangle, minimum width=120pt, minimum height=10pt, rounded corners] at (-25pt,-280pt) {\footnotesize{3o. ANO}};
		
		
		\node [draw, rotate=90, black,rectangle, minimum width=40pt, minimum height=10pt, rounded corners] at (-10pt,-240pt) {\footnotesize{7o. quad}};
		
		\disciplina [blue!20]{0pt}{-240pt}{bis0505}{84}{Sistemas Digitais\\(2-2-4)}
		
		\disciplina [blue!20]{84pt}{-240pt}{bis0505}{84}{Análise de Algoritmos\\(4-0-4)}
		
		\disciplina [blue!20]{168pt}{-240pt}{bis0505}{84}{Programação Orientada a Objetos\\(2-2-4)}
		
		\disciplina [blue!20]{252pt}{-240pt}{bis0505}{126}{Álgebra Linear\\(6-0-5)}
		
		\disciplina [blue!20]{378pt}{-240pt}{bis0505}{42}{Computadores, Ética e Sociedade\\(2-0-4)}
		
		%Q8
		\node [draw, rotate=90, black,rectangle, minimum width=40pt, minimum height=10pt, rounded corners] at (-10pt,-280pt) {\footnotesize{8o. quad}};
		
		\disciplina [blue!20]{0pt}{-280pt}{bis0505}{84}{Arquitetura de Computadores\\(4-0-4)}
		
		\disciplina [blue!20]{84pt}{-280pt}{bis0505}{84}{Algoritmos e Estruturas de Dados II\\(2-2-4)}
		
		\disciplina [blue!20]{168pt}{-280pt}{bis0505}{84}{Teoria dos Grafos\\(3-1-4)}
		
		\disciplina [blue!20]{252pt}{-280pt}{bis0505}{84}{Banco de Dados\\(3-1-4)}
		
		\disciplina [blue!20]{336pt}{-280pt}{bis0505}{84}{Inteligência Artificial\\(3-1-4)}
		
		
		%Q9
		\node [draw, rotate=90, black,rectangle, minimum width=40pt, minimum height=10pt, rounded corners] at (-10pt,-320pt) {\footnotesize{9o. quad}};
		
		
		\disciplina [blue!20]{0pt}{-320pt}{bis0505}{84}{Redes de Computadores\\(3-1-4)}
		
		\disciplina [blue!20]{84pt}{-320pt}{bis0505}{84}{Sistemas Operacionais\\(3-1-4)}
		
		\disciplina [blue!20]{168pt}{-320pt}{bis0505}{84}{Linguagens Formais e Automata\\(3-1-4)}
		
		\disciplina [blue!20]{252pt}{-320pt}{bis0505}{84}{Engenharia de Software\\(4-0-4)}
		
		\disciplina [yellow!20]{336pt}{-320pt}{bis0505}{42}{Projeto Dirigido\\(0-2-10)}
		
		\disciplina [green!20]{378pt}{-320pt}{bis0505}{84}{Livre\\(4 créditos)}
		
		%Q10
		\node [draw, rotate=90, black,rectangle, minimum width=120pt, minimum height=10pt, rounded corners] at (-25pt,-400pt) {\footnotesize{4o. ANO}};
		
		\node [draw, rotate=90, black,rectangle, minimum width=40pt, minimum height=10pt, rounded corners] at (-10pt,-360pt) {\footnotesize{10o. quad}};
		
		
		\disciplina [blue!20]{0pt}{-360pt}{bis0505}{84}{Sistemas Distribuídos\\(3-1-4)}
		
		\disciplina [blue!20]{84pt}{-360pt}{bis0505}{84}{Compiladores\\(3-1-4)}
		
		\disciplina [blue!20]{168pt}{-360pt}{bis0505}{84}{Paradigmas de Programação\\(2-2-4)}
		
		\disciplina [blue!20]{252pt}{-360pt}{bis0505}{168}{Projeto de Graduação em Computação I\\(0-8-8)}
		
		\disciplina [red!20]{420pt}{-360pt}{bis0505}{168}{Opção Limitada\\(8 créditos)}
		
		%Q11
		\node [draw, rotate=90, black,rectangle, minimum width=40pt, minimum height=10pt, rounded corners] at (-10pt,-400pt) {\footnotesize{11o. quad}};
		
		
		\disciplina [blue!20]{0pt}{-400pt}{bis0505}{84}{Computação Gráfica\\(3-1-4)}
		
		\disciplina [blue!20]{84pt}{-400pt}{bis0505}{84}{Programação Matemática\\(3-1-4)}
		
		\disciplina [blue!20]{168pt}{-400pt}{bis0505}{168}{Projeto de Graduação em Computação II\\(0-8-8)}
		
		\disciplina [red!20]{336pt}{-400pt}{bis0505}{252}{Opção Limitada\\(12 créditos)}
		
		%Q12
		\node [draw, rotate=90, black,rectangle, minimum width=40pt, minimum height=10pt, rounded corners] at (-10pt,-440pt) {\footnotesize{12o. quad}};
		
		\disciplina [blue!20]{0pt}{-440pt}{bis0505}{84}{Segurança de Dados\\(3-1-4)}
		
		\disciplina [blue!20]{84pt}{-440pt}{bis0505}{168}{Projeto de Graduação em Computação III\\(0-8-8)}
		
		\disciplina [red!20]{252pt}{-440pt}{bis0505}{210}{Opção Limitada\\(10 créditos)}
		
		\disciplina [green!20]{462pt}{-440pt}{bis0505}{168}{Livre\\(8 créditos)}
		
		\draw [red,thick,dashed] (420pt,20pt) -- (420pt,-480pt); 
		\node [text=red] at (420pt,35pt) {20 créditos};
		
		
	\end{tikzpicture}
}




\subsection{Mapeamento de Habilidades/ Competências e Atividades Pedagógicas}
A organização curricular foi desenhada para atender aos requisitos estruturais e pedagógicos da UFABC, bem como às Diretrizes Curriculares Nacionais dos cursos de graduação em Computação (Res. CNE/CES no. 5, de 16/11/2016). A seguir, indicamos os componentes pedagógicos que contribuem para a formação e consolidação das habilidades e competências dos egressos. As atividades pedagógicas estão classificadas da seguinte forma:

\begin{itemize}
	\item \textcolor{red}{Disciplinas obrigatórias do BC\&T}
	\item \textcolor{blue}{Disciplinas obrigatórias do BCC}
	\item \textcolor{teal}{Disciplinas de opção limitada do BCC}
	\item \textcolor{violet}{Outras ações}
\end{itemize}


%\begin{tabular}{|ccc|}
%	\multicolumn{3}{l}{Identificar problemas que tenham solução algorítmica}\\
%	\hline
%	Bases Computacionais da Ciência  & Algoritmos e Estruturas de Dados I & Teoria dos Grafos\\
%	Processamento da Informação & Algoritmos e Estruturas de Dados II & Inteligência Artificial\\
%	Programação Estruturada & Análise de Algoritmos & Paradigmas de Programação\\
%	Matemática Discreta & Programação Orientada a Objetos & Programação Matemática\\
%	\hline
%\end{tabular}


\begin{longtable}{|p{.35\textwidth}p{.35\textwidth}p{.3\textwidth}|}
	\multicolumn{3}{l}{Identificar problemas que tenham solução algorítmica}\\
	\hline
	\textcolor{red}{Bases Computacionais da Ciência}  & \textcolor{blue}{Algoritmos e Estruturas de Dados I} & \textcolor{blue}{Teoria dos Grafos}\\
	\textcolor{red}{Processamento da Informação} & \textcolor{blue}{Algoritmos e Estruturas de Dados II} & \textcolor{blue}{Inteligência Artificial}\\
	\textcolor{blue}{Programação Estruturada} & \textcolor{blue}{Análise de Algoritmos} & \textcolor{blue}{Paradigmas de Programação}\\
	\textcolor{blue}{Matemática Discreta} & \textcolor{blue}{Programação Orientada a Objetos} & \textcolor{blue}{Programação Matemática}\\
	\hline
	
	\multicolumn{3}{l}{}\\
	
	\multicolumn{3}{l}{Conhecer os limites da computação}\\
	\hline
	\textcolor{blue}{Análise de Algoritmos} & \textcolor{blue}{Linguagens Formais e Automata} & \textcolor{blue}{Teoria dos Grafos}\\
	\hline
	
	\multicolumn{3}{l}{}\\
	\multicolumn{3}{l}{Resolver problemas usando ambientes de programação}\\
	\hline
	\textcolor{red}{Processamento da Informação} &  \textcolor{blue}{Teoria dos Grafos} & \textcolor{blue}{Engenharia de Software}\\
	\textcolor{blue}{Programação Estruturada} &  \textcolor{blue}{Inteligência Artificial} &  \textcolor{blue}{Programação Matemática}\\
	\textcolor{blue}{Algoritmos e Estruturas de Dados I} & \textcolor{blue}{Programação Orientada a Objetos} & \textcolor{blue}{Paradigmas de Programação}\\
	\textcolor{blue}{Algoritmos e Estruturas de Dados II} & \textcolor{blue}{Banco de Dados} &  \textcolor{blue}{Sistemas Digitais}\\
	\textcolor{blue}{Compiladores} && \\
	\hline
	
	\multicolumn{3}{l}{}\\
	\multicolumn{3}{l}{\makecell[l]{Tomar decisões e inovar, com base no conhecimento do funcionamento e das características técnicas de \\
			hardware e da infraestrutura de software dos sistemas de computação consciente dos aspectos éticos, \\
			legais e dos impactos ambientais decorrentes }}\\
	\hline
	\textcolor{red}{Ciência, Tecnologia e Sociedade} & \textcolor{blue}{Redes de Computadores} & \textcolor{blue}{Segurança de Dados}\\
	\textcolor{red}{Comunicação e Redes} &  \textcolor{blue}{Sistemas Operacionais} & \textcolor{blue}{Banco de Dados}\\
	\textcolor{blue}{Arquitetura de Computadores} & \textcolor{blue}{Sistemas Distribuídos} & \textcolor{blue}{Engenharia de Software}\\
	\textcolor{blue}{Sistemas Digitais} &  \textcolor{blue}{Computadores, Ética e Sociedade} & \\
	\hline
	
	\multicolumn{3}{l}{}\\
	\multicolumn{3}{l}{Compreender e explicar as dimensões quantitativas de um problema}\\
	\hline
	\textcolor{red}{Natureza da Informação} & \textcolor{red}{Geometria Analítica} & \textcolor{blue}{Linguagens Formais e Automata}\\
	\textcolor{red}{Intr. à Probabilidade e Estatística} &  \textcolor{blue}{Álgebra Linear} & \textcolor{blue}{Programação Matemática}\\
	\textcolor{red}{Funções de Uma Variável} & \textcolor{blue}{Matemática Discreta} &  \textcolor{blue}{Engenharia de Software}\\
	\textcolor{red}{Funções de Várias Variáveis} &  \textcolor{blue}{Análise de Algoritmos} &\\
	\hline
	
	\multicolumn{3}{l}{}\\
	\multicolumn{3}{l}{\makecell[l]{Gerir a sua própria aprendizagem e desenvolvimento, incluindo a gestão de tempo e competências \\organizacionais}}\\
	\hline
	\textcolor{blue}{Proj. de Grad. Computaçao I} & \textcolor{blue}{Proj. de Grad. Computaçao II} & \textcolor{blue}{Proj. de Grad. Computaçao III}\\
	\hline
	
	\multicolumn{3}{l}{}\\
	\multicolumn{3}{l}{\makecell[l]{Preparar e apresentar seus trabalhos e problemas técnicos e suas soluções para audiências diversas, em \\formatos apropriados (oral e escrito)}}\\
	\hline
	\textcolor{red}{Projeto Dirigido} & \textcolor{blue}{Proj. de Grad. Computaçao I} & \textcolor{blue}{Proj. de Grad. Computaçao III}\\
	\textcolor{blue}{Engenharia de Software} & \textcolor{blue}{Proj. de Grad. Computaçao II} &\\
	\hline
	
	\multicolumn{3}{l}{}\\
	\multicolumn{3}{l}{Avaliar criticamente projetos de sistemas de computação}\\
	\hline
	\textcolor{red}{Ciência, Tecnologia e Sociedade} & \textcolor{blue}{Segurança de Dados} &\textcolor{blue}{Análise de Algoritmos}\\
	\textcolor{blue}{Engenharia de Software} & \textcolor{blue}{Computadores, Ética e Sociedade}& \textcolor{blue}{Sistemas Distribuídos}\\
	\textcolor{blue}{Redes de Computadores} & &\\
	\hline
	
	\multicolumn{3}{l}{}\\
	\multicolumn{3}{l}{Adequar-se rapidamente às mudanças tecnológicas e aos novos ambientes de trabalho}\\
	\hline
	\textcolor{blue}{Computadores, Ética e Sociedade} & & \\
	\hline
	
	\multicolumn{3}{l}{}\\
	\multicolumn{3}{l}{Ler textos técnicos na língua inglesa}\\
	\hline
	\textcolor{red}{Projeto Dirigido} & \textcolor{blue}{Proj. de Grad. Computaçao II} & \textcolor{blue}{Proj. de Grad. Computaçao III}\\
	\textcolor{blue}{Proj. de Grad. Computaçao I} && \\
	\hline
	
	\multicolumn{3}{l}{}\\
	\multicolumn{3}{l}{Empreender e exercer liderança, coordenação e supervisão na sua área de atuação profissional}\\
	\hline
	&& \\
	\hline
	
	\multicolumn{3}{l}{}\\
	\multicolumn{3}{l}{Ser capaz de realizar trabalho cooperativo e entender os benefícios que este pode produzir}\\
	\hline
	\textcolor{red}{Ciência, Tecnologia e Sociedade} & \textcolor{blue}{Computadores, Ética e Sociedade} & \textcolor{blue}{Engenharia de Software}\\
	\hline
	
	\multicolumn{3}{l}{}\\
	\multicolumn{3}{l}{\makecell[l]{Compreender os fatos essenciais, os conceitos, os princípios e as teorias relacionadas à Ciência da \\Computação para o desenvolvimento de software e hardware e suas aplicações}}\\
	\hline
	\textcolor{red}{Bases Computacionais da Ciência} & \textcolor{blue}{Paradigmas de Programação} & \textcolor{blue}{Teoria dos Grafos}\\
	\textcolor{red}{Processamento da Informação} &  \textcolor{blue}{Algoritmos e Estruturas de Dados I} & \textcolor{blue}{Programação Matemática}\\
	\textcolor{blue}{Programação Estruturada} & \textcolor{blue}{Algoritmos e Estruturas de Dados II} & \textcolor{blue}{Circuitos Digitais}\\
	\textcolor{blue}{Análise de Algoritmos} &\textcolor{blue}{Linguagens Formais e Automata} & \textcolor{blue}{Sistemas Digitais}\\
	\textcolor{blue}{Arquitetura de Computadores} & \textcolor{blue}{Matemática Discreta} & \textcolor{blue}{Sistemas Operacionais}\\
	\textcolor{blue}{Lógica Básica} &   &   \\
	\hline
	
	\multicolumn{3}{l}{}\\
	\multicolumn{3}{l}{\makecell[l]{Reconhecer a importância do pensamento computacional no cotidiano e sua aplicação em circunstâncias \\apropriadas e em domínios diversos }}\\
	\hline
	\textcolor{red}{Comunicação e Redes} & \textcolor{red}{Bases Computacionais da Ciência} & \textcolor{blue}{Lógica Básica}\\
	\textcolor{red}{Processamento da Informação} & \textcolor{blue}{Algoritmos e Estruturas de Dados I}& \textcolor{blue}{Matemática Discreta}\\
	\textcolor{red}{Ciência, Tecnologia e Sociedade} & \textcolor{blue}{Algoritmos e Estruturas de Dados II}&  \textcolor{blue}{Teoria dos Grafos}\\
	\textcolor{blue}{Programação Estruturada} &\textcolor{blue}{Computadores, Ética e Sociedade} & \textcolor{blue}{Programação Matemática}\\
	\hline
	
	\multicolumn{3}{l}{}\\
	\multicolumn{3}{l}{\makecell[l]{Identificar e gerenciar os riscos que podem estar envolvidos na operação de equipamentos de computação \\(incluindo os aspectos de dependabilidade e segurança)}}\\
	\hline
	\textcolor{blue}{Segurança de Dados} & \textcolor{blue}{Computadores, Ética e Sociedade} &  \textcolor{blue}{Circuitos Digitais}\\
	\textcolor{blue}{Banco de Dados} & \textcolor{blue}{Redes de Computadores} & \\
	\hline
	
	\multicolumn{3}{l}{}\\
	\multicolumn{3}{l}{\makecell[l]{Identificar e analisar requisitos e especificações para problemas específicos e planejar estratégias para \\suas soluções}}\\
	\hline
	\textcolor{blue}{Engenharia de Software} & \textcolor{blue}{Arquitetura de Computadores} & \textcolor{blue}{Circuitos Digitais}\\
	\textcolor{blue}{Análise de Algoritmos} & \textcolor{blue}{Banco de Dados} & \textcolor{blue}{Sistemas Digitais}\\
	\hline
	
	\multicolumn{3}{l}{}\\
	\multicolumn{3}{l}{\makecell[l]{Especificar, projetar, implementar, manter e avaliar sistemas de computação, empregando teorias, práticas e \\ferramentas adequadas}}\\
	\hline
	\textcolor{blue}{Sistemas Operacionais} &  \textcolor{blue}{Arquitetura de Computadores} &  \textcolor{blue}{Análise de Algoritmos}\\
	\textcolor{blue}{Banco de Dados} & \textcolor{blue}{Redes de Computadores} &  \textcolor{blue}{Compiladores}\\
	\textcolor{blue}{Engenharia de Software} & \textcolor{blue}{Sistemas Distribuídos} & \textcolor{blue}{Sistemas Digitais}\\
	\hline
	
	\multicolumn{3}{l}{}\\
	\multicolumn{3}{l}{\makecell[l]{Conceber soluções computacionais a partir de decisões visando o equilíbrio de todos os fatores envolvidos}}\\
	\hline
	\textcolor{red}{Processamento da Informação} & \textcolor{blue}{Algoritmos e Estruturas de Dados I}&  \textcolor{blue}{Segurança de Dados}\\
	\textcolor{red}{Ciência, Tecnologia e Sociedade} & \textcolor{blue}{Algoritmos e Estruturas de Dados II}& \textcolor{blue}{Engenharia de Software}\\
	\textcolor{blue}{Programação Estruturada} & \textcolor{blue}{Computadores, Ética e Sociedade}& \textcolor{blue}{Compiladores}\\
	\textcolor{blue}{Inteligência Artificial} & \textcolor{blue}{Paradigmas de Programação}& \\
	\hline
	
	\multicolumn{3}{l}{}\\
	\multicolumn{3}{l}{\makecell[l]{Empregar metodologias que visem garantir critérios de qualidade ao longo de todas as etapas de \\desenvolvimento de uma solução computacional}}\\
	\hline
	\textcolor{blue}{Engenharia de Software} & & \\
	\hline
	
	\multicolumn{3}{l}{}\\
	\multicolumn{3}{l}{\makecell[l]{Analisar quanto um sistema baseado em computadores atende os critérios definidos para seu uso corrente e \\futuro (adequabilidade)}}\\
	\hline
	\textcolor{blue}{Engenharia de Software} & \textcolor{blue}{Análise de Algoritmos} & \textcolor{blue}{Banco de Dados}\\
	\hline
	
	\multicolumn{3}{l}{}\\
	\multicolumn{3}{l}{\makecell[l]{Gerenciar projetos de desenvolvimento de sistemas computacionais}}\\
	\hline
	\textcolor{blue}{Compiladores} & \textcolor{blue}{Sistemas Operacionais}&  \textcolor{blue}{Banco de Dados}\\
	\textcolor{blue}{Engenharia de Software} &  \textcolor{teal}{Gestão de Projetos de Software} & \\
	\hline
	
	\multicolumn{3}{l}{}\\
	\multicolumn{3}{l}{\makecell[l]{Aplicar temas e princípios recorrentes, como abstração, complexidade, princípio de localidade de referência \\(caching), compartilhamento de recursos, segurança, concorrência, evolução de sistemas, entre outros, e \\reconhecer que esses temas e princípios são fundamentais à área de Ciência da Computação}}\\
	\hline
	\textcolor{blue}{Análise de Algoritmos} & \textcolor{blue}{Programação Orientada a Objetos} & \textcolor{blue}{Banco de Dados}\\
	\textcolor{blue}{Segurança de Dados} &  \textcolor{blue}{Linguagens Formais e Automata} & \textcolor{blue}{Sistemas Distribuídos}\\
	\textcolor{blue}{Redes de Computadores} & \textcolor{blue}{Arquitetura de Computadores} & \textcolor{blue}{Sistemas Operacionais}\\
	\textcolor{blue}{Inteligência Artificial} & \textcolor{blue}{Engenharia de Software} & \textcolor{blue}{Sistemas Digitais}\\
	\hline
	
	
	\multicolumn{3}{l}{}\\
	\multicolumn{3}{l}{\makecell[l]{Escolher e aplicar boas práticas e técnicas que conduzam ao raciocínio rigoroso no planejamento, na execução \\e no acompanhamento, na medição e gerenciamento geral da qualidade de sistemas computacionais}}\\
	\hline
	\textcolor{blue}{Análise de Algoritmos} & \textcolor{blue}{Computadores, Ética e Sociedade} & \textcolor{blue}{Banco de Dados}\\
	\textcolor{blue}{Engenharia de Software} & \textcolor{teal}{Sistemas de Informação} & \textcolor{blue}{Sistemas Digitais}\\
	\textcolor{blue}{Compiladores} & & \\
	\hline
	
	
	\multicolumn{3}{l}{}\\
	\multicolumn{3}{l}{\makecell[l]{Aplicar os princípios de gerência, organização e recuperação da informação de vários tipos, incluindo texto, \\imagem, som e vídeo}}\\
	\hline
	\textcolor{blue}{Banco de Dados} & \textcolor{blue}{Algoritmos e Estruturas de Dados I} & \textcolor{blue}{Sistemas Distribuídos}\\
	\textcolor{blue}{Computação Gráfica} & \textcolor{blue}{Algoritmos e Estruturas de Dados II} & \textcolor{teal}{Proc. de Sinais Neurais} \\
	\textcolor{blue}{Redes de Computadores} & \textcolor{teal}{Processamento Digital de Imagens} & \\
	\hline
	
	\multicolumn{3}{l}{}\\
	\multicolumn{3}{l}{\makecell[l]{Aplicar os princípios de interação humano-computador para avaliar e construir uma grande variedade de \\produtos, incluindo interface de usuário, páginas WEB, sistemas multimídia e sistemas móveis}}\\
	\hline
	\textcolor{blue}{Computação Gráfica} & \textcolor{teal}{Interação Humano-Computador} & \textcolor{teal}{Sistemas Multimidia}\\
	\textcolor{teal}{Programação para Web} &  \textcolor{teal}{Prog. Av. de Dispositivos Móveis} & \textcolor{teal}{Visão Computacional}\\
	\textcolor{teal}{Sistemas Inteligentes} & & \\
	\hline
	
	\end {longtable}
