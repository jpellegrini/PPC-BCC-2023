\section{Organização curricular}
\label{sec:organizacao_curricular}

\subsection{Fundamentação legal}
\label{subsec:fund_legal}

A seguir são elencados os documentos legais externos (Diretrizes
Curriculares Nacionais, Leis, Decretos, Resoluções, Pareceres,
Portarias, Normativas etc.), de ordem federal, bem como os internos
(Projeto Pedagógico, Plano de Desenvolvimento Institucional) que
fundamentam a estrutura curricular do curso de Bacharelado em Ciência
da Computação da UFABC:

\begin{itemize}
    
    \item BRASIL. Presidência da República. Casa Civil. Subchefia para Assuntos
    Jurídicos. Lei 9.394, de 20 de dezembro de 1996. Estabelece as diretrizes e
    bases da educação nacional.
    Disponível em: \url{https://www.planalto.gov.br/ccivil_03/Leis/L9394.htm}. 
    Acesso em: 20 de janeiro de 2023.
    
    \item BRASIL. Presidência da República. Casa Civil. Subchefia para Assuntos
    Jurídicos. Lei 10.639, de 9 de janeiro de 2003. Altera a Lei 9.394, de 20
    de dezembro de 1996, que estabelece as diretrizes e bases da educação
    nacional, para incluir no currículo oficial da Rede de Ensino a
    obrigatoriedade da temática ``História e Cultura Afro-Brasileira'', e dá
    outras providências. 
    Disponível em: \url{http://www.planalto.gov.br/ccivil_03/leis/2003/l10.639.htm}.
    Acesso em: 20 de janeiro de 2023.
    
    \item BRASIL. Presidência da República. Casa Civil. Subchefia para Assuntos
    Jurídicos. Lei 11.645, de 10 de março de 2008. Altera a Lei 9.394, de 20 de
    dezembro de 1996, modificada pela Lei 10.639, de 9 de janeiro de 2003, que
    estabelece as diretrizes e bases da educação nacional, para incluir no
    currículo oficial da rede de ensino a obrigatoriedade da temática
    ``História e Cultura Afro-Brasileira e Indígena''. 
    Disponível em: \url{http://www.planalto.gov.br/ccivil_03/_ato2007-2010/2008/lei/l11645.htm}.
    Acesso em: 20 de janeiro de 2023.
    
    \item BRASIL. Ministério da Educação. Conselho Nacional de Educação.
    Conselho Pleno. Resolução 1, de 17 de junho de 2004. Institui Diretrizes
    Curriculares Nacionais para a Educação das Relações Étnico Raciais e para o
    Ensino de História e Cultura Afro-Brasileira e Africana. 
    Disponível em: \url{http://portal.mec.gov.br/cne/arquivos/pdf/res012004.pdf}. 
    Acesso em: 20 de janeiro de 2023.
    
    \item BRASIL. Ministério da Educação. Secretaria da Educação Superior.
    Referenciais Orientadores para os Bacharelados Interdisciplinares e
    Similares. 2010.
    Disponível em: \url{http://www.ufabc.edu.br/images/stories/comunicacao/bacharelados-interdisciplinares_referenciais-orientadores-novembro_2010-brasilia.pdf}.
    Acesso em: 20 de janeiro de 2023.
    
    \item BRASIL. Ministério da Educação. Conselho Nacional de Educação.
    Referenciais orientadores para os Bacharelados Interdisciplinares e
    Similares das Universidades Federais. Câmara de Educação Superior. Parecer
    CNE/CES 266, de 5 de julho de 2011.
    Disponível em:
    \url{http://portal.mec.gov.br/index.php?option=com_content&view=article&id=16418&Itemid=866}
    Acesso em: 20 de janeiro de 2023.
    
    \item BRASIL. Presidência da República. Casa Civil. Subchefia para Assuntos
    Jurídicos. Decreto 5.626, de 22 de dezembro de 2005. Regulamenta a Lei
    10.436, de 24 de abril de 2002, que dispõe sobre a Língua Brasileira de
    Sinais - Libras, e o art. 18 da Lei 10.098, de 19 de dezembro de 2000.
    Disponível em: \url{https://www.planalto.gov.br/ccivil_03/_Ato2004-2006/2005/Decreto/D5626.htm}.
    Acesso em: 20 de janeiro de 2023.
    
    \item  BRASIL. Presidência da República. Casa Civil. Subchefia para
    Assuntos Jurídicos. Lei 9.795, de 27 de abril de 1999. Dispõe sobre a
    educação ambiental, institui a Política Nacional de Educação Ambiental e dá
    outras providências. 
    Disponível em: \url{http://www.planalto.gov.br/ccivil_03/leis/l9795.htm}.
    Acesso em: 20 de janeiro de 2023.
    
    \item BRASIL. Presidência da República. Casa Civil. Subchefia para Assuntos
    Jurídicos. Decreto 4.281, de 25 de junho de 2002. Regulamenta a Lei 9.795,
    de 27 de abril de 1999, que institui a Política Nacional de Educação
    Ambiental, e dá outras providências. 
    Disponível em: \url{http://www.planalto.gov.br/ccivil_03/decreto/2002/D4281.htm}.
    Acesso em: 20 de janeiro de 2023.
    
    \item BRASIL. Ministério da Educação. Conselho Nacional de Educação. Câmara
    de Educação Superior. Resolução 2, de 18 de junho de 2007. Dispõe sobre
    carga horária mínima e procedimentos relativos à integralização e duração
    dos cursos de graduação, bacharelados, na modalidade presencial. 
    Disponível em: \url{http://portal.mec.gov.br/cne/arquivos/pdf/2007/rces002_07.pdf}.
    Acesso em: 20 de janeiro de 2023.
    
    \item BRASIL. Ministério da Educação. Gabinete do Ministro. Portaria
    Normativa 40, de 12 de dezembro de 2007. Institui o e-MEC, sistema
    eletrônico de fluxo de trabalho e gerenciamento de informações relativas
    aos processos de regulação, avaliação e supervisão da educação superior no
    sistema federal de educação, e o Cadastro e-MEC de Instituições e Cursos
    Superiores e consolida disposições sobre indicadores de qualidade, banco de
    avaliadores (Basis) e o Exame Nacional de Desempenho de Estudantes (ENADE)
    e outras disposições.
    Disponível em: \url{http://portal.mec.gov.br/cne/arquivos/pdf/2007/port40_07.pdf}.
    Acesso em: 20 de janeiro de 2023.
    
    \item BRASIL. Comissão Nacional de Avaliação da Educação Superior.
    Resolução 1, de 17 de junho de 2010. Normatiza o Núcleo Docente
    Estruturante e dá outras providências. 
    Disponível em: \url{http://portal.mec.gov.br/index.php?option=com_docman&task=doc_download&gid=6885&Itemid}.
    Acesso em: 20 de janeiro de 2023.
    
    %\item BRASIL. Presidência da República. Casa Civil. Subchefia para Assuntos
    %Jurídicos. Decreto 5.622. Regulamenta o art. 80 da Lei 9.394, de 20 de
    %dezembro de 1996, que estabelece as diretrizes e bases da educação
    %nacional. 
    %Disponível em: \url{http://www.planalto.gov.br/ccivil_03/_ato2004-2006/2005/Decreto/D5622compilado.htm}.
    %%% REVOGADO
    \item BRASIL. Presidência da República. Casa Civil. Subchefia para Assuntos
    Jurídicos. Decreto 9.057. Regulamenta o art. 80 da Lei 9.394, de 20 de
    dezembro de 1996, que estabelece as diretrizes e bases da educação
    nacional. 
    Disponível em: \url{https://www.planalto.gov.br/ccivil_03/_Ato2015-2018/2017/Decreto/D9057.htm}.
    Acesso em: 20 de janeiro de 2023.
    
    \item BRASIL. Ministério da Educação. Portaria nº 4.059, de 10 de dezembro
    de 2004. Regulamentação de disciplinas na modalidade semipresencial.
    Disponível em: \url{http://portal.mec.gov.br/sesu/arquivos/pdf/nova/acs_portaria4059.pdf}.
    Acesso em: 20 de janeiro de 2023.
    
    \item BRASIL. Ministério da Educação. Conselho Nacional de Educação. Câmara
    de Educação Superior Parecer CNE/CES nº 136/2012, aprovado em 8 de março de
    2012 - Diretrizes Curriculares Nacionais para os cursos de graduação em
    Computação. 
    Publicado no DOU no 134, de 12 de julho de 2012.
    Disponível em: \url{http://portal.mec.gov.br/component/content/article?id=12991}.
    Acesso em: 20 de janeiro de 2023.
    
    \item BRASIL. Ministério da Educação. Conselho Nacional de Educação.
    Conselho Pleno. Parecer CNE/CP nº 003, de 10 de março de 2004.
    Disponível em: \url{http://portal.mec.gov.br/cne/arquivos/pdf/003.pdf}. 
    Acesso em: 20 de janeiro de 2023.
    
    \item BRASIL. Ministério da Educação. Conselho Nacional de Educação.
    Conselho Pleno. Resolução nº 1, de 30 de maio de 2012. Estabelece
    Diretrizes Nacionais para a Educação em Direitos Humanos. 
    Disponível em: \url{http://portal.mec.gov.br/index.php?option=com_docman&view=download&alias=10889-
    rcp001-12&category_slug=maio-2012-pdf&Itemid=30192}. 
    Acesso em: 20 de janeiro de 2023.
    
    \item BRASIL. Presidência da República. Casa Civil. Subchefia para Assuntos
    Jurídicos. Lei 12.764, de 27 de dezembro de 2012. Institui a Política
    Nacional de Proteção dos Direitos da Pessoa com Transtorno do Espectro
    Autista; e altera o § 3º do art. 98 da Lei 8.112, de 11 de dezembro de
    1990. 
    Disponível em: \url{http://www.planalto.gov.br/ccivil_03/_ato2011-2014/2012/lei/l12764.htm}.
    Acesso em: 20 de janeiro de 2023.
    
    \item FUNDAÇÃO UNIVERSIDADE FEDERAL DO ABC. Projeto Pedagógico Institucional. Santo
    André, 2017. Disponível em:
    \url{https://www.ufabc.edu.br/images/imagens_a_ufabc/projeto-pedagogico-institucional.pdf}.
    Acesso em: 20 de janeiro de 2023.
    
    \item FUNDAÇÃO UNIVERSIDADE FEDERAL DO ABC. Plano de Desenvolvimento
    Institucional. Santo André, 2013. Disponível em:
    \url{https://www.ufabc.edu.br/a-ufabc/documentos/plano-de-desenvolvimento-institucional-pdi}
    Acesso em: 20 de janeiro de 2023.
    
    \item BRASIL. Ministério da Educação. Conselho Nacional de Educação. Câmara
    de Educação Superior. Parecer CNE/CES nº 136/2012, aprovado em 8 de março
    de 2012 - Diretrizes Curriculares Nacionais para os cursos de graduação em
    Computação. 
    Disponível em: \url{http://portal.mec.gov.br/index.php?option=com_docman&view=download&alias=11205-pces136-11-pdf&category_slug=julho-2012-pdf&Itemid=30192}.
    Acesso em: 20 de janeiro de 2023.
    
    \item BRASIL. Ministério da Educação. Conselho Nacional de Educação. Câmara
    de Educação Superior. Resolução CNE/CES 5, de 16 de novembro de 2016 -
    Institui as Diretrizes Curriculares Nacionais para os cursos de graduação
    na área da Computação, abrangendo os cursos de bacharelado em Ciência da
    Computação, em Sistemas de Informação, em Engenharia de Computação, em
    Engenharia de Software e de licenciatura em Computação, e dá outras
    providências. Disponível em:
    \url{http://portal.mec.gov.br/index.php?option=com_docman&view=download&alias=52101-rces005-16-pdf&category_slug=novembro-2016-pdf&Itemid=30192}.
    Acesso em: 20 de janeiro de 2023.
    
\end{itemize}


\subsection{Regime de ensino}

\subsubsection{Estrutura curricular}

A estrutura do BCC é composta por três grupos de componentes curriculares, que
totalizam \red{3.433} horas: ($i$) disciplinas, ($ii$) atividades de extensão e
cultura e ($iii$) atividades complementares.
As disciplinas, por sua vez, estão divididas em três categorias: obrigatórias
(que devem necessariamente ser cursadas com aprovação para a integralização do
curso), de opção limitada (presentes em um grupo selecionado de disciplinas,
que permitem ao aluno aprofundar seus conhecimentos em determinadas áreas do
conhecimento, fazendo relações interdisciplinares com os conhecimentos
ofertados pelas disciplinas obrigatórias) e livres (quaisquer disciplinas
oferecidas pela UFABC, ou por outra IES reconhecida pelo MEC, de curso de
graduação ou de pós-graduação, necessárias para completar o número total de
créditos exigidos para a integralização do curso).
As disciplinas de opção limitada e livres, portanto, podem ser selecionadas
pelos estudantes, oferecendo autonomia para projetarem esta carga horária de
acordo com seus interesses e aptidões.
As disciplinas obrigatórias são apresentadas na
Seção~\ref{sec:matriz_curricular} e as disciplinas de opção limitada devem ser
selecionadas dentre aquelas constantes no Catálogo de Disciplinas de Opção
Limitada do BCC.

As componentes curriculares são contabilizadas na forma de créditos, sendo que
cada crédito equivale a 12 horas de aula e a hora-aula é de 60 minutos.
A distribuição da quantidade de créditos e da carga horária a serem cumpridos
em cada uma das categorias de disciplinas para a obtenção do grau de Bacharel
em Ciência da Computação é dada na Tabela~\ref{tab:carga_horaria}.
Não contabiliza-se, nessa tabela, o número de horas de estudo individual
extraclasse necessários para o bom acompanhamento das atividades.

\begin{table}[h!]
    \centering
    \caption{Distribuição de créditos e carga horária a serem cumpridos no BCC,
    descontando-se as horas de estudo individual extraclasse.}
    \label{tab:carga_horaria}
    \begin{tabular}{|l|c|c|}
        \hline
        \textbf{Categoria}                        & \textbf{Créditos} & \textbf{Carga horária (horas)} \\
        \hline\hline
        Disciplinas obrigatórias do BC\&T         & 84                & 1.008 \\
        \hline
        Disciplinas obrigatórias do BCC           & 108               & 1.296 \\
        \hline
        Disciplinas de opção limitada do BCC      & 24                & 288 \\
        \hline
        Disciplinas livres                        & 16                 & 192 \\
        \hline
        \textit{Total de disciplinas}             & \textit{232}       & \textit{2.784}\\
        \hline\hline
        \multicolumn{2}{|l|}{Trabalho de Conclusão de Curso}       & 288 \\
        \multicolumn{2}{|l|}{Atividades complementares}            & 18 \\
        \hline
        \multicolumn{2}{|l|}{Atividades de extensão e cultura}      & 343 \\
        \hline
        \multicolumn{2}{|l|}{\textit{Carga horária total}}               & \textbf{3.433}\\
        \hline
    \end{tabular}
\end{table}




\subsubsection{Interdisciplinaridade}

O BC\&T é a base da matriz curricular do BCC, de maneira que a formação
proposta proporciona interdisciplinaridade e flexibilidade curricular.
As disciplinas obrigatórias do BC\&T organizam o conhecimento em eixos
(Energia, Processos de Transformação, Representação e Simulação, Informação e
Comunicação, Estrutura da Matéria e Humanidades), visando despertar o interesse
dos alunos para a investigação de cunho interdisciplinar.
Os cursos de graduação da UFABC estão estruturados em um sistema de créditos
que permite diferentes organizações curriculares, de acordo com os interesses e
aptidões dos alunos.
Através das disciplinas livres, os alunos poderão se aprofundar em quaisquer
áreas do conhecimento, partindo para especificidades curriculares de cursos de
formação profissional ou explorando a interdisciplinaridade e estabelecendo um
currículo individual de formação.

É importante destacar que a interdisciplinaridade do presente projeto
pedagógico e a possibilidade de escolher disciplinas livres permitem que o
discente formado no BCC da UFABC esteja alinhado com as seguintes diretrizes
legais:
\begin{itemize}
    \item Decreto 5.626, de 22 de dezembro de 2005: a disciplina de LIBRAS,
    cuja ementa faz parte do rol de disciplinas dos cursos de licenciatura da
    UFABC, pode ser cursada pelos alunos do BCC.
    
    \item Lei nº 11.64, sobre a obrigatoriedade da temática ``História e
    Cultura Afro-Brasileira e Indígena'', e Resolução 01/2004, de 17 de junho
    de 2004: o aluno do BCC pode escolher cursar disciplinas livres que fazem
    parte do rol de disciplinas da UFABC e que envolvem a temática da História
    e Cultura Afro-Brasileira e Indígenas.
    
    \item Política Nacional de Educação Ambiental (Lei nº 9795/1999 e decreto
    4.281, de 25 de junho de 2002): muitas disciplinas livres oferecidas no rol
    de disciplinas de engenharia ambiental podem ser cursadas pelos alunos do
    BCC, permitindo assim a integração desse projeto pedagógico com a educação
    ambiental.
\end{itemize}



\subsection{Estratégias pedagógicas}

A UFABC foi concebida definindo a interdisciplinaridade como uma referência
pedagógica.
Sendo assim, o desenvolvimento do perfil do egresso é trabalhado com uma
formação interdisciplinar com alto grau de liberdade para incorporar
componentes curriculares à sua formação.
Além de cobrir os assuntos pertinentes à formação definida pelas DCNs de
Computação, esse modelo possibilita que o aluno desenvolva competências em
outras áreas de seu próprio interesse.

Seguindo a recomendação da matriz curricular, os primeiros quadrimestres
letivos de curso são preenchidos por disciplinas do BC\&T, onde o(a) aluno(a)
tem contato com várias áreas da Ciência, fortalecendo sua base científica e
humanística, além de experimentar os primeiros contatos com disciplinas da área
de Computação (Bases Computacionais da Ciência, Natureza da Informação,
Comunicação e Redes, Processamento da Informação).
Aos poucos, o(a) aluno(a) vai encontrar janelas de horários para incluir
disciplinas específicas de Computação enquanto finaliza sua formação no BC\&T.
O projeto pedagógico ainda prevê aproximadamente \red`{15\%} de carga horária exclusivamente 
em disciplinas livres e de opção limitada, em que o(a) aluno(a) poderá escolher os componentes
curriculares que completarão a sua formação.

Na UFABC as disciplinas não possuem pré-requisitos entre si.
Mesmo assim, a estrutura da matriz curricular sugere uma composição que
favorece o desenvolvimento contínuo das competências e habilidades do egresso
durante o desenvolvimento do curso, concentrando disciplinas que abordam temas
avançados e específicos no final do curso e disciplinas fundamentais em seu
início.

O uso de Tecnologias de Informação e Comunicação (TICs) é uma realidade próxima
dos estudantes na UFABC.
Muitas disciplinas utilizam ambientes virtuais de aprendizagem (AVAs) para
gestão de conteúdo em disciplinas presenciais e semipresenciais.
Durante a pandemia de COVID-19, algumas disciplinas foram também ofertadas na
modalidade online com sucesso.
Todos os cursos possuem páginas específicas em que seus conteúdos e documentos
ficam acessíveis à comunidade (projeto pedagógico, informações gerais,
documentos, links para outras páginas de recursos, etc.).
Uma importante parcela da carga horária total é trabalhada em aulas práticas,
ofertadas em laboratórios de informática com computadores ou laboratórios de
{\it hardware} com dispositivos eletrônicos.

Em termos de acessibilidade, a UFABC tem se preocupa com a garantia de acesso
às pessoas com deficiência e/ou com mobilidade reduzida. 
Seguindo as determinações do Decreto nº 5.296/2004 47 e da Lei 10.098/2000 48,
os dois campi da UFABC possuem acessibilidade arquitetônica, garantindo o uso
autônomo dos espaços por pessoas com deficiência e/ou com mobilidade reduzida.
Através do Núcleo de Acessibilidade da Pró-Reitoria de Assuntos Comunitários e
Políticas Afirmativas (PROAP), a UFABC tem procurado a excelência no quesito
inclusão.
Nesse sentido, dentre as disciplinas oferecidas pela UFABC, destacamos o
oferecimento da disciplina Libras.

Políticas de educação ambiental e de educação em direitos humanos são tratadas
por algumas disciplinas ofertadas pela UFABC.
Dentre as relacionadas à educação ambiental, citamos: Educação Ambiental,
Economia e Meio Ambiente; Economia, Sociedade e Meio Ambiente; e Energia, Meio
Ambiente e Sociedade.
Dentre as relacionadas à educação em direitos humanos, citamos: Regime
Internacional dos Direitos Humanos e a Atuação Brasileira; Movimentos
Sindicais, Sociais e Culturais; Diversidade Cultural, Conhecimento Local e
Políticas Públicas; Identidade e Cultura; Cidadania, Direitos e Desigualdades;
Cultura, Identidade e Política na América Latina; e Trajetórias Internacionais
do Continente Africano.



\subsection{Matriz Curricular Recomendada}
\label{sec:matriz_curricular}

Os componentes curriculares na UFABC são oferecidos em ciclos quadrimestrais.
A carga horária das disciplinas são distribuídas entre créditos teóricos (T),
práticos (P), de caráter extensionista ou cultural (E) e de dedicação a estudos
individuais extraclasses (I).
Considera-se, dessa forma, a quantidade de créditos e de horas de trabalho de
cada disciplina apresentada por seu T-P-E-I.
Sugere-se que o aluno pondere o número de horas de estudo individual
extraclasse nos momentos de matrícula, para que considere sempre as horas
necessárias de dedicação às atividades de cada disciplina. 
Para o cômputo dos créditos totais, no entanto, são considerados apenas os
especificados em T e P e, no caso de E, serão contabilizados para compor a
carga horária extensionista e cultural.


Os estudantes da UFABC têm liberdade para organização da própria matriz
curricular, não havendo pré-requisitos entre as disciplinas.
No entanto, é importante observar os requisitos recomendados e o encadeamento
das disciplinas de modo a permitir o melhor aproveitamento dos conteúdos.
Com o objetivo de orientar e auxiliar o aluno a compreender as possibilidades desse
currículo, a Figura~\ref{fig:matriz_curricular} apresenta a Matriz Curricular
Sugerida para o discente que pretende integralizar ambos BC\&T e o BCC em cinco
anos e a Figura~\ref{fig:recomendacoes} apresenta graficamente as recomendações
entre as disciplinas obrigatórias do BCC e do BC\&T.
A descrição completa das ementas das disciplinas obrigatórias encontra-se na
Seção~\ref{sec:disciplinas_obrigatorias}.

\begin{figure}
    \centering
    \resizebox{\textwidth}{!}{
\documentclass{standalone}
\usepackage{amsmath, amsthm, amsfonts, amssymb}
\usepackage{tikz}
\usetikzlibrary{shapes,snakes,positioning,calc}
\usepackage{hyperref}

\newcommand{\disciplina}[6][nyellow]{
\node [draw=black, fill=#1, very thick, rectangle, rounded corners, inner
sep=2pt, inner ysep=2pt, text width=#5pt, minimum height=40pt,
text centered, anchor=west] (#4) at (#2,#3) {
   \begin{minipage}{#5pt}
   \linespread{1.0}\selectfont
       \centering
       \footnotesize{#6}
   \end{minipage}
};
}


\usepackage{xcolor}
\definecolor{nred}{rgb}{0.88, 0.28, 0.33}
\definecolor{nblue}{rgb}{0.34, 0.74, 0.96}
\definecolor{nyellow}{rgb}{0.8, 0.86, 0.38}
\definecolor{npurple}{rgb}{0.25, 0.34, 0.93}
\definecolor{ngreen}{rgb}{0.45, 0.95, 0.66}


\begin{document}

\begin{tikzpicture}

    \def\sz{22pt}
    \def\doisc{38} % 2*\sz - 6
    \def\tresc{60} % 3*\sz - 6
    \def\quatc{82} % 4*\sz - 6
    \def\cincc{104} % 5*\sz - 6
    \def\seisc{126} % 6*\sz - 6
    
    %Q1
    \node [draw, rotate=90, black,rectangle, minimum width=120pt, minimum
    height=10pt, rounded corners] at (-25pt,-40pt) {\footnotesize{1º ANO}};
    
    \node [draw, rotate=90, black,rectangle, minimum width=40pt, minimum
    height=10pt, rounded corners] at (-10pt,0pt) {\footnotesize{1º quad}};
    
    \disciplina [gray!60]{0*\sz}{0pt}{a}{\tresc}{Humanidades 1 \\(3-0-0-4)}
    
    \disciplina [gray!60]{3*\sz}{0pt}{b}{\tresc}{Base Experimental das Ciências Naturais\\ (0-3-0-2)} 
    
    \disciplina [gray!60]{6*\sz}{0pt}{c}{\tresc}{Estrutura da Matéria \\(3-0-0-4)}
    
    \disciplina [gray!60]{9*\sz}{0pt}{d}{\quatc}{Bases Matemáticas\\(4-0-0-5)}
    
    \disciplina [gray!60]{13*\sz}{0pt}{e}{\tresc}{Evolução e Diversificação da Vida na Terra \\(3-0-0-4)}
    
    \disciplina [gray!60]{16*\sz}{0pt}{f}{\doisc}{\scriptsize Bases Computacionais da Ciência \\(0-2-0-2)}
    
    
    %Q2
    \node [draw, rotate=90, black,rectangle, minimum width=40pt, minimum
    height=10pt, rounded corners] at (-10pt,-40pt) {\footnotesize{2º quad}};
    
    \disciplina [gray!60]{0*\sz}{-40pt}{a}{\tresc}{Natureza da Informação\\(3-0-0-4)}
    
    \disciplina [gray!60]{3*\sz}{-40pt}{a}{\cincc}{Fenômenos Mecânicos\\(4-1-0-6)}
    
    \disciplina [gray!60]{8*\sz}{-40pt}{a}{\quatc}{Funções de uma Variável\\(4-0-0-6)}
    
    \disciplina [gray!60]{12*\sz}{-40pt}{a}{\tresc}{Geometria Analítica\\(3-0-0-6)}
    
    \disciplina [gray!60]{15*\sz}{-40pt}{a}{\tresc}{Biodiversidade: Int. Organismos e Ambiente\\(3-0-0-4)}
    
    
    %Q3
    \node [draw, rotate=90, black,rectangle, minimum width=40pt, minimum
    height=10pt, rounded corners] at (-10pt,-80pt) {\footnotesize{3º quad}};
    
    \disciplina [gray!60]{0*\sz}{-80pt}{a}{\quatc}{Processamento da Informação\\(0-4-0-4)}
    
    \disciplina [gray!60]{4*\sz}{-80pt}{a}{\cincc}{Fenômenos Térmicos\\(4-1-0-6)}
    
    \disciplina [gray!60]{9*\sz}{-80pt}{a}{\quatc}{Funções de Várias Variáveis\\(4-0-0-6)}
    
    \disciplina [gray!60]{13*\sz}{-80pt}{a}{\cincc}{Transformações Químicas\\(3-2-0-6)}
    
    
    %Q4
    \node [draw, rotate=90, black,rectangle, minimum width=120pt, minimum
    height=10pt, rounded corners] at (-25pt,-160pt) {\footnotesize{2º ANO}};
    
    \node [draw, rotate=90, black,rectangle, minimum width=40pt, minimum
    height=10pt, rounded corners] at (-10pt,-120pt) {\footnotesize{4º quad}};
    
    \disciplina [gray!60]{0*\sz}{-120pt}{a}{\tresc}{Comunicação e Redes\\(3-0-0-4)}
    
    \disciplina [gray!60]{3*\sz}{-120pt}{a}{\cincc}{Fenômenos Eletromagnéticos\\(4-1-0-6)}
    
    \disciplina [gray!60]{8*\sz}{-120pt}{a}{\quatc}{Introdução às Equações Diferencias Ordinárias\\(4-0-0-4)}
    
    \disciplina [gray!60]{12*\sz}{-120pt}{a}{\tresc}{Introdução à Probabilidade e Estatística\\(3-0-0-4)}
    
    \disciplina [nblue]{15*\sz}{-120pt}{a}{\quatc}{\hyperref[disc:mdI]{Matemática Discreta I}\\(4-0-0-4)}

    
    %Q5
    \node [draw, rotate=90, black,rectangle, minimum width=40pt, minimum
    height=10pt, rounded corners] at (-10pt,-160pt) {\footnotesize{5º quad}};
    
    \disciplina [gray!60]{0*\sz}{-160pt}{a}{\tresc}{Humanidades 2\\(3-0-0-4)}
    
    \disciplina [gray!60]{3*\sz}{-160pt}{a}{\cincc}{Bioquímica: Estrutura, Propriedade e Funções de Biomoléculas\\(3-2-0-6)}
    
    \disciplina [gray!60]{8*\sz}{-160pt}{a}{\quatc}{Física Quântica\\(4-0-0-4)}
    
    \disciplina [nblue]{12*\sz}{-160pt}{a}{\quatc}{\hyperref[disc:pe]{Programação Estruturada}\\(2-2-0-6)}
    

    %Q6
    \node [draw, rotate=90, black,rectangle, minimum width=40pt, minimum
    height=10pt, rounded corners] at (-10pt,-200pt) {\footnotesize{6º quad}};
    
    \disciplina [gray!60]{0*\sz}{-200pt}{a}{\tresc}{Humanidades 3\\(3-0-0-4)}
    
    \disciplina [nblue]{3*\sz}{-200pt}{a}{\quatc}{\hyperref[disc:circ_dig]{Circuitos Digitais}\\(3-1-0-4)}
    
    \disciplina [nblue]{7*\sz}{-200pt}{a}{\seisc}{\hyperref[disc:alge_lin]{Álgebra Linear}\\(6-0-0-5)}
    
    \disciplina [nblue]{13*\sz}{-200pt}{a}{\quatc}{\hyperref[disc:aedI]{Algoritmos e Estruturas de Dados I}\\(2-2-0-6)}
    

    %Q7
    \node [draw, rotate=90, black,rectangle, minimum width=120pt, minimum
    height=10pt, rounded corners] at (-25pt,-280pt) {\footnotesize{3º ANO}};
    
    \node [draw, rotate=90, black,rectangle, minimum width=40pt, minimum
    height=10pt, rounded corners] at (-10pt,-240pt) {\footnotesize{7º quad}};
    
    \disciplina [nblue]{0*\sz}{-240pt}{a}{\quatc}{\hyperref[disc:poo]{Programação Orientada a Objetos}\\(2-2-0-4)}
    
    \disciplina [nblue]{4*\sz}{-240pt}{a}{\quatc}{\hyperref[disc:sist_dig]{Sistemas Digitais}\\(2-2-0-4)}
    
    \disciplina [nblue]{8*\sz}{-240pt}{a}{\quatc}{\hyperref[disc:mdII]{Matemática Discreta II}\\(4-0-0-4)}
    
    \disciplina [nblue]{12*\sz}{-240pt}{a}{\quatc}{\hyperref[disc:aedII]{Algoritmos e Estruturas de Dados II}\\(4-0-0-6)}
    
    
    %Q8
    \node [draw, rotate=90, black,rectangle, minimum width=40pt, minimum
    height=10pt, rounded corners] at (-10pt,-280pt) {\footnotesize{8º quad}};
    
    \disciplina [nyellow]{0*\sz}{-280pt}{a}{\quatc}{Opção Limitada\\(4 créditos)}
    
    \disciplina [nblue]{4*\sz}{-280pt}{a}{\quatc}{\hyperref[disc:arq]{Arquitetura de Computadores}\\(4-0-0-4)}
    
    \disciplina [nblue]{8*\sz}{-280pt}{a}{\quatc}{\hyperref[disc:ia]{Inteligência Artificial}\\(4-0-0-4)}
    
    \disciplina [nblue]{12*\sz}{-280pt}{a}{\quatc}{\hyperref[disc:ag]{Algoritmos em Grafos}\\(4-0-0-4)}
    
    
    %Q9
    \node [draw, rotate=90, black,rectangle, minimum width=40pt, minimum
    height=10pt, rounded corners] at (-10pt,-320pt) {\footnotesize{9º quad}};
    
    \disciplina [nyellow]{0*\sz}{-320pt}{a}{\quatc}{Opção Limitada\\(4 créditos)}
    
    \disciplina [nblue]{4*\sz}{-320pt}{a}{\quatc}{\hyperref[disc:redes]{Redes de Computadores}\\(3-1-0-4)}
    
    \disciplina [nblue]{8*\sz}{-320pt}{a}{\quatc}{\hyperref[disc:es]{Engenharia de Software}\\(4-0-0-4)}
    
    \disciplina [nblue]{12*\sz}{-320pt}{a}{\quatc}{\hyperref[disc:aaI]{Análise de Algoritmos I}\\(4-0-0-4)}
    

    %Q10
    \node [draw, rotate=90, black,rectangle, minimum width=120pt, minimum
    height=10pt, rounded corners] at (-25pt,-400pt) {\footnotesize{4º ANO}};
    
    \node [draw, rotate=90, black,rectangle, minimum width=40pt, minimum
    height=10pt, rounded corners] at (-10pt,-360pt) {\footnotesize{10º quad}};
    
    \disciplina [nyellow]{0*\sz}{-360pt}{a}{\quatc}{Opção Limitada\\(4 créditos)}
    
    \disciplina [nblue]{4*\sz}{-360pt}{a}{\doisc}{\hyperref[disc:ces]{\scriptsize Computadores, Ética e Sociedade}\\(2-0-0-4)}
    
    \disciplina [nblue]{6*\sz}{-360pt}{a}{\quatc}{\hyperref[disc:pf]{Programação Funcional}\\(4-0-0-4)}
    
    \disciplina [nblue]{10*\sz}{-360pt}{a}{\quatc}{\hyperref[disc:aaII]{Análise de Algoritmos II}\\(4-0-0-4))}
    
    
    %Q11
    \node [draw, rotate=90, black,rectangle, minimum width=40pt, minimum
    height=10pt, rounded corners] at (-10pt,-400pt) {\footnotesize{11º quad}};
    
    \disciplina [nyellow]{0*\sz}{-400pt}{a}{\quatc}{Opção Limitada\\(4 créditos)}
    
    \disciplina [nblue]{4*\sz}{-400pt}{a}{\quatc}{\hyperref[disc:mbd]{Modelagem de Banco de Dados}\\(4-0-0-4)}
    
    \disciplina [nblue]{8*\sz}{-400pt}{a}{\quatc}{\hyperref[disc:cg]{Computação Gráfica}\\(0-4-0-4)}
    
    \disciplina [nblue]{12*\sz}{-400pt}{a}{\quatc}{\hyperref[disc:ol]{Otimização Linear}\\(4-0-0-4)}
    
    %Q12
    \node [draw, rotate=90, black,rectangle, minimum width=40pt, minimum
    height=10pt, rounded corners] at (-10pt,-440pt) {\footnotesize{12º quad}};
    
    \disciplina [nyellow]{0*\sz}{-440pt}{a}{\quatc}{Opção Limitada\\(4 créditos)}
    
    \disciplina [nblue]{4*\sz}{-440pt}{a}{\doisc}{\scriptsize Metod.\ e Escr.\ Cient.\ para Comp.\\(2-0-0-2)}
    
    \disciplina [nblue]{6*\sz}{-440pt}{a}{\quatc}{\hyperref[disc:so]{Sistemas Operacionais}\\(3-1-0-4)}
    
    \disciplina [nblue]{10*\sz}{-440pt}{a}{\quatc}{\hyperref[disc:lfa]{Linguagens Formais e Autômatos}\\(4-0-0-4)}
    

    %Q13
    \node [draw, rotate=90, black,rectangle, minimum width=120pt, minimum
    height=10pt, rounded corners] at (-25pt,-520pt) {\footnotesize{5º ANO}};
    
    \node [draw, rotate=90, black,rectangle, minimum width=40pt, minimum
    height=10pt, rounded corners] at (-10pt,-480pt) {\footnotesize{13º quad}};
    
    \disciplina [nyellow]{0*\sz}{-480pt}{a}{\quatc}{Opção Limitada\\(4 créditos)}
    
    \disciplina [nblue]{4*\sz}{-480pt}{a}{\quatc}{Trabalho de Conclusão de Curso I\\(4-0-0-4)}
    
    \disciplina [nblue]{8*\sz}{-480pt}{a}{\quatc}{\hyperref[disc:compi]{Compiladores e Interpretadores}\\(4-0-0-4)}
    
    
    %Q14
    \node [draw, rotate=90, black,rectangle, minimum width=40pt, minimum
    height=10pt, rounded corners] at (-10pt,-520pt) {\footnotesize{14º quad}};
    
    \disciplina [nred]{0*\sz}{-520pt}{a}{\quatc}{Livre\\(4 créditos)}
    
    \disciplina [nblue]{4*\sz}{-520pt}{a}{\quatc}{Trabalho de Conclusão de Curso II\\(4-0-0-6)}
    
    \disciplina [nblue]{8*\sz}{-520pt}{a}{\quatc}{\hyperref[disc:sist_dist]{Sistemas Distribuídos}\\(3-1-0-4)}
    

    %Q15
    \node [draw, rotate=90, black,rectangle, minimum width=40pt, minimum
    height=10pt, rounded corners] at (-10pt,-560pt) {\footnotesize{15º quad}};
    
    \disciplina [nred]{0*\sz}{-560pt}{a}{\quatc}{Livre\\(4 créditos)}

    \disciplina [nblue]{4*\sz}{-560pt}{a}{\quatc}{Trabalho de Conclusão de Curso III\\(4-0-0-6)}
    
    \disciplina [nblue]{8*\sz}{-560pt}{a}{\quatc}{\hyperref[disc:seg]{Segurança de Dados}\\(3-1-0-4)}
    
    
    \draw [red,thick,dashed] (20*\sz,20pt) -- (20*\sz,-580pt); 
    \node [text=red] at (420pt,35pt) {20 créditos};


    % Legenda
    \node [text=black] at (-20pt, -620pt) {\textbf{Legenda:}};
    \disciplina [gray!60]{15pt}{-620pt}{bis0505}{63}{Obrigatória BC\&T\\(T-P-E-I)}
    
    \disciplina [nblue]{88pt}{-620pt}{bis0505}{63}{Obrigatória BCC\\(T-P-E-I)}
    
    \disciplina [nyellow]{161pt}{-620pt}{bis0505}{63}{Opção Limitada BCC\\(T-P-E-I)}
    
    \disciplina [nred]{234pt}{-620pt}{bis0505}{63}{Livre BCC\\(T-P-E-I)}
    
    
\end{tikzpicture}


\end{document}
}
    \caption{Matriz curricular recomendada para integralização do BC\&T e do
    BCC em 5 anos. A largura dos retângulos é diretamente proporcional à
    quantidade de créditos em sala de aula.}
    \label{fig:matriz_curricular}
\end{figure}

\begin{figure}
    \centering
    \resizebox{\textwidth}{!}{
\documentclass{standalone}
\usepackage{amsmath, amsthm, amsfonts, amssymb}
\usepackage{tikz}
\usetikzlibrary{shapes,snakes,positioning,calc}

\newcommand{\disciplina}[6][nyellow]{
\node [draw=black, fill=#1, very thick, rectangle, rounded corners, inner
sep=2pt, inner ysep=2pt, text width=#5pt, minimum height=40pt,
text centered, anchor=west] (#4) at (#2,#3) {
   \begin{minipage}{#5pt}
   \linespread{1.0}\selectfont
       \centering
       \scriptsize{#6}
   \end{minipage}
};
}
\newcommand{\disciplinab}[6][yellow!20]{
\node [draw=black, fill=#1, very thick, rectangle, rounded corners, inner
sep=2pt, inner ysep=2pt, text width=#5pt, minimum height=40pt,
text centered, anchor=west] (#4) at (#2,#3) {
   \begin{minipage}{#5pt}
   \linespread{1.0}\selectfont
       \centering
       \small{#6}
   \end{minipage}
};
}


\usepackage{xcolor}
\definecolor{nred}{rgb}{0.88, 0.28, 0.33}
\definecolor{nblue}{rgb}{0.34, 0.74, 0.96}
\definecolor{nyellow}{rgb}{0.8, 0.86, 0.38}
\definecolor{npurple}{rgb}{0.25, 0.34, 0.93}
\definecolor{ngreen}{rgb}{0.45, 0.95, 0.66}


\begin{document}

\begin{tikzpicture}

    \def\sz{21pt}
    \def\doisc{36} % 2*\sz - 6
    \def\tresc{57} % 3*\sz - 6
    \def\quatc{78} % 4*\sz - 6
    \def\cincc{99} % 5*\sz - 6
    \def\seisc{120} % 6*\sz - 6
    \def\h{40pt}


    \disciplinab [nblue]{-1*\sz}{0*\h}{PE}{\tresc}{Programação Estruturada}
    \disciplinab [nblue]{-1*\sz}{-2*\h}{AEDI}{\tresc}{Algoritmos e Estruturas de Dados I}
    \disciplinab [nblue]{-1*\sz}{-4*\h}{AEDII}{\tresc}{Algoritmos e Estruturas de Dados II}
    \disciplinab [nblue]{-1*\sz}{-6*\h}{AG}{\tresc}{Algoritmos em Grafos}
    \disciplinab [nblue]{-1*\sz}{-8*\h}{AAI}{\tresc}{Análise de Algoritmos I}
    \disciplinab [nblue]{-1*\sz}{-10*\h}{AAII}{\tresc}{Análise de Algoritmos II}
    \disciplinab [nblue]{-1*\sz}{-12*\h}{LFA}{\tresc}{Linguagens Formais e Autômatos}
    \disciplinab [nblue]{-1*\sz}{-14*\h}{COMP}{\tresc}{Compiladores e Interpretadores}

    \disciplinab [nblue]{16*\sz}{-1*\h}{CG}{\tresc}{Computação Gráfica}
    \disciplinab [nblue]{5*\sz}{-1*\h}{IHC}{\tresc}{Interação Humano-Computador}
%    \disciplinab [nblue]{5*\sz}{-1*\h}{CG}{\tresc}{Computação Gráfica}
    \disciplinab [nblue]{5*\sz}{-3*\h}{PF}{\tresc}{Programação Funcional}
    \disciplinab [nblue]{5*\sz}{-5*\h}{IA}{\tresc}{Inteligência Artificial}
    \disciplinab [nblue]{5*\sz}{-8*\h}{ARQ}{\tresc}{Arquitetura de Computadores}
    \disciplinab [nblue]{5*\sz}{-10*\h}{SO}{\tresc}{Sistemas Operacionais}
    \disciplinab [nblue]{5*\sz}{-12*\h}{MDII}{\tresc}{Matemática Discreta II}

    \disciplinab [nblue]{10*\sz}{-4*\h}{CD}{\tresc}{Circuitos Digitais}
    \disciplinab [nblue]{10*\sz}{-6*\h}{SD}{\tresc}{Sistemas Digitais}
    \disciplinab [nblue]{10*\sz}{-10*\h}{MD}{\tresc}{Matemática Discreta I}

    \disciplinab [nblue]{15*\sz}{-14*\h}{AL}{\tresc}{Álgebra Linear}

    \disciplinab [nblue]{20*\sz}{0*\h}{POO}{\tresc}{Programação Orientada a Objetos}
    \disciplinab [nblue]{20*\sz}{-2*\h}{RD}{\tresc}{Redes de Computadores}
    \disciplinab [nblue]{20*\sz}{-4*\h}{SDT}{\tresc}{Sistemas Distribuídos}

    \disciplinab [nblue]{25*\sz}{0*\h}{ES}{\tresc}{Engenharia de Software}
    \disciplinab [nblue]{25*\sz}{-2*\h}{BD}{\tresc}{Modelagem de Banco de Dados}
    \disciplinab [nblue]{25*\sz}{-4*\h}{SEG}{\tresc}{Segurança de Dados}

    \disciplinab [nblue]{20*\sz}{-14*\h}{CES}{\tresc}{Computadores, Ética e Sociedade}
    \disciplinab [nblue]{25*\sz}{-14*\h}{MET}{\tresc}{Metod. e Escr. Cient. para Comp.}

    \disciplinab [gray!50]{-1*\sz}{2*\h}{PI}{\tresc}{Processamento da Informação}
    \disciplinab [gray!50]{5*\sz}{2*\h}{BCC}{\tresc}{Bases Computacionais da Ciência}
    \disciplinab [gray!50]{10*\sz}{2*\h}{NI}{\tresc}{Natureza da Informação}
    \disciplinab [gray!50]{15*\sz}{-4*\h}{FEMAG}{\tresc}{Fenômenos Eletromagnéticos}
    \disciplinab [gray!50]{15*\sz}{-8*\h}{FEMEC}{\tresc}{Fenômenos Mecânicos}
    \disciplinab [gray!50]{15*\sz}{-10*\h}{FUV}{\tresc}{Funções de Uma Variável}
    \disciplinab [gray!50]{20*\sz}{-10*\h}{GA}{\tresc}{Geometria Analítica}
    \disciplinab [gray!50]{15*\sz}{-12*\h}{BM}{\tresc}{Bases Matemáticas}
    \disciplinab [gray!50]{10*\sz}{-12*\h}{FVV}{\tresc}{Funções de Várias Variáveis}

    \draw[seta] (AEDI) -- (PE);
    \draw[seta] (AEDII) -- (AEDI);
    \draw[seta] (AG) -- (AEDII);
    \draw[seta] (AAI) -- (AG);
    \draw[seta] (AAII) -- (AAI);
    \draw[seta] (LFA) -- (AAII);
    \draw[seta] (AG) -- (MDII);
    \draw[seta] (MDII) -- (MD);
    %\draw[seta] (OL) -- (AAII);
    %\draw[seta] (OL) -- (AL);
    \draw[seta] (COMP) -- (LFA);
    \draw[seta] (CG) -- (AEDI);
    \draw[seta] (CG) -- (POO);
    \draw[seta] (POO) -- (PE);
    \draw[seta] (PF) -- (AEDI);
    \draw[seta] (IA) -- (AEDI);
    \draw[seta] (ARQ) -- (AEDI);
    \draw[seta] (ARQ) -- (SD);
    \draw[seta] (SD) -- (CD);
    \draw[seta] (SO) -- (ARQ);
    \draw[seta] (SO) -- (AEDII);
    \draw[seta] (ES) -- (POO);
    \draw[seta] (BD) -- (POO);
    \draw[seta] (RD) -- (POO);
    \draw[seta] (RD) -- (AEDI);
    \draw[seta] (SDT) -- (RD);
    \draw[seta] (SDT) -- (SO);
    \draw[seta] (SEG) -- (RD);

    \draw[seta,gray] (PE) -- (PI);
    \draw[seta,gray] (PI) -- (BCC);
    \draw[seta,gray] (NI) -- (BCC);
    \draw[seta,gray] (CD) -- (NI);
    \draw[seta,gray] (CD) -- (FEMAG);
    \draw[seta,gray] (FEMAG) -- (FEMEC);
    \draw[seta,gray] (FEMEC) -- (GA);
    \draw[seta,gray] (FEMEC) -- (FUV);
    \draw[seta,gray] (GA) -- (BM);
    \draw[seta,gray] (CG) -- (GA);
    \draw[seta,gray] (FUV) -- (BM);
    \draw[seta,gray] (AL) -- (GA);
    \draw[seta,gray] (MD) -- (FUV);
    %\draw[seta,gray] (OL) -- (FVV);
    \draw[seta,gray] (FVV) -- (FUV);
    \draw[seta,gray] (IHC) -- (PI);

\end{tikzpicture}


\end{document}
}
    \caption{Recomendações entre as disciplinas obrigatórias do BCC e do BC\&T.}
    \label{fig:recomendacoes}
\end{figure}

As regras de transição entre a matriz sugerida do PPC anterior e a matriz
sugerida do PPC atual são apresentadas na Seção~\ref{sec:regras_transicao}.

\subsection{Mapeamento de habilidades/competências e atividades pedagógicas}

A organização curricular foi desenhada para atender aos requisitos estruturais
e pedagógicos da UFABC, bem como às Diretrizes Curriculares Nacionais dos
cursos de graduação em Computação (Resolução CNE/CES nº 5, de 16 de novembro de
2016).
Na Tabela~\ref{tab:mapeamento_competencias}, indicamos os componentes
pedagógicos que contribuem para a formação e consolidação das habilidades e
competências dos egressos.
As atividades pedagógicas estão classificadas da seguinte forma:

\begin{itemize}
    \item \textcolor{nred}{Disciplinas obrigatórias do BC\&T}
    \item \textcolor{nblue}{Disciplinas obrigatórias do BCC}
    \item \textcolor{nyellow}{Disciplinas de opção limitada do BCC}
    \item \textcolor{npurple}{Outras ações}
\end{itemize}


\begin{longtable}{|p{.3\textwidth}p{.3\textwidth}p{.25\textwidth}|}
    \caption{Componentes pedagógicos e suas contribuições de acordo com as habilidades e competências dos egressos.}
    \label{tab:mapeamento_competencias}
    \endfirsthead
    \endhead

    \multicolumn{3}{p{0.95\textwidth}}{Identificar problemas que tenham solução algorítmica}\\
    \hline
    \textcolor{nred}{Bases Comput. da Ciência}  &
    \textcolor{nblue}{Algs. e Estruturas de Dados I} &
    \textcolor{nblue}{Otimização Linear} \\
    \textcolor{nred}{Processamento da Informação} &
    \textcolor{nblue}{Algs. e Estruturas de Dados II} &
    \textcolor{nblue}{Algs. em Grafos} \\
    \textcolor{nblue}{Progr. Estruturada} &
    \textcolor{nblue}{Análise de Algoritmos I} &
    \textcolor{nblue}{Inteligência Artificial} \\
    \textcolor{nblue}{Matemática Discreta} &
    \textcolor{nblue}{Análise de Algoritmos II} &
    \textcolor{nblue}{Progr. Funcional} \\
    \textcolor{nblue}{Matemática Discreta II} &
    \textcolor{nblue}{Progr. Orientada a Objetos} & \\

    \hline
    \multicolumn{3}{p{0.95\textwidth}}{}\\
    
    \multicolumn{3}{p{0.95\textwidth}}{Conhecer os limites da computação}\\
    \hline
    \textcolor{nblue}{Análise de Algoritmos I} &
    \textcolor{nblue}{Análise de Algoritmos II} &
    \textcolor{nblue}{Ling. Formais e Autômatos} \\

    \hline
    \multicolumn{3}{p{0.95\textwidth}}{}\\

    \multicolumn{3}{p{0.95\textwidth}}{Resolver problemas usando ambientes de programação}\\
    \hline
    \textcolor{nred}{Processamento da Informação} &
    \textcolor{nblue}{Algs. em Grafos} &
    \textcolor{nblue}{Engenharia de Software} \\
    \textcolor{nblue}{Progr. Estruturada} &
    \textcolor{nblue}{Inteligência Artificial} &
    \textcolor{nblue}{Otimização Linear}\\
    \textcolor{nblue}{Algs. e Estruturas de Dados I} &
    \textcolor{nblue}{Progr. Orientada a Objetos} &
    \textcolor{nblue}{Progr. Funcional}\\
    \textcolor{nblue}{Algs. e Estruturas de Dados II} &
    \textcolor{nblue}{\small Model. de Banco de Dados} &
    \textcolor{nblue}{Sistemas Digitais}\\
    \textcolor{nblue}{Compiladores e Interpretadores} & & \\
    \hline
    
    \multicolumn{3}{p{0.95\textwidth}}{}\\

    \multicolumn{3}{p{0.95\textwidth}}{Tomar decisões e inovar, com base no
    conhecimento do funcionamento e das características técnicas de hardware
    e da infraestrutura de software dos sistemas de computação consciente dos
    aspectos éticos, legais e dos impactos ambientais decorrentes}\\
    \hline
    \textcolor{nred}{Ciência, Tecnologia e Sociedade} &
    \textcolor{nblue}{Redes de Computadores} &
    \textcolor{nblue}{Segurança de Dados}\\
    \textcolor{nred}{Comunicação e Redes} &
    \textcolor{nblue}{Sistemas Operacionais} &
    \textcolor{nblue}{\small Model. de Banco de Dados}\\
    \textcolor{nblue}{Arquitetura de Computadores} &
    \textcolor{nblue}{Sistemas Distribuídos} &
    \textcolor{nblue}{Engenharia de Software} \\
    \textcolor{nblue}{Sistemas Digitais} &
    \textcolor{nblue}{Comput., Ética e Sociedade} & \\
    \hline
    
    \multicolumn{3}{p{0.95\textwidth}}{}\\
    \multicolumn{3}{p{0.95\textwidth}}{Compreender e explicar as dimensões quantitativas de um
    problema}\\
    \hline
    \textcolor{nred}{Natureza da Informação} &
    \textcolor{nred}{Geometria Analítica} &
    \textcolor{nblue}{Análise de Algoritmos I} \\
    \textcolor{nred}{Intr. à Probab. e Estatística} &
    \textcolor{nblue}{Álgebra Linear} &
    \textcolor{nblue}{Análise de Algoritmos II} \\
    \textcolor{nred}{Funções de Uma Variável} &
    \textcolor{nblue}{Matemática Discreta} &
    \textcolor{nblue}{Ling. Formais e Autômatos} \\
    \textcolor{nred}{Funções de Várias Variáveis} &
    \textcolor{nblue}{Matemática Discreta II} &
    \textcolor{nblue}{Engenharia de Software}\\
    \textcolor{nblue}{Otimização Linear} & & \\
    \hline
    
    \multicolumn{3}{p{0.95\textwidth}}{}\\
    \multicolumn{3}{p{0.95\textwidth}}{Gerir a sua própria aprendizagem e
    desenvolvimento, incluindo a gestão de tempo e competências
    organizacionais}\\
    \hline
    \textcolor{nblue}{Trab. de Concl. de Curso I} &
    \textcolor{nblue}{Trab. de Concl. de Curso II} &
    \textcolor{nblue}{Trab. de Concl. de Curso III} \\
    \hline
    
    \multicolumn{3}{p{0.95\textwidth}}{}\\
    \multicolumn{3}{p{0.95\textwidth}}{Preparar e apresentar seus trabalhos e
    problemas técnicos e suas soluções para audiências diversas, em formatos
    apropriados (oral e escrito)}\\
    \hline
    \textcolor{nblue}{\small Metod. e Escr. Cient. para Comp.} &
    \textcolor{nblue}{Trab. de Concl. de Curso I} &
    \textcolor{nblue}{Trab. de Concl. de Curso II}\\
    \textcolor{nblue}{Engenharia de Software} &
    \textcolor{nblue}{Trab. de Concl. de Curso III} & \\
    \hline
    
    \multicolumn{3}{p{0.95\textwidth}}{}\\
    \multicolumn{3}{p{0.95\textwidth}}{Avaliar criticamente projetos de sistemas de computação}\\
    \hline
    \textcolor{nred}{Ciência, Tecnologia e Sociedade} &
    \textcolor{nblue}{Segurança de Dados} &
    \textcolor{nblue}{Análise de Algoritmos I} \\
    \textcolor{nblue}{Engenharia de Software} &
    \textcolor{nblue}{Comput., Ética e Sociedade} &
    \textcolor{nblue}{Análise de Algoritmos II} \\
    \textcolor{nblue}{Sistemas Distribuídos} & 
    \textcolor{nblue}{Redes de Computadores} & \\
    \hline
    
    \multicolumn{3}{p{0.95\textwidth}}{}\\
    \multicolumn{3}{p{0.95\textwidth}}{Adequar-se rapidamente às mudanças
    tecnológicas e aos novos ambientes de trabalho}\\
    \hline
    \textcolor{nblue}{Comput., Ética e Sociedade} & & \\
    \hline
    
    \multicolumn{3}{p{0.95\textwidth}}{}\\
    \multicolumn{3}{p{0.95\textwidth}}{Ler textos técnicos na língua inglesa}\\
    \hline
    \textcolor{nblue}{\small Metod. e Escr. Cient. para Comp.} &
    \textcolor{nblue}{Trab. de Concl. de Curso I} &
    \textcolor{nblue}{Trab. de Concl. de Curso II} \\
    \textcolor{nblue}{Trab. de Concl. de Curso III} & & \\
    \hline
    
    \multicolumn{3}{p{0.95\textwidth}}{}\\
    \multicolumn{3}{p{0.95\textwidth}}{Empreender e exercer liderança, coordenação
    e supervisão na sua área de atuação profissional}\\
    \hline
    \textcolor{npurple}{Ações de extensão} && \\
    \hline
    
    \multicolumn{3}{p{0.95\textwidth}}{}\\
    \multicolumn{3}{p{0.95\textwidth}}{Ser capaz de realizar trabalho cooperativo e entender os
    benefícios que este pode produzir}\\
    \hline
    \textcolor{nred}{Ciência, Tecnologia e Sociedade} &
    \textcolor{nblue}{Comput., Ética e Sociedade} &
    \textcolor{nblue}{Engenharia de Software}\\
    \hline
    
    \multicolumn{3}{p{0.95\textwidth}}{}\\
    \multicolumn{3}{p{0.95\textwidth}}{Compreender os fatos essenciais, os
    conceitos, os princípios e as teorias relacionadas à Ciência da Computação
    para o desenvolvimento de software e hardware e suas aplicações}\\
    \hline
    \textcolor{nred}{Bases Comput. da Ciência} &
    \textcolor{nblue}{Progr. Funcional} & 
    \textcolor{nblue}{Algs. em Grafos} \\
    \textcolor{nred}{Processamento da Informação} &
    \textcolor{nblue}{Algs. e Estruturas de Dados I} &
    \textcolor{nblue}{Otimização Linear}\\
    \textcolor{nblue}{Progr. Estruturada} & 
    \textcolor{nblue}{Algs. Estruturas de Dados II} &
    \textcolor{nblue}{Circuitos Digitais}\\
    \textcolor{nblue}{Análise de Algoritmos I} &
    \textcolor{nblue}{Ling. Formais e Autômatos} &
    \textcolor{nblue}{Sistemas Digitais}\\
    \textcolor{nblue}{Análise de Algoritmos II} &
    \textcolor{nblue}{Matemática Discreta} &
    \textcolor{nblue}{Sistemas Operacionais}\\
    \textcolor{nblue}{Arquitetura de Computadores} & 
    \textcolor{nblue}{Matemática Discreta II} & \\
    \hline
    
    \multicolumn{3}{p{0.95\textwidth}}{}\\
    \multicolumn{3}{p{0.95\textwidth}}{Reconhecer a importância do pensamento
    computacional no cotidiano e sua aplicação em circunstâncias apropriadas e
    em domínios diversos}\\
    \hline
    \textcolor{nred}{Comunicação e Redes} &
    \textcolor{nblue}{Progr. Estruturada} &
    \textcolor{nblue}{Matemática Discreta} \\
    \textcolor{nred}{Processamento da Informação} &
    \textcolor{nblue}{Algs. e Estruturas de Dados I}&
    \textcolor{nblue}{Matemática Discreta II}\\
    \textcolor{nred}{Ciência, Tecnologia e Sociedade} &
    \textcolor{nblue}{Algs. e Estruturas de Dados II}&
    \textcolor{nblue}{Algs. em Grafos}\\
    \textcolor{nred}{Bases Comput. da Ciência} &
    \textcolor{nblue}{Comput., Ética e Sociedade} &
    \textcolor{nblue}{Otimização Linear}\\
    \hline
    
    \multicolumn{3}{p{0.95\textwidth}}{}\\
    \multicolumn{3}{p{0.95\textwidth}}{Identificar e gerenciar os riscos que podem
    estar envolvidos na operação de equipamentos de computação (incluindo os
    aspectos de dependabilidade e segurança)}\\
    \hline
    \textcolor{nblue}{Segurança de Dados} &
    \textcolor{nblue}{Comput., Ética e Sociedade} &
    \textcolor{nblue}{Circuitos Digitais}\\
    \textcolor{nblue}{\small Model. de Banco de Dados} &
    \textcolor{nblue}{Redes de Computadores} & \\
    \hline
    
    \multicolumn{3}{p{0.95\textwidth}}{}\\
    \multicolumn{3}{p{0.95\textwidth}}{Identificar e analisar requisitos e
    especificações para problemas específicos e planejar estratégias para suas
    soluções}\\
    \hline
    \textcolor{nblue}{Engenharia de Software} &
    \textcolor{nblue}{Arquitetura de Computadores} &
    \textcolor{nblue}{Circuitos Digitais} \\
    \textcolor{nblue}{Análise de Algoritmos I} &
    \textcolor{nblue}{\small Model. de Banco de Dados} &
    \textcolor{nblue}{Sistemas Digitais} \\
    \textcolor{nblue}{Análise de Algoritmos II} & & \\
    \hline
    
    \multicolumn{3}{p{0.95\textwidth}}{}\\
    \multicolumn{3}{p{0.95\textwidth}}{Especificar, projetar, implementar, manter e
    avaliar sistemas de computação, empregando teorias, práticas e ferramentas
    adequadas}\\
    \hline
    \textcolor{nblue}{Sistemas Operacionais} &
    \textcolor{nblue}{Arquitetura de Computadores} &
    \textcolor{nblue}{Análise de Algoritmos I} \\
    \textcolor{nblue}{\small Model. de Banco de Dados} &
    \textcolor{nblue}{Redes de Computadores} &
    \textcolor{nblue}{Análise de Algoritmos II} \\
    \textcolor{nblue}{Engenharia de Software} &
    \textcolor{nblue}{Sistemas Distribuídos} &
    \textcolor{nblue}{Sistemas Digitais}\\
    \textcolor{nblue}{Compiladores e Interpretadores} & & \\
    \hline
    
    \multicolumn{3}{p{0.95\textwidth}}{}\\
    \multicolumn{3}{p{0.95\textwidth}}{Conceber soluções computacionais a partir de
    decisões visando o equilíbrio de todos os fatores envolvidos}\\
    \hline
    \textcolor{nred}{Processamento da Informação} &
    \textcolor{nblue}{Algs. e Estruturas de Dados I}&
    \textcolor{nblue}{Segurança de Dados}\\
    \textcolor{nred}{Ciência, Tecnologia e Sociedade} &
    \textcolor{nblue}{Algs. e Estruturas de Dados II}&
    \textcolor{nblue}{Engenharia de Software}\\
    \textcolor{nblue}{Progr. Estruturada} &
    \textcolor{nblue}{Comput., Ética e Sociedade}&
    \textcolor{nblue}{Prog. Funcional} \\
    \textcolor{nblue}{Inteligência Artificial} &
    \textcolor{nblue}{Compiladores e Interpretadores} & \\
    \hline
    
    \multicolumn{3}{p{0.95\textwidth}}{}\\
    \multicolumn{3}{p{0.95\textwidth}}{Empregar metodologias que visem garantir
    critérios de qualidade ao longo de todas as etapas de desenvolvimento de
    uma solução computacional}\\
    \hline
    \textcolor{nblue}{Engenharia de Software} & & \\
    \hline
    
    \multicolumn{3}{p{0.95\textwidth}}{}\\
    \multicolumn{3}{p{0.95\textwidth}}{Analisar quanto um sistema baseado em
    computadores atende os critérios definidos para seu uso corrente e futuro
    (adequabilidade)}\\
    \hline
    \textcolor{nblue}{Engenharia de Software} &
    \textcolor{nblue}{Análise de Algoritmos I} &
    \textcolor{nblue}{\small Model. de Banco de Dados}\\
    \hline
    
    \multicolumn{3}{p{0.95\textwidth}}{}\\
    \multicolumn{3}{p{0.95\textwidth}}{Gerenciar projetos de desenvolvimento de
    sistemas computacionais}\\
    \hline
    \textcolor{nblue}{Compiladores e Interpretadores} &
    \textcolor{nblue}{Sistemas Operacionais}&
    \textcolor{nblue}{\small Model. de Banco de Dados}\\
    \textcolor{nblue}{Engenharia de Software} &
    \textcolor{nyellow}{Gestão de Projetos de Software} & \\
    \hline
    
    \multicolumn{3}{p{0.95\textwidth}}{}\\
    \multicolumn{3}{p{0.95\textwidth}}{Aplicar temas e princípios recorrentes, como
    abstração, complexidade, princípio de localidade de referência (caching),
    compartilhamento de recursos, segurança, concorrência, evolução de
    sistemas, entre outros, e reconhecer que esses temas e princípios são
    fundamentais à área de Ciência da Computação}\\
    \hline
    \textcolor{nblue}{Análise de Algoritmos I} &
    \textcolor{nblue}{Progr. Orientada a Objetos} &
    \textcolor{nblue}{\small Model. de Banco de Dados}\\
    \textcolor{nblue}{Análise de Algoritmos II} &
    \textcolor{nblue}{Segurança de Dados} &
    \textcolor{nblue}{Sistemas Distribuídos}\\
    \textcolor{nblue}{Redes de Computadores} & 
    \textcolor{nblue}{Arquitetura de Computadores} &
    \textcolor{nblue}{Sistemas Operacionais}\\
    \textcolor{nblue}{Inteligência Artificial} & 
    \textcolor{nblue}{Engenharia de Software} &
    \textcolor{nblue}{Sistemas Digitais}\\
    \hline
    
    
    \multicolumn{3}{p{0.95\textwidth}}{}\\
    \multicolumn{3}{p{0.95\textwidth}}{Escolher e aplicar boas práticas e técnicas
    que conduzam ao raciocínio rigoroso no planejamento, na execução e no
    acompanhamento, na medição e gerenciamento geral da qualidade de sistemas
    computacionais}\\
    \hline
    \textcolor{nblue}{Análise de Algoritmos I} &
    \textcolor{nblue}{Comput., Ética e Sociedade} &
    \textcolor{nblue}{\small Model. de Banco de Dados}\\
    \textcolor{nblue}{Análise de Algoritmos II} &
    \textcolor{nyellow}{Sistemas de Informação} &
    \textcolor{nblue}{Sistemas Digitais}\\
    \textcolor{nblue}{Engenharia de Software} &
    \textcolor{nblue}{Compiladores e Interpretadores} & \\
    \hline
    
    
    \multicolumn{3}{p{0.95\textwidth}}{}\\
    \multicolumn{3}{p{0.95\textwidth}}{Aplicar os princípios de gerência,
    organização e recuperação da informação de vários tipos, incluindo texto,
    imagem, som e vídeo}\\
    \hline
    \textcolor{nblue}{\small Model. de Banco de Dados} &
    \textcolor{nblue}{Algs. e Estruturas de Dados I} & 
    \textcolor{nblue}{Sistemas Distribuídos}\\
    \textcolor{nblue}{Computação Gráfica} &
    \textcolor{nblue}{Algs. e Estruturas de Dados II} &
    \textcolor{nyellow}{Proc. de Sinais Neurais} \\
    \textcolor{nblue}{Redes de Computadores} & 
    \textcolor{nyellow}{Process. Digital de Imagens} & \\
    \hline
    
    \multicolumn{3}{p{0.95\textwidth}}{}\\
    \multicolumn{3}{p{0.95\textwidth}}{Aplicar os princípios de interação
    humano-computador para avaliar e construir uma grande variedade de
    produtos, incluindo interface de usuário, páginas WEB, sistemas multimídia
    e sistemas móveis}\\
    \hline
    \textcolor{nblue}{Computação Gráfica} & 
    \textcolor{nyellow}{\small Interação Humano-Computador} &
    \textcolor{nyellow}{Sistemas Multimidia}\\
    \textcolor{nyellow}{Programação para Web} &
    \textcolor{nyellow}{\small Prog. Av. de Dispositivos Móveis} & 
    \textcolor{nyellow}{Visão Computacional}\\
    \textcolor{nyellow}{Sistemas Inteligentes} & & \\
    \hline
    
\end{longtable}
