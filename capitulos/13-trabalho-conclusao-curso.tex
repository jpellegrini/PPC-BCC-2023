\section{Trabalho de conclusão de curso}
\label{sec:trabalho_conclusao_curso}

Segundo as DCNs da área de Computação, o Trabalho de Conclusão de Curso (TCC) é
uma atividade de síntese, integração e aplicação de conhecimentos de caráter
científico ou tecnológico. 
O TCC do BCC consiste em uma pesquisa orientada, apresentada sob forma de texto
científico, cobrindo algum tema necessariamente relacionado à área de Ciência
da Computação. 
O aluno deve mostrar capacidade de avaliar a teoria/tecnologia existente de
maneira crítica, bem como de buscar novas tecnologias de forma independente.
Portanto, o TCC não pode se configurar como uma mera aplicação direta dos
métodos e tecnologias abordadas no curso, mas sim uma experiência na qual o
aluno deve revelar seu domínio da área de Computação e sua capacidade de buscar
soluções criativas para problemas relevantes e não triviais.


No BCC, o TCC é um instrumento \textit{obrigatório} para integralização do
curso, representado pelo aproveitamento com direito a créditos, das disciplinas
TCC I, TCC II e TCC III.
Por se tratar de uma atividade de síntese, recomenda-se fortemente que seja
desenvolvida no final do curso, quando o estudante já terá uma experiência
curricular consolidada em relação à matriz do curso.

A regulamentação, critérios de matrícula, procedimentos, mecanismo de avaliação
e diretrizes do TCC são regidos pela \red{Resolução no XXXXX/XXXX} ou documento
mais recente que a substitua. 
